\usepackage{luatex85}
\def\pgfsysdriver{pgfsys-pdftex.def}
\usepackage[utf8]{luainputenc}
\usepackage{fontspec}

%\usepackage[utf8]{inputenc}
%\usepackage{fontenc}

\usepackage[german]{babel}
\usepackage{amsmath}
\usepackage{amsfonts}
\usepackage{amssymb}
\usepackage{amsthm}
\usepackage{graphicx}
\usepackage{tikz,pgf}
\usetikzlibrary{cd}
\usetikzlibrary{babel}
\usepackage{mathrsfs}
\usepackage{mathtools}
\usepackage{framed}
\usepackage{ulem}
\usepackage{tabularx}
\usepackage{csquotes}
\usepackage{dsfont}
\usepackage{enumitem}
\usepackage{wrapfig}
\usepackage{chngcntr}
\usepackage{makeidx}
\usepackage[hidelinks]{hyperref}
\usepackage{etoolbox}
\usepackage{environ}
\usepackage{xifthen}

\setlist[enumerate,1]{label=\alph*)}
\setlist[enumerate,2]{label=(\roman*)}

\newcounter{exmlistitem}
\newenvironment{exmlist}
{\let\oldtheorem\thetheorem
	\renewcommand{\thetheorem}{\oldtheorem.\arabic{exmlistitem}}
	\stepcounter{theorem}
	\let\oldexm\exm
	\let\oldendexm\endexm
	\renewenvironment{exm}{\addtocounter{theorem}{-1}\stepcounter{exmlistitem}\oldexm}{\oldendexm}}
{\setcounter{exmlistitem}{0}}

\counterwithin*{equation}{section}
\newcommand{\autotag}{\stepcounter{equation}\tag{Gl. \arabic{section}.\arabic{equation}}}
\newcommand{\indexed}[1]{\textbf{#1}\index{#1}}

\newcommand{\N}{\ensuremath{\mathbb{N}}}
\newcommand{\Z}{\ensuremath{\mathbb{Z}}}
\newcommand{\Q}{\ensuremath{\mathbb{Q}}}
\newcommand{\R}{\ensuremath{\mathbb{R}}}
\newcommand{\C}{\ensuremath{\mathbb{C}}}
\newcommand{\F}{\ensuremath{\mathbb{F}}}
\newcommand{\A}{\ensuremath{\mathbb{A}}}

\newcommand{\la}{\ensuremath{\lambda}}
\newcommand{\al}{\ensuremath{\alpha}}
\newcommand{\ol}[1]{\overline{#1}}
\newcommand{\ul}[1]{\underline{#1}}
\newcommand{\todomark}[1]{\fbox{\Large Hier könnte \sout{Ihre Werbung} #1 stehen}}
\newcommand{\abs}[1]{\left|#1\right|}
\newcommand{\norm}[1]{\left\|#1\right\|}
\newcommand{\scp}[1]{\left\langle#1\right\rangle}
\newcommand{\mapsfrom}{\ensuremath{\mathrel{\reflectbox{\mapsto}}}}
\newcommand{\cha}{\operatorname{char}}
\newcommand{\bigtimes}{\operatornamewithlimits{\times}}
\newcommand{\lcm}{\operatorname{lcm}}
\newcommand{\id}{\operatorname{id}}
\renewcommand{\projlim}[1]{\lim\limits_{\overleftarrow{#1}}}
\newcommand{\isom}{\overset{\sim}{=}}
\newcommand{\ord}{\operatorname{ord}}
\newcommand{\weight}{\operatorname{gew}}
\newcommand{\grp}[1]{\left\langle#1\right\rangle}
\newcommand{\rad}{\operatorname{rad}}

\newcommand{\Kern}{\operatorname{Kern}}
\newcommand{\Img}{\operatorname{Im}}
\newcommand{\Potset}{\mathscr P}
\newcommand{\Hom}{\operatorname{Hom}}
\newcommand{\Aut}{\operatorname{Aut}}
\newcommand{\Fr}{\operatorname{Fr}}

\newcommand{\scA}{\mathscr A}
\newcommand{\scB}{\mathscr B}
\newcommand{\scE}{\mathscr E}
\newcommand{\scF}{\ensuremath{\mathscr{F}}}
\newcommand{\scG}{\mathscr G}
\newcommand{\scL}{\mathscr L}
\newcommand{\scO}{\mathscr O}

\theoremstyle{plain}
\newtheorem{theorem}{Theorem}[section]
\newtheorem{lem}[theorem]{Lemma}
\newtheorem{kor}[theorem]{Korollar}
\newtheorem{satz}[theorem]{Satz}
\newtheorem{rem}[theorem]{Erinnerung}
\newtheorem*{rem*}{Erinnerung}

\theoremstyle{definition}
\newtheorem{definition}[theorem]{Definition}
\newtheorem{prop}[theorem]{Proposition}

\theoremstyle{remark}
\newtheorem{bem}[theorem]{Bemerkung}
\newtheorem*{bem*}{Bemerkung}
\newtheorem{exm}[theorem]{Beispiel}
\newtheorem*{exm*}{Beispiel}

\NewEnviron{killcontent}{$~$}
\newcommand{\hideenv}[2][]{
	\expandafter\renewcommand\csname#2\endcsname{\ifthenelse{\isempty{#1}}{ }{\stepcounter{#1} }\killcontent}
	\expandafter\renewcommand\csname end#2\endcsname{\endkillcontent\vspace{-0.6\baselineskip}}
}



%\hideenv[theorem]{definition}
%\hideenv[theorem]{kor}
%\hideenv[theorem]{lem}
%\hideenv[theorem]{prop}
%\hideenv[theorem]{satz}
%\hideenv[theorem]{theorem}
%\hideenv{proof}
%\hideenv[theorem]{exm}
%\hideenv[theorem]{bem}

\title{Algebra WiSe 17/18}
\author{Prof. Scheithauer\\
Mitschrift von Daniel Kallendorf\\
\small Danke an Sandra Kühne für ihre Mitschriften}
\date{Version vom \today}

\begin{document}
	
	\maketitle
	\tableofcontents
	
	\section{Wiederholung}
	
	\begin{satz}
		Seien $\mathfrak a\subset A$, dann
		\begin{enumerate}
			\item $\mathfrak a$ ist Primideal $\Leftrightarrow$ $A/\mathfrak p$ ist Integritätsbereich (nullteilerfrei)
			\item $\mathfrak a$ ist maximales Ideal $\Leftrightarrow$ $A/\mathfrak a$ ist ein Körper.
		\end{enumerate}
	\end{satz}
	\begin{proof}
		\begin{enumerate}
			\item \begin{description}
				\item[$\Rightarrow$] Sei $a+\mathfrak a\in A/p$ ein Nullteiler, dann existiert $x\in A\setminus p$, sodass
				\[(a+\mathfrak a)(x+\mathfrak a)=ax+\mathfrak a=p\]
				Also ist $ax\in \mathfrak a$ und da $\mathfrak a$ Primideal folgt $a\in\mathfrak a$.
				\item[$\Leftarrow$] Sei $A/\mathfrak a$ Integritätsbereich und sei $ab\in\mathfrak a$, dann ist
				\[(a+\mathfrak a)(b+\mathfrak a)=ab+\mathfrak a=\mathfrak a\]
				Da $A/\mathfrak a$ Integritätsbereich ist gilt $a+\mathfrak a=\mathfrak a$ oder $b+\mathfrak a=\mathfrak a$, also $a\in\mathfrak a$ oder $b\in\mathfrak a$.
			\end{description}
			\item \begin{description}
				\item[$\Rightarrow$] Sei $I/\mathfrak a$ ein Ideal in $A/\mathfrak a$.\\
				Hierbei ist $I$ eine Ideal in $A$ welches $\mathfrak a$ enthält, also $\mathfrak a\subseteq I\subseteq A$.\\
				Da $\mathfrak a$ maximal ist, muss $\mathfrak a=I$ oder $\mathfrak a=A$. Also ist $A/\mathfrak a$ ein Körper.
				\item[$\Leftarrow$] Sei $I$ ein Ideal in $A$ mit $\mathfrak a\subseteq I\subseteq A$.\\
				Dann ist $I/\mathfrak a$ eine Ideal in $A/\mathfrak a$, d.h.
				\[I/\mathfrak a=\mathfrak a/\mathfrak a\quad\text{oder}\quad I/\mathfrak a=A/\mathfrak a\]
				
				Damit folgt $I=\mathfrak a$ oder $I=\mathfrak A$.
			\end{description}
		\end{enumerate}
	\end{proof}
	\begin{bem*}
		Insbesondere ist jedes maximale ideal prim.
	\end{bem*}

	\begin{definition}
		Sei $A\neq\emptyset$. Eine \textbf{Relation} auf $A$ ist eine Teilmenge $R\subset A\times A$.\\
		$R$ heißt \textbf{partielle Ordnung} wenn
		\begin{enumerate}
			\item $\forall a\in A$ gilt $(a,a)\in R$ (Reflexivität)
			\item $\forall a,b,c\in A$ gilt $(a,b)\in R$ und $(b,c)\in R$, so gilt auch $(a,c\in R)$ (Transitivität)
			\item $\forall a,b\in A$ mit $(a,b\in R)$ und $(b,a)\in\R$, dann gilt $a=b$. (Antisymmetrie)
		\end{enumerate}
	
	Ist $R$ eine partielle Ordnungn auf $A$ so schrieben wir für $(a,b)\in R$ auch $a\leq b$.\\
	Zwei Elemente $a,b\in A$ heißen \textbf{vergleichbar}, wenn $a\leq b$ oder $b\leq a$ ist.\\
	Eine Teilmenge $B\subset A$ heißt \textbf{Kette}, wenn für alle $a,b\in B$ gilt, dass $a\leq b$ oder $b\leq a$.
	\end{definition}

	\begin{lem}
		Sei $A\neq \emptyset$ partielle geordnet. Hat jede Kette $B\neq \emptyset$ in $A$ eine obere Schranke in $A$, d.h. es gibt ein $a\in A$,sodass $b\leq a$ für alle $b\in B$., so besitzt $A$ ein maximales Element.
	\end{lem}

	\begin{theorem}
		Sei $A\neq 0$ ein Ring, dann besitzt $A$ ein maximales Ideal.
	\end{theorem}
	\begin{proof}
		Sei $\Sigma=\{I\subset A\mid \text{$I$ ist Ideal}\}$. Dann ist $O\in\Sigma$ und $\Sigma$ ist partielle geordnet durch die mengentheoretische Inklusion.\\
		Sei $(C_i)_{i\in I}$ eine Kette in $\Sigma$. Dann ist
		\[C=\bigcup_{i\in I}C_i\]
		ein Ideal in $A$. Aus $I\notin C_i$ für alle $i\in I$ folgt, dass $I\notin C$,d.h. $C\in\Sigma$. Somit hat $\Sigma$ ein maximales Element.
	\end{proof}

	\begin{kor}
		Sei $A$ ein Ring und $I\subsetneq A$ ein Ideal, dann ist $I$ in einem maximalen Ideal enthalten.
	\end{kor}
	\begin{kor}
		Sei $A$ ein Ring und $a\in A\setminus A^*$. Dann ist $a$ in einem maximalen Ideal enthalten.\\
	\end{kor}
	\begin{proof}
		Betrachte $(a)=Aa\neq A$.
	\end{proof}

	\subsection{Lokale Ringe}
	\begin{definition}
		Ein Ring $A$ mit nur eine maximalen Ideal $\mathfrak m$ heißt \textbf{lokaler Ring} und $A/\mathfrak m$ heißt \textbf{Restklassenkörper} von $A$.
	\end{definition}
	
	\begin{satz}
		Sei $A$ ein Ring und $\mathfrak m\neq A$ eine Ideal in $A$.\\
		Ist jedes $x\in A\setminus \mathfrak+ m$ eine Einheit, si ist $A$ ein lokaler Ring mit maximalen Ideal $\mathfrak m$.
	\end{satz}
	\begin{proof}
		Für jedes Ideal $I\subsetneq A$ gilt $I\cap A^*=\emptyset$, enthält also keine Einheiten und ist somit in $\mathfrak m$ enthalten. Somit ist $\mathfrak m$ das einzige maximale Ideal.
	\end{proof}

	\begin{satz}
		Sei $A$ ein Ring und $\mathfrak m\subset A$ eine maximales Ideal, sodass jedes Element $m$ eine Einheit in $A$ ist. Dann ist $A$ ein lokaler Ring.
	\end{satz}

	\begin{exmlist}
		\begin{exm}
			Jedes Ideal in $\Z$ ist der Form $(m)=\Z m$ mit $m\in\Z_{\geq 0}$.\\
			Es gilt, dass $(m)$ genau dann Primideal ist, wenn $m=0$ oder $m$ Primzahl.\\
			Ist $\mathfrak p$ Primzahl, so ist $(p)$ maximal.
			\item Sei $K$ ein Körper und $A=K[X_1,...,X_n]$. Dann ist der Kern des Homomorphismus $\phi:A\to K,f\mapsto f(0)$ ein maximales Ideal in $A$.
		\end{exm}
	\end{exmlist}

	\subsection{Radikale}
	\begin{satz}
		Sei $A$ eine Ring und $N=\{a\in A\mid \text{$a$ ist nilpotent}\}$. Dann ist $N$ ein Ideal in $A$ und $A/N$ enthält keine nilpotenten Elemente $\neq 0$.
	\end{satz}
	\begin{proof}
		\begin{itemize}
			\item Zz: $N$ ist eine additive Untergruppe von $A$\\
			Seien $x,y\in N$ mit $x^n=y^m=0$. Dann ist
			\[(x+y)^{n+m}=\sum_{k=0}^{n+m}\binom{n+m}{k}x^ky^{n+m-k}=0\]
			denn kann nicht sowohl $k<n$, als auch $n+m-k<m$ sein.
			\item Z.z. $AN\subset N$.\\
			Sei $x\in N$ mit $x^n=0$ und $a\in A$.
			Dann ist $(ax)^n=a^nx^n=0$, also $ax\in N$.\\
			Also ist $N$ Ideal in $A$.\\
			Sei nun $a+N\in A/N$ nilpotent. Dann ist $(a+N)^n=0$ für ein $n>0$. Also ist $a^n+N=0$, also $a^n\in N$.\\
			Dann ist $(a^n)^m=0$ udn somit $a^{nm}=0$, also nilpotent. Es folgt, dass $a\in N$.
		\end{itemize}
	\end{proof}

	\begin{definition}
		Das Ideal $N=\{a\in A\mid\text{$a$ ist Nilpotent}\}$ heißt das \textbf{Nilikal} von $A$.
	\end{definition}

	\begin{definition}
		Sei $A$ ein Ring dann nennt man $J=\{x\in A\mid \forall y\in A:\text{$1-xy$ ist Einheit}\}$ das \textbf{Jacobsonradikal}.
	\end{definition}

	\begin{satz}
		Sei $A$ eine Ring, dann ist 
		\begin{enumerate}
			\item das Nilradikal von $A$ der Schnitt aller Primideal von $A$.
			\item das Jacobsonradikal von $A$ der Schnitt aller Maximalen Ideale von $A$.
		\end{enumerate}
	\end{satz}

	\begin{definition}
		Sei $A$ ein Ring und $\mathfrak a\subset A$ ein Ideal in $A$. Dann wird
		\[r(a):=\{x\in A\mid \text{$x^n\in \mathfrak a$ für ein $n>0$}\}\]
		als \textbf{Radikal} von $\mathfrak a$ bezeichnet. (auch $\mathrm{Rad}(\mathfrak a),\sqrt{\mathfrak a}$)
	\end{definition}
	\begin{proof}
		Sei $\pi:A\to A/\mathfrak a$ die Kanonische Projektion. Dann ist $r(a)=\pi^{-1}\left(N_{A/\mathfrak a}\right)$.\\
		Also ist $r(a)$ ein Ideal.
	\end{proof}
	
	\begin{satz}
		Sei $\mathfrak a,\mathfrak b$ ein Ideal, dann gilt
		\begin{enumerate}
			\item $\mathfrak a\subseteq r(\mathfrak a)$
			\item $r\big(r(\mathfrak a)\big)=r(\mathfrak a)$
			\item $r(\mathfrak a\mathfrak a)=r(\mathfrak a\cap \mathfrak b)=r(\mathfrak a)\cap r(\mathfrak b)$
			\item $r(\mathfrak a)=A\Leftrightarrow\mathfrak a=A$.
			\item $r(\mathfrak a+\mathfrak b)=r\big(r(\mathfrak a)+r(\mathfrak b)\big)$.
		\end{enumerate}
	\end{satz}

	%VL 23.10.2017
	\subsubsection{Operationen auf Radikalen}
	%TODO Zerlegen in Definition und Satz
	\begin{definition}
		Sein $A$ ein Ring.
		\begin{enumerate}
			\item Seien $\mathfrak a,\mathfrak b\subset A$ Ideale in $A$.\\
			Dann ist
			\[a+b=:\{x+y\mid x\in\mathfrak{a},y\in\mathfrak{b}\}\]
			ein Ideal in $A$.
			\item Analog: Sei $(\mathfrak{a}_i)_{i\in I}$ eine Familie von Idealen in $A$, für eine Indexmenge $I$. Dann ist
			\[\sum_{i\in I}\mathfrak{a}_i=:\left\{\sum_{i\in I}x_i\mid \text{$x_i\in \mathfrak{a}_i$ und fast alle $x_i=0$}\right\}\]
			ein Ideal in $A$.
			\item Sei $(\mathfrak{a}_i)_{i\in I}$ eine Familie von Idealen in $A$, für eine Indexmenge $I$. Dann ist der Schnitt
			\[\bigcap_{i\in I}\mathfrak a_i\]
			ein Ideal in $A$.
			\item Seien $\mathfrak a,\mathfrak b\subset A$ Ideal in $A$. Dann ist
			\[\mathfrak a\mathfrak b=\left\{\sum_{i=1}^{n}a_ib_i\mid a_i\in\mathfrak a,b_i\in\mathfrak b,n\in\N\right\}\] 
			ein Ideal in $A$.
		\end{enumerate}
	\end{definition}

	\begin{satz}
		Die Operationen Summe, Durchschnitt und Produkt auf Idealen sind kommutativ und Assoziativ und es gilt das Distributivgesetz.
	\end{satz}

	\begin{definition}
		Sei $A$ ein Ring. Zwei Ideale $\mathfrak a,\mathfrak b\subseteq A$ heißen \textbf{teilerfremd}, wenn $\mathfrak a+\mathfrak b=A=(1)$.
	\end{definition}

	\begin{satz}
		Sei $A$ ein Ring, $\mathfrak a,\mathfrak b\subset A$ Ideale in $A$. Dann sind äquivalent:
		\begin{enumerate}
			\item $\mathfrak a,\mathfrak b$ sind Teilerfremd
			\item Es gibt ein $x\in\mathfrak a,y\in\mathfrak b$, sodass $x+y=1$.
		\end{enumerate}
	\end{satz}
	\begin{proof}
		\begin{description}
			\item[2)$\Rightarrow$1)] Sei $z\in A$ und $x\in\mathfrak a,y\in\mathfrak b$, mit $x+y=1$.\\
			Dann ist $z=zx+zy$, wobei $zx\in\mathfrak a,zy\in\mathfrak b$, also $z\in\mathfrak a+\mathfrak b$.
			\item[1)$\Rightarrow$2)] 
		\end{description}
	\end{proof}
	
	\begin{satz}
		Sei $A$ ein Ring und seinen $\mathfrak a_1,...,\mathfrak a_n$ paarweise teilerfremde Ideal in $A$. Dann gilt
		\begin{enumerate}
			\item Jedes $\mathfrak a_i$ ist teilerfremd zu $\prod_{\substack{j=1\\j\neq i}}^{n}\mathfrak a_j$.
			\item Es gilt
			\[\prod_{i=1}^{n}\mathfrak a_i=\bigcap_{i=1}^n \mathfrak a_i\]
		\end{enumerate}
	\end{satz}
	\begin{proof}
		\begin{enumerate}
			\item Sei $i$ fest. Es gibt Elemente $x_j\in \mathfrak a_i, y_j\in\mathfrak a_j$ mit $1=x_j+y_j$ für $i\neq j$. Dann ist
			\[
			1=\prod_{\substack{j=1\\j\neq i}}(x_j+y_j)
			=\underbrace{x}_{\mathclap{\in\mathfrak a_i}}+\underbrace{\prod_{\substack{j=1\\j\neq i}}}_{\in \prod_{\substack{j=1\\j\neq i}}\mathfrak a_j}
			\in\mathfrak a_i+\prod_{\substack{j=1\\j\neq i}}\mathfrak a_j
			\]
			\item Durch Induktion über $n$.
			\begin{description}
				\item[$n=2$] Sei $z\in\mathfrak a\cap\mathfrak b$. Schreieb $1=x+y$ mit $x\in\mathfrak a,y\in\mathfrak b$. Dann ist $z=zx+zy\in\mathfrak a\mathfrak b$.
				\item[$n>2$] Sei 
				\[\mathfrak b=\prod_{i=1}^{n-1}a_i\]
				Wir nehmen an es gelte
				\[\prod_{i=1}^{n-1}a_i=\prod_{i=1}^{n-1}\mathfrak a_i\]
				Dann ist aber
				\[\prod_{i=1}^n \mathfrak a_i=\mathfrak a_i\mathfrak b_i=\mathfrak a_i\cap\mathfrak b=\bigcap_{i=1}^n a_i\]
			\end{description}
		\end{enumerate}
	\end{proof}

	\begin{definition}
		Sei $A$ ein Ring und seinen $\mathfrak a_i,....,\mathfrak a_n$ Ideale in $A$.\\
		Wir definieren die Abbildung 
		\begin{align*}
		\phi:A&\to\prod_{i=1}^n(A/\mathfrak a_i)\\
		a&\mapsto(a+\mathfrak a_1,...,a+\mathfrak a_n)
		\end{align*}
	\end{definition}

	\begin{prop}
		\begin{enumerate}
			\item $\phi$ ist ein Ringhomomorphismus und
			\[\Kern(\phi)=\bigcap_{i=1}^n\mathfrak a_i\]
			\item $\phi$ ist genau dann surjektiv, wenn die $\mathfrak a_i$ paarweise disjunkt sind.
		\end{enumerate}
		Insbesondere ist
		\[A/\prod_{i=1}^n\mathfrak a_i\simeq \prod_{i=1}^nA/\mathfrak a_i\]
	\end{prop}
	\begin{proof}
		\begin{enumerate}
			\stepcounter{enumi}
			\item \begin{description}
				\item[$\Rightarrow$] Sei $\phi$ surjektiv. Wir zeigen, dass $\mathfrak a_1$ und $\mathfrak a_2$ teilerfremd sind.\\
				Es gibt ein $x\in A$ mit $\phi(x)=(1_{A/\mathfrak a_1},0,...,0)$.\\
				Also ist $x=1\mod \mathfrak a_i$ und $x=x\mod \mathfrak a_2$.\\
				Dann ist
				\[1=\underbrace{(1-x)}_{\in\mathfrak a_i}+\underbrace{x}_{\mathclap{\in\mathfrak a_2}}\in \mathfrak a_1+\mathfrak a_2\]
				\item[$\Leftarrow$] Seien un die $\mathfrak a_i$ paarweise teilerfremd.\\
				Es reicht zu zeigen,dass es Elemente $x_i\in A$ mit
				\[\phi(x_i)=(0,...,0,1,0,...,0)\]
				($1$ an der $i$-ten Position) gibt.\\
				Wir zeigen für $i=1$:\\
				Da $\mathfrak a_1+\mathfrak a_j=A$ für alle $j>1$, gibt es $x_j\in \mathfrak a_1, y_j\in\mathfrak a_j$ mit $x_j+y_j=1$\\
				Sei nun
				\[x:=\prod_{i=2}^ny_j=\prod_{i=2}^n1-x_j=1\mod\mathfrak a_1\]
				und $x=0\mod\mathfrak a_j$ für $j>1$.
			\end{description}
		\end{enumerate}
	\end{proof}
	
	\subsection{Ringe von Brüchen}
	\begin{definition}
		Sei $A$ ein Ring. Eine Teilmenge $S\subset A$ heißt \textbf{multiplikativ abgeschlossen}, wenn
		\begin{enumerate}
			\item Für alle $s,t\in S$ gilt, dass $st\in S$
			\item $1\in S$.
		\end{enumerate}
	\end{definition}

	\begin{bem}
		Auf $A\times S$ wird durch 
		\[(a,s)\sim (b,t)\Leftrightarrow (at-bs)u=0\text{ für ein $u\in S$}\]
		eine Äquivalenzklasse definiert.\\
		Für die Transitivität wird die multiplikative Abgeschlossenheit von $S$ benötigt.\\
		Die Äquivalenzklassen von $(a,s)$ wird mit $a/s$ bezeichnet.\\
		Die Menge der Äquivalenzklasssen wir als $S^{-1}A$ geschrieben.
	\end{bem}

	\begin{definition}		
		Seien $a/s,b/t\in S^{-1}A$. Man definiert
		\begin{itemize}
			\item $a/s+b/t:=(at+bs)/st$
			\item $a/s\cdot b/t:=ab/st$
		\end{itemize}
	\end{definition}

	\begin{definition}
		Diese Verknüpfungen sind wohldefiniert und versehen $S^{-1}A$ mit einer Ringstruktur.\\
		$S^{-1}A$ wird als der \textbf{Ring der Brüche} von $A$ bezüglich $S$ bezeichnet.
	\end{definition}

	\begin{exm}
		Sei $A=\Z$ und $S=\Z\setminus\{0\}$. Dann ist $S^{-1}A$ isomorph zu $\Q$.
	\end{exm}


	\begin{kor}
		Die Abbildung
		\begin{align*}
			\varphi_S:A&\to S^{-1}A\\
			a\mapsto a/1
		\end{align*}
		hat folgende Eigenschaften:
		\begin{enumerate}
			\item $\varphi_S$ ist ein Ringhomomorphismus. (i.A. nicht injektiv)
			\item Sei $s\in S$, dann ist $\varphi_S(s)$ eine Einheit in $S^{-1}A$.
			\item $\Kern(\varphi_S)=\{a\in A\mid\text{$as=0$ für ein $s\in S$}\}$.
			\item Jedes Element in $S^{-1}A$ ist der Form $\varphi_S(a)\varphi_S(s)^{-1}$ für ein $a\in A$, $s\in S$.
		\end{enumerate}
	\end{kor}
	\begin{proof}
		\begin{enumerate}
			\stepcounter{enumi}
			\item Sei $s\in S$, dann ist $s/1\cdot 1/s=s/s=1/1=1_{S^{-1}A}$
			\item Sei $a\in\Kern(\varphi_S)$, dann ist $a/1=0/1$, also $(a1-01)s=0$ für ein $s\in S$. Also ist $as=0$ für ein $s\in S$.
			\item  Sei $a/s\in S^{-1}A$. Dann ist
			\begin{align*}
			\varphi_S(a)&=a/1& \varphi_S(s)&=s/1 &\varphi_S(s)^{-1}&=1/s
			\end{align*}
			Es folgt
			\[\varphi_S(a)\varphi(s)^{-1}=a/1\cdot 1/s=a/s\]
		\end{enumerate}
	\end{proof}

	\begin{satz}
		Seien $A,B$ Ringe und $S\subset A$ multiplikativ abgeschlossen. Sei $g:A\to B$ ein Ringhomomorphismus, der 1)-3) aus %TODO ref Kor
		erfüllt, dann gibt es einen eindeutigen Isomorphismus $h:S^{-1}A\to B$ mit $h\circ \varphi_S=g$.
		%TODO Komm Diag
			\[
			\begin{tikzcd}
				A \ar[r,"g"] \arrow[d,"\varphi_S"] & B\\
				S^{-1}A \arrow[ur,"h"']&
			\end{tikzcd}
			\]
	\end{satz}

%VL 25.10.2017
	\begin{definition}
		Sei $A$ ein Integritätsbereich und $S=A\setminus \{0\}$. Dann nennt man $S^{-1}A$  den \textbf{Quotientenkörper}
	\end{definition}

	\begin{lem}
		Der Quotientenkörper ist ein Körper, $\varphi_S$ ist injektiv und wir können $A$ mit seinem Bild in $S^{-1}A$ identifizieren.
	\end{lem}

	\begin{definition}
		Sei $A$ ein Ring. Sei $\mathfrak p$ ein Primideal in $A$. Man schreibt $A_{\mathfrak p}$ für $S^{-1}A$ und nennt $A_{\mathfrak p}$ die \textbf{Lokalisierung} von $A$ bezüglich $\mathfrak p$.
	\end{definition}

	\begin{lem}
		Sei $A$ ein Ring. Sei $\mathfrak p$ ein Primideal in $A$.\\
		Dann ist $S=A\setminus \mathfrak p$ multiplikativ Abgeschlossen.
	\end{lem}

	\begin{lem}
		Sei $A=\Z$ und $p\in\Z$ eine Primzahl. Dann ist $\Z_{(p)}=\{m/n\mid m/n\in\Q,p\not|n\}$.
	\end{lem}

	\begin{satz}
		Sei $A$ ein Ring und $S\subset A$ multiplikativ abgeschlossen. Dann ist
		\begin{enumerate}
			\item Ist $I$ ein Ideal in $A$ so ist auch $S^{-1}I=\{a/s\mid a\in I\}$ ein Ideal in $S^{-1}A$\\
			\item Die Ideale in $S^{-1}A$ sind der Form $S^{-1}I$, wobei $I$ ein Ideal in $A$ ist.
			\item Sind $I,J$ Ideal in $A$, dann gilt
			\begin{align*}
				S^{-1}(I+J)&=S^{-1}I+S^{-1}J\\
				S^{-1}(I\cap J)&=S^{-1}I\cap S^{-1}J\\
				S^{-1}(IJ)&=(S^{-1}I)(S^{-1}J)
			\end{align*}
		\end{enumerate}
	\end{satz}
	\begin{proof}
		Wir beweisen nur 2).\\
		Sei $J$ ein Ideal in $S^{-1}A$. Dann ist $I=\varphi_S^{-1}(J)$ ein Ideal in $A$ und $J=S^{-1}I$:\\
		Sei $a/s\in S^{-1}I$. Aus $I=\varphi^{-1}_S(J)$ folgt, dass $\varphi_S(a)\in J$. Also ist
		\[a/s=\underbrace{a/1}_{\varphi_S(a)}\cdot\underbrace{1/s}_{\in S^{-1}A}\in J\]
		d.h. $s\in\varphi^{-1}_S(J)=I$ und $a/s\in S^{-1}I$.
	\end{proof}

	\subsection{Integritätsbereiche und Hauptidealringe}
	\begin{definition}
		Sei $A$ ein Ring. Ein Ideal der Form $(a)=Aa$ heißt \textbf{Hauptideal}.
	\end{definition}

	\begin{definition}
		Ein Ring $A$ heißt \textbf{Hauptidealring}, wenn jede Ideal in $A$ Hauptideal ist.
	\end{definition}

	\begin{definition}
		Ein Ring $A$ heißt \textbf{euklidisch}, wenn es eine Abbildung 
		\[\la:A\setminus \{0\}\to \N_0\]
		gibt, sodass zu je zwei Elementen $a,b\in A$ mit $b\neq 0$ Elemente $q,r\in A$ existieren mit $a=qb+r$ wobei $\la(r)<\la(b)$ oder $r=0$ .
	\end{definition}

	\begin{exm}
		\begin{enumerate}
			\item $\Z$ ist euklidisch unter $\la(x)=|x|$.
			\item Sei $K$ ein Körper. Dann ist $K[X]$ euklidisch mit $\la(f)=\deg(f)$.
		\end{enumerate}
	\end{exm}

	\begin{satz}
		Sei $A$ ein euklidischer Ring. Dann ist $A$ ein Hauptidealring.
	\end{satz}
	\begin{proof}
		Sei $\mathfrak a\neq 0$ in Ideal in $A$. Dann hat
		\[\la(x)\mid x\in a,x\neq 0\] ein kleinstes Element, d.h. es gibt ein $x\in \mathfrak a\setminus\{0\}$ mit $\la(x)\leq\la(y)$ für alle $y\in \mathfrak a\setminus \{0\}$.\\
		Es gilt $\mathfrak a=(x)$.\\
		Sei $y\in a\setminus\{0\}$. Schreibe $y=qx+r$ mit $r=0$ oder $\la(r)<\la(x)$.\\
		Dann ist $r\in\mathfrak a$ und aus der Minimalität von $\la(x)$ folgt $r=0$ und damit $\mathfrak a\subset (x)$.
	\end{proof}

	\begin{definition}
		Sei $A$ ein Ring und seinen $a,b\in A$.\\
		$d\in A$ heißt \textbf{Größter gemeinsamer Teiler} von $a$ und $b$, wenn gilt
		\begin{enumerate}
			\item $d|a$ und $d|b$.
			\item Wenn es $g\in A$ gibt mit $g|a$ und $g|b$, dann muss $g|d$.
		\end{enumerate}
		Wir schreiben $d=\gcd(a,b)=(a,b)$
	\end{definition}

	\begin{definition}
		Sei $A$ ein Ring und seinen $a,b\in A$.\\
		$d\in A$ heißt \textbf{kleinstes gemeinsames Vielfaches} von $a$ und $b$, wenn gilt
		\begin{enumerate}
			\item $a|v$ und $b|v$.
			\item Wenn es $g\in A$ gibt mit $a|g$ und $b|g$, dann muss $v|v$.
		\end{enumerate}
		Wir schreiben $v=\lcm(a,b)=(a,b)$
	\end{definition}

	\begin{satz}
		Sei $A$ ein Hauptidealring und seien $a,b\in A$.\\
		Dann existiert ein $d=\gcd(a,b)$ und $v=\lcm(a,b)$ von $a,b$ und es gilt
		\begin{enumerate}
			\item $(a)+(b)=(d)$
			\item $(a)\cap (b)=(v)$
		\end{enumerate}
	\end{satz}
	\begin{proof}
		\begin{itemize}
			\item Da $A$ ein Hauptidealring ist, gilt $(a)+(b)=(d)$ für ein $d\in A$.\\
			Es gilt $a,b\in(d)$, also $d|a$ und $d|b$.\\
			Sei $g\in A$ mit $g|a$ und $g|b$. Dann ist $(a)\subset (g)$ und $(b)\subset(g)$.\\
			Daraus folgt, dass $(a)+(b)\subseteq(g)$, also $(d)\subset (g)$. Damit folgt $g|d$.
			\item Analog für $\lcm$.
		\end{itemize}
	\end{proof}

	\begin{definition}
		Sei $A$ in Integritätsbereich. Zwei Elemente $a,b\in A$ heißen \textbf{assoziiert}, wenn
		 \begin{itemize}
		 	\item $a|b$ und $b|a$.
			\item (äquivalent) $a=bu$ für ein $u\in A^*$.
			\item (äquivalent) $(a)=(b)$.
		 \end{itemize}
	 Man schreibt dann $a\sim b$.
	\end{definition}

	\begin{definition}
		Sei $A$ in Integritätsbereich. Ein Element $p\in A$ heißt \textbf{prim}, \textbf{Primelement}, wenn
		\begin{itemize}
			\item $p\notin A^*$, $p\neq0$ und aus $p|ab$ folgt $p|a$ oder $p|b$.
			\item (äquivalent) $p\neq 0$ und $(p)$ ist Primideal.
		\end{itemize}
	\end{definition}

	\begin{definition}
		Sei $A$ in Integritätsbereich. $c\in A$ heißt \textbf{irreduzibel} oder \textbf{unzerlegbar}, wenn
		\begin{enumerate}
			\item für $c\notin A^*$ und $c\neq 0$ aus $c=ab$ folgt, dass $a\in A^*$ oder $b\in A^*$.
			\item (äquivalent) für $c\neq 0$ für alle $a\in A$ gilt, dass aus $(c)\subset(a)$ folgt, dass $(a)=A$ oder $(a)=(c)$.
		\end{enumerate}
	\end{definition}

	\begin{satz}
		Sei $A$ ein Integritätsbereich und $p\in A$ prim. Dann ist $p$ irreduzibel.
	\end{satz}
	\begin{proof}
		Sei $p=ab$, dann gilt $p|ab$. Es folgt $p|a$ oder $p|b$.\\
		Angenommen $p|a$, dann ist $a=px$ für ein $x\in A$ und $p=pxb$. Es folgt, dass $p(1-bx)=0$ und da $A$ Integritätsbereich ist $1-bx=0$.\\
		Also muss $bx=1$ also ist $b\in A^*$.
	\end{proof}

	\begin{satz}
		Sei $A$ ein Hauptidealring und Integritätsbereich. Dann gilt für $c\in A$
		\[\text{$c$ prim}\Leftrightarrow\text{$c$ irreduzibel}\]
	\end{satz}
	\begin{proof}
		Sei $c$ irreduzibel, also ist $(c)$ maximal. Daraus folgt, dass $(c)$ Primideal ist und somit $c$ prim.
	\end{proof}

	\begin{definition}
		Ein Integritätsbereich heißt \textbf{faktoriell}, wenn
		\begin{enumerate}
			\item Jedes $a\in A\setminus A^*$, $a\neq 0$ zerfällt in ein Produkt von irreduziblen Elementen.
			\item Die Zerlegung ist bis auf Reihenfolge und Einheiten eindeutig. D.h.
		\end{enumerate}
		D.h. wenn $a=c_1\cdot ...\cdot c_m=d_1\cdot...\cdot d_n$ mit $c_1,d_1$ irreduzibel, so folgt $m=n$ und es gibt $\pi\in S_n$ mit $c_1\sim d_{\pi(i)}$ für alle $i=1,...,n$.
	\end{definition}

	\begin{bem}
		Die Eindeutigkeit der Faktorisierung impliziert, dass es irreduzibles Element in einem faktoriellen Integritätsbereich prim ist.
	\end{bem}

%VL 30.10.2017

	\begin{lem}
		Sei $A$ ein Hauptidealring und $S$ eine nichtleere Menge von Idealen in $A$. Dann hat $S$ ein maximales Element (bezüglich $\subset$)
	\end{lem}
	\begin{proof}
		Angenommen $S$ hat kein maximales Element. Dann gibt es zu jedem $\mathfrak a_1\in S$ ein $\mathfrak a_2\in S$ mit $\mathfrak a_1\subsetneq \mathfrak a_2$. Es gibt also eine unendliche Kette
		\[\mathfrak a_1\subsetneq \mathfrak a_2\subsetneq...\]
		von Idealen in $S$. Sei nun $\mathfrak a:=\bigcup_{j=1}^\infty \mathfrak a_i$.\\
		Dann ist $a$ ein Ideal in $A$, also ist $\mathfrak a$ ein Hauptideal und $\mathfrak a=(x)$ für ein $x\in A$.\\
		Dann folgt insbesondere, dass $x\in\mathfrak a$. Damit folgt, dass es $j_0\in\N$ gibt, mit $x\in \mathfrak a_{j_0}$.\\
		Somit ist $(x)\subset\mathfrak a_{j_0}$ und somit $\mathfrak a=\mathfrak a_{j_0}$.\\
		Dies bedeutet aber, dass die Kette stationär wird, was ein Widerspruch zur Annahme ist.
	\end{proof}
	
	\begin{theorem}
		Sei $A$ ein Integritätsbereich. Ist $A$ ein Hauptidealring, so ist $A$ faktoriell.
	\end{theorem}
	\begin{proof}
		\begin{description}
			\item[Zerlegbarkeit der Elemente] Sei $S=\{(a)\mid a\in A,a\notin A^*,a\neq 0\text{a zerfällt nicht in irreduzible Faktoren}\}$.\\
			Angenommen $S\neq\emptyset$. Dann hat $S$ eine maximales Element $(a)$ und $a$ ist nicht irreduzibel.\\
			Dann gibt es $b,c\in A\setminus A^*$, mit $a=bc$.\\
			Also ist $(a)\subsetneqq (b)$ und $(a)\subsetneqq (c)$. Da $(a)$ maximal in $S$ ist folgt daraus, dass $(b),(c)\notin S$.\\
			Somit zerfallen $b,c$ in irreduzible Faktoren und damit gilt $a\in S$. Widerspruch!.
			\item[Eindeutigkeit der Zerlegung] Sei $a\in A$. Angenommen es gäbe zwei irreduzible Zerlegungen $a=c_1...c_m=d_1...d_n$ mit $m\leq n$.\\
			Dann ist $c_1$ irreduzibel und somit prim. Also muss $c_1|d_i$ für ein $i$ gelte.\\
			Nach Umnummerierung gilt $c_1|d_1$, also $d_1=u_1c_1$ für $u_1\in A^*$.\\
			Also ist
			\begin{align*}
			c_1...c_m&=u_1c_1d_2...d_n\\
			\Rightarrow\quad c_2...c_m&=d_2...d_n\\
			\end{align*}
			Fortsetzen des Argumentes liefert
			\[1=u_1...u_md_{m+1}...d_n\]
			für geeignete $u_i\in A^*$.\\
			Dann sind aber $d_{m+1},...,d_n$ Einheiten und damit Eindeutig bis auf Einheiten und Reihenfolge.
		\end{description}
	\end{proof}

\subsection{Inverse und direkte Limiten}

	\begin{definition}
		Man nennt $I$ eine unter $\leq$ \textbf{partiell geordnete Menge}, wenn für alle $x,y,z\in I$ gilt
		\begin{enumerate}
			\item $x\leq x$.
			\item Aus $x\leq y$ und $y\leq z$ folgt $x\leq z$.
			\item Aus $x\leq y$ und $y\leq x$ folgt $x=y$.
		\end{enumerate}
	\end{definition}

	\begin{definition}
		Für jedes $i\in I $sei $A_i$ ein Ring und sei für jedes Paar $i,j\in I$ mit $i\leq j$ die Abbildung $f_{ij}:A_j\to A_i$ ein Ringhomomorphismus, sodass
		\begin{enumerate}
			\item $f_{ii}=\id_{A_i}$ für alle $i\in I$
			\item $f_{ik}=f_{ij}\circ f_{jk}$ falls $i\leq j\leq k$.
		\end{enumerate}
		Dann nennt man das System $(A_i,f_{ij})_{i,j\in I}$ \textbf{projektives System} von Ringen.
	\end{definition}

	\begin{definition}
		Ein Ring $A$ zusammen mit dem Homomorphismus $f_i:A\to A_i$, sodass $f_i=f_{ij}\circ f_{j}$ für $i\leq j$ heißt \textbf{projektiver Limes} oder \textbf{inverser Limes} des Systems $(A_i,f_{ij})$, wenn folgende universelle Eingenschaft erfüllt ist:\\
		Sind $h_u:B\to A_i$ für alle $i\in I$ Ringhomomorphismen mit $h_i=f_{ij}\circ h_j$ für $i\leq j$, so existiert genau ein Ringhomomorphismus $h:B\to A$ mit $h_i=f_i\circ h$ für alle $i\in I$.
		%TODO Tikzcd 1
		\[\begin{tikzcd}
		B \ar[rr,"\exists!h"] \ar[rdd,"h_i"] \ar[rd,"h_j"] && A \ar[ld,"f_j"] \ar[ldd,"f_i"]\\
		& A_j \ar[d,"f_j"]&\\
		& A_i &
		\end{tikzcd}\]
	\end{definition}

	\begin{bem}
		Falls ein projektiver Limes existiert, so ist er bis auf kanonische Isomorphie eindeutig:\\
		Sind $(A,f_i)$ und $(B,h_i)$ projektive Limiten von $(A_i,f_{ij})$, so gibt es Homomorphismen $h:B\to A$ und $g:A\to B$, die die oben beschrieben Verträglichkeitsbedingungen erfüllen.\\
		Durch Zusammensetzen dieser Homomorphismen erhalten wir Abbildungen
		%TODO tikzcd 2,3,4,5
		Die Eindeutigkeitsbedingung Impliziert nun, dass $g\circ h=\id_B$ und $h\circ g=\id_A$.\\
		Man schreibt auch $A=\projlim{i\in I}A_i$ für den projektiven Limes des Systems $(A_i,f_{ij})$.
	\end{bem}

	\begin{proof}[Existenz des Projektiven Limes]
		Sei $(A_i,f_{ij})_{i,j\in I}$ ein projektives System von Ringen.\\
		Setze 
		\[A=\{(x_i)_{i\in I}\mid \text{$f_{ij}(x_j)=x_i$ für $i\leq j$}\}\subset\prod_{i\in I}A_i\]
		und $h_j:A\to A_j,(x_i)_{i\in I}\mapsto x_j$.\\
		Dann ist $(A,h_i)_{i\in I}$ ein projektiver Limes von $(A_i,f_{ij})$.\\
		Insebsondere definiert jede Famiele $(x_i)_{i\in I}$ mit $f_{ij}(x_j)=x_i$ ein eindeutiges Element $x\in\projlim{i\in I}A_i$.
	\end{proof}

	\begin{exm}
		Ein Beispiel für einen projektiven Limes sind die $p$-adisches ganzen Zahlen.\\
		Sei $p\in\Z$ eine Primzahl, $I=\N$, mit der Ordnung $\leq$.\\
		Für $n\geq 1$ sei $A_n=\Z/p^n\Z$. Sei
		\begin{align*}
		f_{mn}:A_n=\Z/p^n\Z&\to A_m=\Z/p^m\Z\\
		x&\mapsto x\mod p^m
		\end{align*}
		Dann ist $(A_m,f_{mn})_{m,n\geq 1}$ ein projektives System. Der projektive Limes wird als Ring der $p$-adischen ganzen Zahlen 
		\[\Z_p=\projlim{n\geq 1}A_n\]
		bezeichnet. Also ist
		\begin{align*}
		\Z_p&=\{(x_n)_{n\geq 1}\mid x_n\in\Z/p^n\Z,f_{mn}(x_n)=x_n\text{ für }m\leq n\}\\
		&=\{(x_n)_{n\geq 1}\mid x_n\in\Z/p^n\Z,x_n\mod p^{n-1}=x_{n-1}\}
		\end{align*}
		Wir schreiben die Elemente aus $\Z_p$ auch als Folgen
		\[x=(x_n)_{n\geq 1}=(...,x_{n+1},x_n,....,x_1)\]
		mit $x_n\mod p^{n-1}=x_{n-1}$.\\
		Addition und Multiplikation erfolgen komponentenweise.\\
		Sie Abbildung
		\begin{align*}
		\Z&\to\Z_p\\
		m&\mapsto(...,m+p^n,...,m+p)
		\end{align*}
		ist in injektiver Ringhomomorphismus.\\
		\\
		%VL 01.11.2017
		
		Sei $x=(...,x_n,x_{n-1},...,x_1)$. Ist $x\neq 0$, so ist $x$ der Form $(...,x_{n+1},x_n,0,...,0)$ und für $j\leq n$ sind alle Einträge $x_j\neq$.\\
		Weiterhin gilt
		\[p|x\Leftrightarrow\text{$x|x_n$ für alle $n\geq 1$}\]
	\end{exm}

	\begin{satz}
		Sei $x\in\Z_p$. Dann ist
		\begin{enumerate}
			\item $x\in\Z_p^*$ $\Leftrightarrow$ $p\not| x$
			\item Ist $x\neq 0$, so lässt sich $x$ eindeutig schreiben als $x=p^nu$ mit $u\in\Z_p^*$ und $n\geq 0$.
		\end{enumerate}
	\end{satz}

	\begin{proof}
		\begin{enumerate}
			\item \begin{description}
				\item[$\Rightarrow$] Sei $x=(...,x_n,...,x_1)\in\Z_p^*$. Dann exitsiert ein $y=(...,y_n,...,y_1)\in\Z^p$ mit
				\begin{align*}
				xy&=(...,x_n,...,x_1)(...,y_n,...,y_1)\\
				&=(...,x_ny_n,...,x_1y_1)\\
				&=(...,1,...,1)=1
				\end{align*}
				d.h. jeder Eintrag von $x_j$ von $x$ ist invertierbar, d.h. $p\not| x_n$ für alle $n\geq 1$.
				\item[$\Leftarrow$] Angenommen $p\not| x$, dann muss $p\not| x_n$ für ein $n\geq 1$.\\
				Dann muss aber $p\not|x_n$ für alle $n\geq 1$.\\
				d.h. jedes $x_n$ ist invertierbar. Sei
				\[y=(...,x_n^{-1},...,x_1^{-1})\in\prod_{n\geq 1}\Z/p\Z\]
				dann erfüllt $y$ die Kompatibilitätsbedingungen, d.h. $y\in\Z_p$ und $xy=1$.
			\end{description}
			\item Ist klar.
		\end{enumerate}
	\end{proof}

	\begin{definition}
		Sei $x\in\Z_p$, $x\neq 0$. Schreibe $x=p^nu$ mit $u\in \Z_p^*$. Dann heißt\\
		\[n=\nu_p(x)\]
		die \textbf{$p$-adische Bewertung} von $x$.\\
		Man setzt $\nu_p(0)=\infty$.\\
		Man bezeichnet $|x|_p=p^{-\nu_p(x)}$ als den \textbf{$p$-adischen Betrag}.
	\end{definition}

	\begin{lem}
		Für die $p$-adische Bewertung gilt:
		\begin{enumerate}
			\item $\nu_p(xy)=\nu_p(x)+\nu_p(y)$
			\item $\nu_p(x+y)\geq\inf\big\{\nu_p(x),\nu_p(y)\big\}$
		\end{enumerate}
	\end{lem}

	\begin{satz}
		$\Z_p$ ist ein Integritätsbereich.\\
		Der Quotientenkörper $\Q_p$ von $\Z_p$ wird als Körper der $p$-adischen Zahlen bezeichnet.\\
		$\Q_p$ kann auch (analytisch) als Vervollständigung von $\Q$ bezüglich des $p$-adischen Betrage konstruiert werden.
	\end{satz}

	\begin{definition}
		Man nennt $I$ eine unter $\leq$ \textbf{gerichtete Menge}, wenn für alle $x,yz\in I$ gilt
		\begin{enumerate}
			\item $x\leq x$
			\item Aus $x\leq y$ und $y\leq z$ folgt $x\leq z$
			\item Für alle $x,y$ exitsiert ein $z\in I$mit $x\leq z$,$y\leq z$
		\end{enumerate}
	\end{definition}

	\begin{definition}
		Für jedes $i\in I$ sei ein Ring $A_i$ und für jedes Paar $i,j\in I$ mit $i\leq j$ sei ein Ringhomomorphismus $f_{ij}:A_i\to A_j$ gegeben, mit
		\begin{enumerate}
			\item $f_{ii}=\id_{A_i}$ für alle $i\in I$
			\item $f_{ik}=f_{jk}\circ f_{ij}$ für alle $i\leq j\leq k$
		\end{enumerate}
		\[\begin{tikzcd}
			A_i\ar[r,"f_{ij}"] \ar[rr,"f_{ik}"',bend right] &A_j\ar[r,"f_{jk}"]&A_k
		\end{tikzcd}\]
		Ein solches System $(A_j,f_{ij})$ heißt \textbf{induktives System} von Ringen.
	\end{definition}

	\begin{definition}
		Ein Ring $A$ zusammen mit dem einem Homomorphismus $f_i:A_i\to A$, sodass gilt $f_i=f_j\circ f_{ij}$ für $i\leq j$ heißt \textbf{induktiver Limes} oder \textbf{direkter Limes} des Systems $(A_i,f_{ij})$, wenn folgende Universelle Eigenschaft erfüllt ist:\\
		Ist $B$ ein Ring, und sind $h_i:A_i\to B$, $i\in I$ Ringhomomorphismen mit $h_i=h_j\circ f_{ij}$ für $i\leq j$, so existiert genau ein Ringhomomorphismus $h:A\to B$ mit $h_i=h\circ f_i$ für alle $i\in I$.
	\end{definition}

	\begin{lem}
		Falls ein indktiver Limes existiert, so ist er eindeutig.
	\end{lem}
	\begin{proof}
		Sei 
		\[\hat A=\dot\bigcup_{i\in I}A_i=\bigcup_{i\in I}\{(i,x)\mid x\in A_i\}\]
		Wir definieren die Äquivalenzrelation $\sim$ auf $\hat A$:\\
		Seien $x,y\in \hat A$, d.h. $x\in A_i,y\in A_j$.
		\[x\sim y\Leftrightarrow \text{ ex gibt ein $k\in I$ mit $i\leq k$ und $j\leq k$ und $f_{ik}(x)=f_{jk}(x)$}\]
	\end{proof}


%VL 06.11.2017
\section{Polynomringe}
\subsection{Polynome mit einer Variable}
Sei in diesem Abschnitt $A$ ein Ring.

	\begin{definition}
		Sei $A[X]$ die Menge der Folgen $(a_0,a_1,...,)$ mit $a_i\in A$ und $a_i=0$ für fast alle $i\in\N$.\\
		Die Elemente dieser Menge heißen \textbf{Polynome}.
	\end{definition}
	
	\begin{definition}
		$A[X]$ ist ein Ring mit 
		\begin{align*}
		(a_0,a_1,...,)+(b_0,b_1,...)&=(a_0+b_0,a_1+b_1,...)\\
		(a_0,a_1,...,)\cdot(b_0,b_1,...)&=(c_0,c_1,...)
		\end{align*}
		mit $c_n=\sum_{k=0}^n a_{n-k}b_k$.\\
		Das Nullelement ist $0=(0,0,...)$ und $1=(1,0,0,...)$ ist das Neutrale Element der Multiplikation.\\
	\end{definition}

	\begin{definition}
		$A[X]$ wird als der \textbf{Polynomring} in der \textbf{Variablen} $X$ bezeichnet.\\
	\end{definition}

	\begin{prop}
		\begin{enumerate}
			\item Die Abbildung $A\to A[X],a\mapsto(a,0,0,...)$ ist ein Injektiver Ringhomomorphismus und $A$ ist Unterring von $A[X]$.
			\item Sei $X=(0,1,0,...)$. Dann ist $X^n=(0,0,...,0,1,0,...)$ an $n$-ter Stelle und $aX^n=(0,...,0,a,0,...)$. 
			\item Polynome lassen sich schreiben als
			\[(a_0,a_1,...)=\sum_{i=0}^na_iX^i\]
			\item Dann gilt für Addition und Multiplkation:
			\begin{align*}
			\sum_{k}a_kX^k+\sum_kb_kX^k&=\sum_k(a_k+b_k)X^k
			\left(\sum_{k}a_kX^k\right)\left(\sum_kb_kX^k\right)&=\sum_k c_kX^k
			\end{align*}
			mit $c_k=\sum_{i+j=k}a_ib_j$.
		\end{enumerate}
	\end{prop}

	\begin{definition}
		\begin{enumerate}
			\item Für ein Polynom $f=\sum_ka_kX^k$ heißt $a_k$ der $k$-te \textbf{Koeffizient} von $f$.
			\item Für $f\neq0$ heißt
			\[\deg(f)=\max\{i\mid a_i\neq 0\}\]
			der \textbf{Grad} von $f$. (Falls $f=0$, dann ist $\deg f:=-\infty$)
			\item Der Koeffizient $a_n$ mit $n=\deg(f)$ heißt \textbf{Führender Koeffizient} von $f$.
			\item Ist der führende Koeffizient $a_n=1$, so heißt $f$ \textbf{normiert}
		\end{enumerate}
	\end{definition}

	\begin{theorem}
		Seien $f,g\in A[X]$. \begin{enumerate}
			\item Dann ist $\deg(f+g)\leq\max(\deg (f),\deg (g))$ und $\deg(fg)\leq\deg(f)+\deg(g)$.\\
			\item Sind die führenden Koeffizienten von $f$ oder $g$ keine Nullteiler, so idt $\deg(fg)=\deg(f)+\deg (g)$.
		\end{enumerate}
	\end{theorem}

	\begin{kor}\label{kor:IBPily1V}
		$A$ ist genau dann Integritätsbereich wenn $A[X]$ Integritätsbereich ist.\\
		In diesem Fall gilt $A[X]^*=A^*$.
	\end{kor}
	\begin{proof}
		\begin{description}
			\item[$\Leftarrow$] Ist $A[X]$ ein Integritätsbereich, dann ist insbesondere $A\subset A[X]$.
			\item[$\Rightarrow$] Sei $A$ ein Integritätsbereich. Dann gilt $\deg(fg)=\deg(f)+\deg(g)$. Sei zusätzlich $f,g\in A[X]$ mit $fg=0$, dann ist $\deg(fg)=-\infty$.\\
			Also muss $\deg(f)=-\infty$ oder $\deg(g)=-\infty$. Damit $f=0$ oder $g=0$. Also ist $A[X]$ Integritätsbereich.
		\end{description}
		Sei nun $fg=1$, dann ist $\deg(fg)=0$, also muss $\deg(f)=\deg(g)=0$. Dann sind $f,g\in A^*$.
	\end{proof}

	\begin{satz}
		Sei $f:A\to B$ ein Ringhomomorphismus und $b\in B$. \\
		Dann gibt es genau einen Homomorphismen $\varphi_A:A[X]\to B$ mit $\varphi_B\mid_A=\varphi$ und $\varphi_b(X)=b$.
	\end{satz}

	\begin{proof}
		\begin{description}
			\item[Existenz] Für $\varphi=\sum_j a_jX^j$ setze $\varphi_s(f)=\varphi_b(a)=\varphi(a)$ für alle $a\in A$ und $\varphi_b(X)=b$.\\
			Dann ist $\varphi_s$ ein Homomorphismus.
			\item[Eindeutigkeit] Sei $\psi:A[X]\to B$ ein zweiter Homomorphismus mit $\psi|_A=\varphi$ und $\psi(X)=b$. Dann ist
			\begin{align*}
			\psi(f)&=\psi\left(\sum_ja_jX^j\right)=\sum_j\psi(a_i)\psi(X)^j\\
			&=\sum_j\varphi(a_i)b^j=\varphi_b(f)
			\end{align*}
		\end{description}
	\end{proof}
	
	\begin{exm}
		Sei $I$ ein Ideal in $A$. Die Komposition $A\to A/I\to(A/I)[X]$ ist ein Ringhomomorphismus. Dieser induziert einen Ringhomomorphismus $\pi:A[X]\to(A/I)[X]$ mit $\pi(x)=x$.\\
		Diese Abbildung ist die Reduktion der Koeffizienten modulo $I$.
		\[\Kern(\pi)=\{\sum_ia_iX^i\mid a_i\in I\}=I[X]\]
		und somit
		\[(A/I)[X]\isom A[X/I[X]]\]
	\end{exm}

	\begin{lem}
		Es gilt $I$ ist Primideal in $A$ $\Leftrightarrow$ $I[X]$ ist Primideal in $A[X]$.
	\end{lem}

	\begin{theorem}
		Sei $g\in A[X]$, $g\neq 0$ mit führendem Koeffizient $b_n\in A^*$ und sei $f\in A[X]$.\\
		Dann existieren eindeutige Polynome $q,r\in A[X]$ mit $f=qg+r$ mit $\deg(r)<\deg(g)$.
	\end{theorem}
	\begin{proof}
		\begin{description}
			\item[Existenz] Falls $f=0$ dann gilt $q,r=0$. Falls $\deg(f)<\deg(g)$ dann ist $q=0,r=f$.\\
			Sei als $f\neq 0$, $m=\deg(f)$ $h=\deg(g)$ und $m\geq h$.\\
			$f,g$ lassen sich schreiben als
			\begin{align*}
			f&=\sum_{k=1}^{m}a_kX^k& g=\sum_{j=0}^{n}b_jX^j
			\end{align*}
			Dabei ist $b_n\in A^*$.\\
			Durch Induktion:
			\begin{description}
				\item[$m=0$] Dann gilt $f=a_0$, $g=b_0$. Dann $q=a_0b_0^{-1}$ und $r=0$.
				\item[$m\geq 1$] Definiere $h:=f-X^{m-n}a_mb_n^{-1}g$.\\
				Dann ist $\deg(h)\leq m-1$. Nach der Induktionshypothese gibt es $q,r\in A[X]$ mit $h=qg+r$ und $\deg(r)<\deg(g)$.\\
				Es folgt
				\[f=X^{m-n}a_mb_n^{-1}qg+r=(X^{m-n}a_mb_n^{-1}+q)g+r\]
			\end{description}
			\item[Eindeutigkeit] Angenommen es gibt $q_1,q_2,r_1,r_2\in A[X]$, sodass $f=gq_1+r_1=gq_2+r_2$ und $\deg(r_1),\deg(r_2)<\deg(g)$.\\
			Dann folgt aber
			\begin{align*}
			&\Rightarrow&(q_1-q_2)g&=r_2-r_1\\
			&\Rightarrow&\deg(q_1-q_2)+\deg(g)&=\underbrace{\deg(r_1-r_2)}_{<\deg(g)}\\
			&\Rightarrow&\deg(q_1-q_2)&\leq 0
			\end{align*}
			daraus folgt $q_1-q_2=0$ und damit $r_1=r_2$.
		\end{description}
	\end{proof}

	\begin{kor}
		Sei $K$ ein Körper. Dann ist $K[X]$ ein euklidischer Ring unter der $\deg$-Abbildung und somit ein Hauptidealring.\\
		Die Einheiten sind die konstruieren Polynome.
	\end{kor}

	\begin{satz}
		Sei $A$ ein Integritätsbereich. Dann ist
		\[\text{$A[X]$ ist Hauptidealring}\Leftrightarrow A\text{ ist Körper}\]
	\end{satz}

	\begin{proof}
		\begin{description}
			\item[$\Rightarrow$] bereits gezeigt.
			\item[$\Leftarrow$] Die Abbildung $\varphi_0:A[X]\to A$,$f\mapsto f(0)$ ist ein Ring-Homomorphismus. Dann ist
			\[\Kern(\varphi_0)=\{f=\sum_{k=0}^{n}a_kX^k\in A[X]\mid a_0=0\}\]
			Dann gilt $\Kern(\varphi_0)$ ist ein Primideal in $A[X]$ und somit ein maximales Ideal.\\
			Also ist $A[X]/\Kern(\varphi_0)$ ist Körper und $A[X]/\Kern(\varphi_0)\isom A$.\\
			Also ist $A$ Körper.
		\end{description}
	\end{proof}


%VL 08.11.2017
	\subsection{Nullstellen von Polynomen}
	\begin{definition}
		Sei $f\in A[X]$, $f\neq 0$.\\
		$a\in A$ heißt \textbf{Nullstelle} von $f$, wenn $f(a)=0$.
	\end{definition}

	\begin{satz}
		Sei $f\in A[X]$, $f\neq 0$ und $a\in A$. Dann gilt
		\[\text{$a$ ist Nullstelle von $f$}\Leftrightarrow (x-a)|f\]
	\end{satz}
	\begin{proof}
		\begin{description}
			\item[$\Rightarrow$] Sei $f(a)=0$. Division mit Rest liefert
			\[ f=q(x-a)+r\]
			mit $\deg(r)< 1$. Aus $f(a)=r$ folgt $(x-a)|f$
		\end{description}
	\end{proof}

	\begin{satz}
		Sei $f\in A[X]$, $f\neq 0$ ein Polynom das eine Nullstelle in $A$ hat.\\
		Dann gibt es paarweise verschiedene Elemente $a_1,...,a_m\in A$ und $n_1,...,n_m\in\N$ und ein Polynom $g\in A[X]$, welchen keine Nullstellen in $A$ hat, sodass
		\[f=g\prod_{i=1}^m(x-a_i)^{n_i}\]
		ist.\\
		Es gilt
		\[\sum_{i=1}^{m}n_i\leq\deg(f)\]
	\end{satz}
	\begin{proof}
		Teilen mit Rest.
	\end{proof}

	\begin{definition}
		Lässt sich $f\in A[X]$, $f\neq 0$ schreiben als
		\[f=c\prod_{i=1}^{m}(x-a_i)^{n_i}\]
		mit $c,a_1,...,a_m\in A$ und $n_1,...,n_m\in\N$, dann sag man $f$ \textbf{zerfällt in Linearfaktoren}.
	\end{definition}

	\begin{satz}\label{satz:polyNst}
		Sei $A$ ein Integritätsbereich. Dann hat $f\in A[X]$ mit $f\neq 0$ höchsten $n=\deg(f)$ verschiedene Nullstellen in $A$.
	\end{satz}
	\begin{proof}
		Durch Induktion über $n$:
		\begin{description}
			\item[Induktionanfang:] Sei $n=0$. (Konstantes Polynom $\Rightarrow$ keine Nullstelle)
			\item[Induktionsschritt:] Sei $n>0$. Ist $a\in A$ eine Nullstelle von $f$, so ist $f=g(x-a)$ mit $\deg(q)=n-1$.\\
			Sei $b\neq a$ eine weitere Nullstelle von $f$, dass ist $0=f(b)=q(b)(b-a)$.\\
			Da aber $(b\neq a)$ ist, muss $b$ Nullstelle von $q$ sein.\\
			Nach Induktionsannahme hat $q$ höchstens $n-1$ verschiedene Nullstellen.
		\end{description}
	\end{proof}

	\begin{kor}
		Sei $A$ ein unendlicher Integritätsbereich und $f\in A[X]$, $f\neq0$. Dann gibt es ein $a\in A$ mit $f(a)\neq 0$.
	\end{kor}

	\begin{exm}
		Sei $K$ ein endlicher Körper und sei
		\[f=\prod_{a\in K}(x-a)\]
		Dann ist $f(a)=0$ für alle $a\in K$.
	\end{exm}

	\begin{satz}
		Sei $G_1$ zyklische Gruppe der Ordnung $n_1$, $G_2$ zyklische Gruppe der Ordnung $n_2$.\\
		Sein $n_1,n_2$ Teilerfremd, so ist $G_1\times G_2$ zyklisch.
	\end{satz}

	\begin{proof}
		Sei $G_1=<x_1>$ und $G_2=<x_2>$. Die Abbildung
		\begin{align*}
		\Z&\to G_1\times G_2\\
		m&\mapsto (mx_1,mx_2)
		\end{align*}
		hat den Kern $n_1,n_2\in\Z$ und ist surjektiv nach \ref{ChiRestSatz}.\\
		Dann ist
		\[\Z/n_1n_2\Z\isom G_1\times G_2\] 
	\end{proof}

	\begin{theorem}
		Sei $K$ ein Körper und $G\subset K^*$ Untergruppe. Ist $G$ endlich, so ist $G$ zyklisch.
	\end{theorem}
	\begin{proof}
		Da $G$ einen endliche abelsche Gruppe ist zerfällt $G$ in
		\[g=\bigotimes_{\text{$p$ prim}}G_p\]
		Dabei ist $G_p=\{g\in G\mid \text{$g^q=1$ für ein $q=p^n$}\}$.\\
		\\
		Angenommen $G_p$ ist nicht zyklisch. Dann ist $\ord(g)\leq |G_p|$ für alle $g\in G_p$ und es gibt ein $q=p^n<|G_p|$ mit $g^q=1$ für alle $q\in G_p$.\\
		Dann hat aber das Polynom $X^q-1$ mehr als $q$ Nullstellen in $K$. Widerspruch!\\
		\\
		Also sind alle $G_p$ zyklisch. Dann folgt nach \ref{Satz davor}, dass $G$ zyklisch ist.
	\end{proof}

	\begin{kor}
		Ist $K$ endlicher Körper, so ist $K^*$ zyklisch.
	\end{kor}

	\begin{satz}
		Sie $A$ ein faktorieller Integritätsbereich mit Quotientenkörper $K$.\\
		Sei
		\[f=a_nX^n+a_{n-1}X^{n-1}+...+a_iX^1+a_0\]
		ein Polynom in $K[X]$.\\
		Ist $b=c/d$ eine Nullstelle von $f$ in $K$ mit teilerfremden $c,d$, so gilt
		\[\text{$c|a_0$ und $d|a_n$}\]
	\end{satz}
	\begin{proof}
		Aus $f(b)=0$ folgt\[a_n\left(c/d\right)_{n}+a_{n-1}(c/d)^{n-1}+...+a_0\]
		Dann ist (nach Multiplikation mit $d^n$)
		\[a_nc^n+a_{n-1}c^{n-1}d+...+a_nd^n=0\]
		Dann ist
		\begin{align*}
		a_nd^n&=c(...)\\
		a_nc^n&=d(...)
		\end{align*}
		Also gilt $c|a_0$ und $d|a_n$
	\end{proof}

	\begin{definition}
		Sei $f\in A[X]$, $f\neq 0$. Ist $a\in A$ eine Nullstelle von $f$, so gibt es ein $n\in\N$ mit
		\begin{align*}
		(x-a)^n&|f\\
		(x-a)^{n-1}&\not |f
		\end{align*} 
		Dann heißt $n$ die \textbf{Vielfachheit} oder \textbf{Multiplizität} von $a$ und man nennt $a$ eine \textbf{$n$-fache Nullstelle} von $f$.
	\end{definition}

	\begin{definition}
		Die Abbildung
		\begin{align*}
		D: A[X]&\to A[X]\\
		\sum_{j=0}^{n}a_jX^j&\mapsto\sum_{j=1}^n ja_jX^{j-1}
		\end{align*}
		Man schreibt $f':=D(f)$.
	\end{definition}
	\begin{lem}
		Seien $f,g\in A[X]$, $a,b\in A$
		Für die Ableitung $D$ gilt
		\begin{enumerate}
			\item $D(af+bg)=aD(f)+bD(g)$ (Linearität)
			\item $D(fg)=(Df)g+f(Dg)$ (Produktregel)
		\end{enumerate}
	\end{lem}

	\begin{satz}
		Sei $f\in A[X]$, $f\neq 0$. Sei $a\in A$ eine Nullstelle von $f$. Dann gilt
		\[\text{$a$ hat Vielfachheit $1$}\Leftrightarrow f'(a)\neq 0\]
	\end{satz}
	\begin{proof}
		Da $a$ eine Nullstelle von $f$ ist gilt
		\[f=q(x-a)\]
		für ein $q\in A[X]$. Es folgt
		\[f'=q+q'(X-a)\]
		und $a$ hat genau dann Vielfachheit 1, wenn $(x-a)\not| q$, also $(x-a)\not | f'$, bzw. $f'(a)\neq 0$.
	\end{proof}

	\begin{definition}
		Die Abbildung
		\begin{align*}
		\chi:\Z&\to A\\
		n&\mapsto n\cdot 1
		\end{align*}
		
			Ist ein Ringhomomorphismus und
		\[\Kern(\chi)=(n)=n\Z\]
		für ein $n\in\Z$, $n\geq 0$.\\
		$n$ heißt die \textbf{Charakteristik} von $A$ und man schreibt $n=\cha(A)$.
	\end{definition}

	\begin{lem}
		Ist $A$ ein Integritätsbereich, so ist $n=0$ oder $n$ ist prim.
	\end{lem}
	
	\begin{satz}
		Sei $K$ ein Körper und $f\in K[X]$ $f\neq $const, dann gilt
		\begin{enumerate}
			\item Ist $\cha(K)=0$, so gilt 
			\[\deg(f')=\deg(f)-1\]
			\item Ist $\cha(K)=p>0$, so gilt
			\[\deg(f')\leq \deg(f)-1\]
		\end{enumerate}
		Weiterhin gilt
		\[f'=0\Leftrightarrow f(X)=g(X^p)\text{ für ein $g\in K[X]$}\]
	\end{satz}

	%VL 13.11.2017
	%TODO?
	\subsection{Polynome mehrerer Veränderlicher}
	\begin{definition}
		Sei $A$ ein Ring. Dann ist der \emph{Polynomring in mehreren Variablen} $A[X_1,...,X_n]$ induktiv definiert als 
		\begin{align*}
		A[X_1,X_2]&:=A[X_1][X_2]\\
		A[X_1,...,X_n]&:=A[X_1,...,X_{n-1}][X_n]
		\end{align*}
		und ein Polynom $f\in A[X_1,...,X_n]$ lässt sich schrieben als
		\[f=\sum_{i_1,...,i_n}\underbrace{a_{i_1...i_n}}_{\in A}X_1^{i_1}...X_n^{i_n}\]
	\end{definition}

	\begin{definition}
		Die Elemente $X_1^{i_1}...X_n^{i_n}\in A[X_1,...,X_n]$ heißen \emph{primitve Monome}.
	\end{definition}

	\begin{definition}
		Der \emph{Grad} des Polynoms $f\in A[X_1,....,X_n]$ ist definiert als
		\[\deg(f)=\max\left\{\sum_{j=1}^{n}i_j\mid a_{i_1...i_n}\neq 0\right\}\]
		falls $f\neq 0$ und sonst $=-\infty$.
	\end{definition}

	\begin{definition}
		Ein Polynom $f\in A[X_1,...,X_n]$ heißt \emph{homogen vom Grad $m$}, falls alle Monome in $f$ Grad $m$ haben.
	\end{definition}

	\begin{satz}
		Sei $\varphi:A\to B$ ein Ringhomomorphismus und $b_1,...,b_n\in B$.\\
		Dann existiert genau ein Ringhomomorphismus $\psi:A[X_1,...,X_n]\to B$ mit $\psi|_A=\varphi$ und $\psi(X_i)=b_i$ für alle $i$.
	\end{satz}

	\begin{satz}
		Sei $B$ ein Ring und $A\subset B$ ein Unterring. Seien $b_1,...,b_n\in B$. Die Inklusion $\iota:A\hookrightarrow B$ lässt sich eindeutig fortsetzen zu einem Homomorphismus $\varphi:A[X_1,...,X_n]\to B$ mit $\varphi|_A=\iota$ und $\varphi(X_j)=b_j$.
	\end{satz}

	\begin{kor}
		Sei $B$ ein Ring und $A\subset B$ ein Unterring. Seien $b_1,...,b_n\in B$.\\
		Dann ist $A[b_1,...,b_n]$ der Kleinste Unterring von $B$ der $A$ und $b_1,...,b_n$ enthält.
	\end{kor}

	\begin{kor}
		Ist $\varphi:A[X_1,...,X_n]\to B$ mit $\varphi(X_j)=b_j$ injektiv, so ist $A[b_1,...,b_n]$ isomorph zu $A[X_1,...,X_n]$.
	\end{kor}

	\begin{definition}
		Ist $\varphi:A[X_1,...,X_n]\to B$ mit $\varphi(X_j)=b_j$ injektiv, so nennt man die $b_j$ \emph{algebraisch unabhängig}.\\
		Ist $\varphi$ nicht injektiv, so heißen die $b_j$ \emph{algebraisch abhängig}.
	\end{definition}

	\begin{satz}[\ref{kor:IBPily1V} für mehrere Variablen]
		Sei $A$ ein Ring
		\[\text{$A$ ist Integritätsbereich }\Leftrightarrow\text{ $A[X_1,...,X_n]$ ist Integritätsbereich}\]
	\end{satz}
	\begin{proof}
		Korollar \ref{kor:IBPily1V} iterativ anwenden.
	\end{proof}

	\begin{satz}
		Sei $A$ Integritätsbereich. Dann gilt
		\[A^*=A[X_1,...,X_n]^*\]
	\end{satz}

	\begin{satz}
		%TODO Entfernen für offizielle Fassung
		Es war einmal ein Integritätsbereich $A$. Der Integritätsbereich $A$ hatte unendliche Teilmengen $T_1,...,T_n\subset A$ als Freunde.\\
		Dann kam ein nettes $f\in A[X_1,...,X_n]$ für welches $f(t_1,...,t_n)=0$ für alle $t_1\in T_1,...,t_n\in T_n$ war.\\
		Der Held wusste sofort, dass $f=0$ gellten musste.\footnote{by Sandra}
	\end{satz}
	\begin{proof}
		Induktion über $n$:
		\begin{description}
			\item[$n=1$] durch Negation von $\ref{satz:polyNst}$.
			\item[$n>1$] Schreibe $f\in A[X_1,...,X_n]$ als
			\[f=\sum_{j=0}^n\underbrace{\varphi_j(X_1,...,X_{n-1})}_{\in A[X_1,...,X_n]}X_n^j\]
			Angenommen es existieren $t_1,...,t_{n-1}$, sodass $\varphi_j(t_1,...,t_{n-1})\neq 0$ für ein $j$.\\
			Dann ist $f$ ein Polynom welches unendlich viele verschieden Nullstellen hat aber $\neq 0$ ist. Widerspruch!\\
			Also muss $f=0$.
		\end{description}
	\end{proof}

	\begin{definition}
		Sei $I\neq\emptyset$ ein Indexmenge. Dann bezeichnet $\N^(I)$ die Menge der Form $(a_i)_{i\in I}$ mit $a\in\in\N_0$ und $a_i=0$ für fast alle $i\in I$.\\
		Die Addition auf $\N^{(I)}$ ist definiert durch
		\[(a_i)_{i\in I}+(b_i)_{i\in I}=(a_i+b_i)_{i\in I}\]
		mit neutralem Element $0=(0)_{i\in I}$.
	\end{definition}

	\begin{definition}
		Für Indexmengen $I$ ist $A[(X_i)_{i\in I}]$ definiert als die Menge der Abbildungen $\varphi:\N^{(I)}\to A$ mit $\varphi(\al)=0$ für fast alle $\al\in \N^{(I)}$, mit Addition und Multiplikation
		\begin{align*}
		(f+g)(\al)&=f(\al)+g(\al)\\
		(fg)(\al)&=\sum_{\substack{\beta+\gamma=\al\\\beta,\gamma\in N(I)}}f(\beta)g(\gamma)
		\end{align*}
		Dann ist $A[(X_i)_{i\in I}]$ ein Ring mit neutralem Element der Addition $0=0(\al)$ und der Multiplikation $e(\al)=1$ falls $\al=0$ und $e(\al)=0$ sonst.
	\end{definition}

	\begin{bem*}
		Einem Element $a\in A$ ordnen wir die Abbildung $\zeta$ mit \[\zeta(\al)\begin{cases}
		a&=\al=0\\
		0\al\neq 0
		\end{cases}\]
		Dies liefert eine Einbettung von $A$ in $A[(X_i)_{i\in I}]$ mit
		\[X^\al(\beta)=\begin{cases}
		0&\beta\neq \al\\
		1&\beta=\al
		\end{cases}\]
		Für ein beliebiges $f\in A[(X_i)_{i\in I}]$ ist dann
		\[\zeta=\sum_{\al\in\N^{(I)}}f(\al)X^\al\]
		und es gilt $X^\al X^\beta=X^{\al+\beta}$ und für $f=\sum f(\al)X^\al$ $g=\sum g(\al)X^\al$ ist
		\begin{align}
		f+g&=\sum\big(f(\al)+g(\al)\big)X^\al\\
		f\cdot g&=\sum h(\al)X^\al
		\text{mit }h(\al)&=\sum_{\beta+\gamma=\al}f(\beta)g(\gamma)
		\end{align}
	\end{bem*}

	\begin{bem}
		Für jedes $j\in I$ setzen wir $e_j=(b_i)_i$ mit $b_j=1$, $b_i=0$ f+r $j\neq i$.\\
		Dann könne wir ein beliebiges $\al=(a_i)_i$ schreiben als $\al=\sum a_ie_i$.\\
		Wir definieren $X^{e_i}:=X_i$. Dann ist
		\[X^\al=X^{\sum a_ie_i}=\prod X^{a_ie_i}=\prod X_i^{a_i}\]
		Die $X^\al$ sind die primitiven Monome in den Variablen $X_i$ und die Elemente aus $A[(X_i)_{i\in I}]$ lass sich eindeutig schreiben als
		\[\sum_{\al\in \N^{(I)}}c_\al\prod X_i^{a_i}\]
		mit eindeutig bestimmten Koeffizienten $c_\al$ die fast alle verschwinden.
	\end{bem}



	
	%VL 15.11.2017
	
	\subsection{Bewertungen}
	\begin{definition}
		Sei $K$ ein Körper. Ein \textbf{Betrag} auf $K$ ist eine Abbildung
		\[|\cdot|:K\to\R\] mit
		\begin{enumerate}
			\item $\abs{x}\geq 0$ und $\abs{x}=0\Leftrightarrow x=0$
			\item $\abs{xy}=\abs x\abs y$
			\item $\abs{x+y}\leq \abs x\abs y$
		\end{enumerate}
	\end{definition}
	\begin{definition}
		Ein Betrag $\abs{\cdot}$ heißt \textbf{Archimedisch}, wenn es $x,y\in K$ gibt, sodass
		\[\abs{x+y}>\max\{\abs x\abs y\}\]
		bzw \textbf{nicht-archimedisch}, wenn für alle $x,y$ gilt, dass$|x+y|\leq \max\{\abs x,\abs y\}$.
	\end{definition}

	\begin{satz}
		Sei $\abs\cdot$ ein  nicht-archimedischer Betrag auf $K$. Ist $\abs x\neq \abs y$, so gilt
		\[\abs{x+y}=\max\{\abs x,\abs y\}\]
	\end{satz}
	\begin{proof}
		Sei $\abs x\leq \abs y$. Dann ist 
		\[\abs{x+y}\leq\max\{\abs x,\abs y\}=\abs x\]
		Andererseits ist $x=(x+y)+(-y)$, sodass
		\[\abs x=\abs{(x+y)+(-y)}\leq\max\{\abs{x+y},\abs y\}=\abs{x+y}\]
		also $\abs x\leq\abs{x+y}$.
	\end{proof}

	\begin{definition}
		Sei $A$ ein Integritätsbereich. Eine \textbf{Bewertung} auf $A$ ist eine Abbildung
		\[\nu:A\to\R\cup\{\infty\}\]
		mit
		\begin{enumerate}
			\item $\nu(a)=\infty\Leftrightarrow a=0$
			\item $\nu(ab)=\nu(a)+\nu(b)$
			\item $\nu(a+b)\geq\min\{\nu(a),\nu(b)\}$
		\end{enumerate}
	\end{definition}

	\begin{satz}
		Sei $A$ ein Integritätsbereich und $\nu:A\to\R\cup\{\infty\}$ eine Bewertung auf $A$.
		\begin{enumerate}
			\item $\nu$ kann zu einer Bewertung auf dem Quotientenkörper $K$ von $A$ fortgesetzt werden, durch
			\[\nu(a/b)=\nu(a)-\nu(b)\]
			\item Sei $c\in\R$ und $c> 1$. Dann definiert
			\[\abs{x}=c^{-\nu(x)}\]
			einen nciht-archimedischen Betrag auf $K$.
		\end{enumerate}
	\end{satz}

	\begin{exmlist}
		\begin{exm}
			Sei $A$ ein faktorieller Integritätsbereich und $p\in A$ prim. Dann lässt sich ein beliebiges $a\in A\setminus\{0\}$ schreiben als
			 \[a=a'p^{\nu_p(a)}\]
			 mit $\gcd(a',p)=1$ und $\nu_p(a)\in\N_0$.\\
			 Mit der Bedingung, dass $\nu_p(0)=\infty$, ist die Abbildung
			 \[\nu:A\to\R\cup\{\infty\}\]
			 eine Bewertung auf $A$.\\
			 Diese setzt sich zu einer Bewertung auf dem Quotientenkörper fort.
		\end{exm}
		\begin{exm}
			Sei $p\in \Z$ eine positive Primzahl. Dann definiert
			\[\nu_p:\Z\to\R\cup\{\infty\}\]
			wie Oben einen Bewertung auf $\Z$. Diese setzt sich zu einer Bewertung auf $\Q$ fort. Man definiert für $x\in\Q$ 
			\[\abs{x}_p:=p^{-\nu_p(x)}\]
			Dies liefert einen Betrag auf $Q$.\\
			\\
			Sei $x\in\Q$. Schreibe $x=a/bp^n$ mit $p\not|ab$. Dann ist $\abs{x}_p=p^{-n}$ und die Folge $1,p,p^2,...$ ist eine Nullfolge, bzgl $\abs{\cdot}_p$.\\
			Die Vervollständigung von $Q$ bezüglich $\abs{\cdot}_p$ ist isomorph zu $\Q_p$.
		\end{exm}
	\end{exmlist}

	\begin{theorem}[Lemma von Gauß]
		Sei $A$ in Integritätsbereich mit Quotientenkörper $K$ und sei $\nu:A\to\R\cup\infty$ eine Bewertung auf $A$.\\
		Setze $\nu$ fort zu einer Bewertung auf $K$ durch
		\[\nu(a/b)=\nu(a)-\nu(b)\]
		Für $f=\sum a_jX^j\in K[X]$ definieren wir
		\[\nu(f)=\min\{\nu(a_i)\}\]
		für $f\neq0$ und $\nu(0)=\infty$.\\
		Dann ist $\nu$ eine Bewertung auf $K[X]$.
	\end{theorem}
	\begin{proof}

		Wir zeigen
		\[\nu(fg)=\nu(f)+\nu(g).\]
		\begin{itemize}
			\item Seien $f,g$ Konstant, dann ist die Aussage klar.\\
			\item Sei nun $g=c\in K$. Dann ist 
				\begin{align*}
				\nu(gf)&=\nu(cf)\\
				&=\min\{\nu(ca_i)\}=\min\{\nu(c)+\nu(a_i)\}\\
				&=\nu(c)+\min\{\nu(a_i)\}\\
				&=\nu(g)+\nu(f)
				\end{align*}
			\item Seien nun $f,g$ nicht Konstant.\\
			Durch multipliaktion mit geeigenter Konstante können wir erreichen, dass 
			\[\nu(f)=\nu(g)=0\]
			Es ist zu zeigen, dass $\nu (fg)=0$.\\
			Sei dazu $f=\sum_{i=0}^{n}a_iX^i$, $g=\sum_{j=0}^mb_jx^j$. Dann ist
			\[fg=\sum_{k=0}^{m+n}c_kX^k\]
			mit
			\[c_k=\sum_{i+j=k}a_ib_j\]
			Es gilt
			\[\nu(c_k)\geq \min\{\underbrace{\nu(a_ib_j)}_{=\nu(a_i)+\nu(b_j)\geq 0}\}\geq 0\]
			sodass $\nu(fg)\geq 0$.\\
			\\
			Aus $c_{s+t}=a_0b_{s+t}+a_1b_{s+t-1}+...+a_sb_t+...+a_{s+t}b_0$ folgt
			\[a_sb_t=c_{s+t}-a_0b_{s+t}-a_1b_{s+t-1}-....-a_{s+t}b_0\]
			Dann ist also
			\[\nu(a_sb_t)\geq \min\{\nu(c_{s-t}),\underbrace{\nu(a_0b_{s+t})}_{=\nu(a_0)+\nu(b_{s+t}>0)},...,\nu(a_{s+t}b_0)\}>0\]
			damit $\nu(a_s)+\nu(b_t)>0$. Widerspruch!
		\end{itemize}
	\end{proof}

	\subsection{Der Satz von Gauß}

	\begin{definition}
		Sei $A$ ein faktorieller Integritätsbereich mit Quotientenkörper $K$.\\
		Ein Polynom
		\[f=a_nX^n+a_{n-1}X^{n-1}+...+a_o\in A_[X]\]
		heißt \textbf{primitiv}, wenn für seine Koeffizienten gilt: $\gcd(a_0,...,a_n)=1$.\\
		Äquivalent dazu $\nu_p(f)=1$ für alle Primelemente $p\in A$.\\
		Ein Polynom $f\in K[X]$, $f\neq 0$ lässt sich schreiben als $f=c\tilde f$ mit $\tilde f\in A[X]$ primitiv und $c\in K$.
	\end{definition}
	
	%VL 20.11.2017
	\begin{satz}\label{vorsatzvong}
		Sei $A$ ein faktorieller Integritätsbereich mit Quotientenkörper $K$ und $f\in A[X]$ primitiv mit $\deg(f)\geq1$.\\
		Dann gilt
		\[\text{$f$ ist irreduzibel in $A[X]$ }\Leftrightarrow \text{ $f$ ist irreduzibel in $K[X]$}\]
	\end{satz}
	\begin{proof}
		\begin{description}
			\item[$\Rightarrow$] Sei $f$ irreduzibel in $A[X]$. Sei $f=gh$ eine Zerlegung von $f$ in $K[X]$.\\
			Schreibe
			\begin{align*}
			g&=c\tilde g&h&=d\tilde h
			\end{align*}
			mit $\tilde{g},\tilde h\in A[X]$ primitiv. Dann ist
			\[f=cd\tilde g\tilde h\]
			und insbesondere
			\[\underbrace{\nu_p(f)}_{\geq 0}=\nu_p(cd)+\underbrace{\nu_p(\tilde g)}_{=0}+\underbrace{\nu_p(\tilde h)}_{=0}\]
			Also $\nu_p(cd)\geq 0$ für alle $p\in A$ prim.\\
			Dann muss aber die Potenz von jedem Primfaktor des Nenners $=0$ sein.\\
			Also ist $a=cd\in A$. Da $A[X]^*=A^*$ und $f=a\tilde g\tilde h$ und da $f$ irreduzibel ist muss $a\tilde g$ oder $\tilde h$ eine Einheit in $A[X]$ sein.\\
			Dann ist $a\tilde g$ oder $\tilde h$ in $A^*$, also $g$ oder $h$ Konstant und somit in $K^*=K[X]^*$.
			\item[$\Leftarrow$] Sei $f$ irreduzibel in $K[X]$. Sei $f=gh$ in $A[X]$. Dann ist $g$ oder $h$ in $K[X^*]$, also konstant.\\
			Sei $g=c$ für ein $c\in A$, dann ist
			\[\nu_p(f)=\nu_p(c)+\nu_p(h)\]
			Da $f$ primitiv ist, ist $\nu_p(f)=0$.\\
			Dann gilt $\nu_p(c)=\nu_p(h)=0$ für alle $p\in A$ prim.\\
			Also muss $c\in A^*=A[X]^*$.
		\end{description}
	\end{proof}

	\begin{bem*}
		Sei $A$ wie Oben, $f\in A[X]$, nicht zwingend Primitiv mit $\deg(f)\geq1$ und $f$ irreduzible in $K[X]$, dann ist $f$irreduzibel in $A[X]$.
	\end{bem*}

	\begin{theorem}[Satz von Gauß]
		Sei $A$ ein faktorieller Integritätsbereich. Dann ist auch $A[X]$ ein faktorieller Integritätsbereich.
	\end{theorem}
	\begin{proof}
		Sei $K$ der Quotientenkörper von $A$. Sei $f\in A[X]\setminus(A[X^*]\cup\{0\})$.\\
		Wir zeigen, dass $f$ über $A[X]$ in irreduzible Faktoren zerfällt.\\
		Wir schreiben $f=c\tilde f$ mit $\tilde f\in A[X]$ primitiv und $c\in A$.\\
		$c$ zerfällt in $A$ in irreduzible Faktoren.\\
		Diese sind auch irreduzibel in $A[X]$.\\
		Da $K[X]$ auch faktoriell ist, zerfällt $\tilde f$ in $K[X]$ in irreduzible Faktoren $\tilde f=\tilde f_1\cdot...\cdot\tilde f_n$ mit $\deg(\tilde f_i)\geq 1$.\\
		Es gibt insbesondere eine Zerlegung
		\[\tilde f=d\tilde f_1\cdot...\cdot \tilde f_n\]
		mit $d\in K$ und $\tilde f_i\in A[X]$ primitiv und $\deg(\tilde f_i)\geq 1$.\\
		Mit \ref{vorsatzvong} sind die $\tilde f_i$ auch irreduzible in $A[X]$.\\
		Aus
		\[\underbrace{\nu_p(\tilde f)}_{=0}=\nu_p(d)+\underbrace{\nu_p(\tilde f_1)}_{=0}+...+\underbrace{\nu_p(\tilde f_n)}_{=0}\]
		folgt $\nu_p(d)=0$ für alle $p\in A$ prim.\\
		\\
		Jetzt ist noch zu zeigen, dass die gefundenen Zerlegung eindeutig ist. Se
		\begin{align*}
		f&=c_1\cdot..\cdot c_mg_1\cdot...\cdot g_r\\
		&=d_1\cdot...\cdot d_nh_1\cdot ...\cdot h_s
		\end{align*}
		mit $c_i,d_j\in A$ irreduzibel und $g_i.h_j\in A[X]$ irreduzibel mit $\deg\geq 1$.\\
		Dann ist
		\[c/d\cdot g_1\cdot...\cdot g_r=h_1\cdot...\cdot h_s\]
		mit $c=c_1\cdot...\cdot c_m$, $d=d_1\cdot...\cdot d_n$ sind die $g_i,h_j$ irreduzible in $A[X]$ und somit auch in $K[X]$.\\
		Da $K[X]$ faktoriell ist, ist $r=s$ und nach Umsortierung ist
		\begin{align*}
		c/d\cdot g_1&=x_1h_1\\
		g_j&=x_jh_j
		\end{align*}
		für alle $j>1$.\\
		Dann ist
		\begin{align*}
		\nu_p(c/d)+\underbrace{\nu_p(g_1)}_{=0}&=\nu_p(x_1)+\underbrace{\nu_p(h_1)}_{=0}\\
		\nu_p(x_i)-\nu_p(c/d)&=0\\
		\nu_p(x_i\cdot d/c)&=0
		\end{align*}
		Wir definieren $\epsilon_1:=x_i\cdot d/c$. Dann ist $\epsilon_1\in A^*$.\\
		Zusätzlich ist
		\[\underbrace{\nu_p(g)}_{\geq 0}=\nu_p(x_j)+\underbrace{\nu_p(h_j)}_{=0}\]
		Sei $\epsilon_j=x_j$ für $j\geq 1$.
		Dann ist $\epsilon_j=x_j\in A^*$.\\
		Also ist
		\[g_i=\underbrace{\epsilon_i}_{\in A^*}h_i\]
		Weiterhin folgt $c=\epsilon d$ für ein $\epsilon\in A^*$.\\
		\\
		Da $A$ faktoriell ist, gilt $m=n$ und nach Umnummerieren $c_i\eta_id_i$ mit $\eta_id\in A^*$.
	\end{proof}

	\begin{kor}
		Sie $K$ ein Körper, dann ist $K[X_1,....,X_n]$ ein faktorieller Integritätsbereich.
	\end{kor}

	\begin{exmlist}
		\begin{exm}
			$\Z[X]$ ist ein faktorieller Integritätsbereich aber kein Hauptidealring.
		\end{exm}
		\begin{exm}
			Sei $K$ ein Körper. $K[X]$ ist ein Hauptidealring und somit faktorieller. $K[X,Y]$ ist kein Hauptidealring aber faktoriell.
		\end{exm}
	\end{exmlist}
	
	
	\subsection{Der Hilbertsche Basissatz}
	\begin{theorem}[Hilbertscher Basissatz]
		Sei $A$ ein noetherscher Ring. Dann ist auch $A[X]$ noethersch.
	\end{theorem}
	\begin{proof}
		Sei $I\subset A[X]$ ein Ideal. Wr zeigen, dass $I$ endlich erzeugt ist. Für $n\in\N_0$ sei
		\[I_n:=\{f\in I\mid\deg(f)\leq n\}\]
	\end{proof}
	Für $f=\sum_{a_iX^i\in A[X]}$ sei $b_n(f)=a_n$.\\
	Dann gilt
	\begin{align*}
	b_n(f+g)&=n_b(f)+b_n(g)\\
	b_n(af)&=ab_n(f)
	\end{align*}
	für alle $f,g\in A[X]$ und $a\in A$.\\
	Die Menge $I(n):=b_n(I_n)$ ist ein Ideal in $A$ und es gilt
	\[I(0)\subset I(1)\subset...\]
	den $f\in I_n$ impliziert $Xf\in I_{n+1}$. Dann ist $b_n(f)=b_{n+1}(Xf)\in I(n+1)$.\\
	Da $A$ noethersch ist wird jede Folge stationär. Also gibt es $m\in\N$, mit 
	\[I(m)=I(m+1)=...\]
	Für jedes$ n=0,1,....$ wähle Polynome $f_{n_j}$, sodass $I(n)$ von den Koeffizienten $b_n(f_{n_j})$ erzeugt wird.\\
	Dann wird $I$ von den $f_{n_j}$ über $A[X]$ erzeugt:\\
	Sei $f\in I$ vom Grad $t$. 
	\begin{itemize}
		\item Ist $t\leq m$, so hat
		\[f-\sum_t a_{t_j} f_{t_j}\in I\]
		Grad $\leq t-1$.\\
		Nach endlich vielen Schritten hat man $f$ als Linearkombination der $f_{n_j}$ dargestellt. 
		\item Ist $t>m$, so reduziert man den Grad von $f$ durch
		\[f-\sum a_{t_j}X^{t-m}f_{m_j}\in I\]
	\end{itemize}

	\subsection{Eigenschaften von Polynomringen}
	Sei $A$ ein Ring.
	\begin{enumerate}
		\item $A$ Integritätsbereich $\Leftrightarrow$ $A[X_1,...,X_n]$ Integritätsbereich.\\
		Dann gilt $A[X-1,...,X_n]^*=A^*$.
		\item (Gauss) $A$ faktorieller Integritätsbereich $\Leftrightarrow$ $A[X_1,...,X_n]$ faktorieller Integritätsbereich.
		\item (Hilbert) $A$ noethersch $\Leftrightarrow$ $A[X_1,...,X_n]$ noethersch.
		\item Sei $A$ zusätzlich Integritätsbereich, dann ist\\
		$A$ Körper $\Leftrightarrow$ $A[X]$ Hauptidealring.
		%TODO Referenzen angeben
	\end{enumerate}

	\subsection{Irreduziblitätskriterien}
	\begin{theorem}[Eisenstein]
		Sei $A$ ein faktoriell Integritätsbereich mit Quotientenkörper $K=Q(A)$.\\
		Sei
		\[f=a_nX^n+...+a_0\in A[X]\]
		mit $\deg(f)=n\geq 1$. Sei $p\in A$ prim mit $p|a_i$ für $i=0,...,n-1$ und $a\not|a_n$ und $p^2\not |a_0$.\\
		Dann ist $f$ irreduzibel in $K[X]$.\\
		Ist $f$ zusätzlich primitiv, so ist $f$ auch irreduzibel in $A[X]$.
	\end{theorem}
	\begin{proof}
		Sei $f=c\tilde f$ mit $\tilde f\in A[X]$ primitiv und $c\in A$.\\
		Es reciht zu ziegen, dass $\tilde f$ irreduzibel in $A[X]$ ist.\\
		Angenommen $f=gh$ mit $g,h\in A[X]\setminus A$. Sei
		\begin{align*}
		\tilde f&=\sum_{k=0}^{n}\tilde a_kX^k\\
		g&=\sum_{k=0}^{s}b_kX^k\\
		h&=\sum_{k=0} a_kX^k
		\end{align*}
		Dann folgt aus $p\not|a_n$, dass $p\not|c$ und aus $p|a_0$, dass $p|\tilde a_0=b_0d_0$.\\
		Wir können annehmen. dass $p|b_0$.\\
		Aus $p^2\not |a_0$ folgt, das $p\not| d_0$. Es gibt aber $j$, sodass $p\not|b_j$ (da sonst $p|g$).\\
		Wähle nun $j$, sodass $p|b_i$ für alle $i<j$ und $p\not| b_j$.\\
		Dann muss $1\leq j\leq s\leq n$. Aus 
		\[\tilde a_j=b_0d_j+b_1d_{j-1}+...+b_jd_0\]
		folgt, (da $p|\tilde a_j$), dass $p|b_jd_0$ und $p|d_0$. Widerspruch!.
	\end{proof}

	\begin{exm}
		Sei $p\in\Z$ eine positive Primzahl. dann ist das $p$-te Kreisteilunsgpolynom
		\[f=X^{p-1}+X^{p-2}+...+1\]
		irreduzibel in $Z[X]$.
	\end{exm}

	\begin{satz}[Reduktionskriterium]
		Sei $A$ ein faktorieller  Integritätsbereich mit Quotientenkörper $K$, $p\in A$ prim und $d=a_nX^n+...+a_0$ ein Polynom in $A[X]$ mit $\deg(f)\geq 1$ und $\neq a_n$.\\
		Sei
		\[\pi:A[X]\to\big(A/(p)\big)[X]\]
		und $\pi(f)$ irreduzible in $\big(A/(p)\big)[X]$, dann ist $f$ irreduzibel in $K[X]$.
	\end{satz}
	\begin{proof}
		Wir nehmen an, dass $f$ primitiv ist.\\
		Ist $f$ reduzibel über $K[X]$ so auch über $A[X]$.\\
		Sei $f=gh$ mit $g,h\in A[X]\setminus A$. Da $p$ den höchsten Koeffizienten von $f$ nicht teilt, gilt dies auch für $g$ und $h$ dun es gilt
		\[\pi(f)=\pi(gh)=\pi(g)\pi(h)\]
		d.h. $\pi(f)$ zerfällt in $\big(A/(p)\big)[X]$.\\
		Sei $f$ nun beliebig. Schreibe $f=c\tilde f$ mit $x\in A$ und $\tilde f\in A[X]$ primitiv.\\
		Angenommen $f$ ist nicht irreduzibel in $K[X]$, dann gilt $f$ reduzibel in $K[X]$ $\Rightarrow$ $\tilde f$ ist reduzibel in $K[X]$ $\Rightarrow $ $\tilde f$ ist reduzzibel in $A[X]$ $\Rightarrow$ $\tilde f=gh$ mit $g,h\in A[X]\setminus A$ $\Rightarrow$ $f=cgh$.\\
		Somit ist
		\[\pi(f)=\pi(cg)\pi(h)\]
		eine Zerlegung von $\pi(f)$.
	\end{proof}

	\begin{exmlist}
		\begin{exm}
			Wir zeigen, dass $F=X^2+3X^2$ irreduzibel in $\Q[X]$ ist. Wir fassen $f$ als Polynom über $\Z$ auf und reduzierten die Koeffizienten $\mod 3$.\[\pi(f)=X^3-X-1\]
			Da $\pi(f)(t)\neq 0$ für alle $t\in\Pi_3$ ist, ist $\pi(f)$ irreduzibel über $\Pi_3$ und somit auch über $\Q$.
		\end{exm}
		\begin{exm}
			Das Polynom $f=X^4+1$ ist irreduzibel in $\Q[X]$ und in $Z[X]$. Allerdings ist $\pi(f)\in\Pi_p[X]$ reduzibel für alle positiven Primzahlen $p$.
		\end{exm}
	\end{exmlist}

	\subsection{Symmetrische Polynome}
	\begin{definition}
		Für $f\in A[X_1,...,X_n]$ und $\sigma\in S_n$ sei
		\[\sigma(f)=\sigma(f(X_1,...,X_n)):=f(X_{\sigma(1)},...,X_{\sigma(n)})\]
		Dies liefert eine Operation von $S_n$ auf $A[X_1,...,X_n]$.\\
	\end{definition}

	\begin{bem}
		Insbesondere gilt für $\sigma,\tau\in S_n$, dass $(\sigma\tau)(f)=\sigma(\tau(f))$.
	\end{bem}

%VL 27.11.2017

	\begin{definition}
		Die Polynome in $A[X_1,...,X_n]^{S_n}$ (invariant unter $S_n$) werden als \textbf{symmetrische Polynome} bezeichnet.
	\end{definition}
	
	\begin{prop}
		Die Gruppenoperationen $\sigma\in S_n$ sind Automorphismen auf $A[X_1,...,X_n][X]$.
	\end{prop}
	
	\begin{satz}
		\begin{enumerate}
			\item $A[X_1,...X_n]^{S_n}$ enthält $A$ und ist ein Unterring von $A[X_1,...,X_n]$.\\
			\item $S_n$ operiert auf $A[X_1,...,X_n][X]$ durch
			\[\sigma\left(\sum_{j=0}^n a_jX^j\right)=\sum_{j=0}^{n}\sigma(a_j)X^j\]
			\item Sei $f=(X-X_1)(X-X_2)...(X-X_n)$. Dann ist
			\[f=X^n+\sum_{j=1}^{n}(-1)^js_jX^{n-j}\]
			für eindeutig bestimmte Polynome $s_j\in A[X_1,...,X_n]$
			\item $\sigma(f)=f$

		\end{enumerate}
	\end{satz}

	\begin{definition}
		Sei $f\in[X-1,...,X_n][X]$, $\sigma\in S_n$.\\
		Dann bezeichnet man die $s_j$ in
		\[f=\sigma(f)=\sigma\left(X^n+\sum_{j=1}^{n}(-1)^js_jX^{n-j}\right)\]
		als \textbf{elementarsymmetrische Polynome}.
	\end{definition}

	\begin{lem}
		Die elementarsymmetrischen Polynom sind symmetrisch, d.h. $\sigma(s_j)=s_j$. Sie sind gegeben durch
		\begin{align*}
		s_1&=X_1+X_2+...+X_n
		s_2&=X_1X_1+X_1X_3+...+X_1X_n+X_2X_3+...+X_{n-1}X_n\\
		&=\sum_{i\leq j} X_iX_j
		\vdots&\\
		s_n&=X_1...X_n
		\end{align*}
	\end{lem}

	\begin{satz}
		Die Polynome $s_j$ sind homogen vom Grad $j$.\\
	\end{satz}

	\begin{definition}
		Das Monom $X_1^{i_1}...X_n^{i_n}\in A[X_1,...,X_n]$ hat Grad $i_1+...+i_n$.\\
		Für den \textbf{Grad} $\deg(f)$ für $f\in A[X_1,...,X_n]$ ist das Maximum über den Grad der Monome.
	\end{definition}

	\begin{definition}
		Das Monom $X_1^{i_1}...X_n^{i_n}\in A[X_1,...,X_n]$ hat Gewicht $i_1+2i_2+...+ni_n$.\\
		Das Gewicht $\weight(f)$ für $f\in A[X_1,...,X_n]$ ist das Maximum über das Gewicht der Monome.
	\end{definition}

	\begin{theorem}
		\begin{enumerate}
			\item Sei $f\in A[X_1,..,X_n]^{S_n}$ mit  $\deg(f)=d$.\\
			Dann gibt es eine Polynom $g\in A[X_1,...,X_n]$ mit $\weight(g)\leq d$, sodass $f=g(s_1,...,s_n)$.\\
			\item Ist $f$ zusätzlich homogen, so hat jedes Monom Gewicht $d$.
		\end{enumerate}

	\end{theorem}

	\begin{proof}
		\begin{enumerate}
			\item Wir beweisen durch vollständige Induktion über $n$.
			Für $n=1$ gilt die Behauptung, da $s_1=x_1$.\\
			Angenommen die Behauptung gilt für Polynome in $A[X_1,...,X_{n_1}]^{S_{n-1}}$.\\
			Sei $f\in A[X-1,...,X_n]$. Es ist zu zeigen, dass $f$ ein Polynom in $s_1,...,s_n$ ist.\\
			Setzt man $X_n=0$ in 
			\[\prod_{j=1}^{n}(X-X_j)=X^n+\sum_{j=1}^{n}(-1)^js_jX^{n-j}\]
			für $s_j=s_j(X_1,...,X_n)$, so erhält man
			\[(X-X_1)(X-X_2)...(X-X_{n-1})(X)=X^n\sum_{j=1}^{n}(-1)^j (s_j)_0X^{n-j}\]
			mit $(s_j)_0:=s_j(X_1,...,X_{n-1},0)$.\\
			Andererseits ist
			\[(X-X_1)(X-X_2)...(X-X_{n-1})(X)=X\left(X^{n-1}\sum_{j=1}^{n-1}(-1)^j \tilde s_j X^{n-1-j}\right)\]
			Dann muss aber $(s_1)_0=\tilde s_1$,...,$(s_{n-1})_0=\tilde{s}_{n-1}$ und $(s_n)_0=0$.\\
			Wir beweisen die Aussage durch Induktion über $d=\deg(f)$.\\
			Hat $f$ Grad $0$, so ist die Behauptung trivial.\\
			Sei also $\deg(f)=d>0$. Dann gibt es ein Polynom $g_1\in A[X_1,...,X_{n-1}]$ mit $\weight(g)\leq d$, sodass
			\[f(X_1,...,X_{n-1},0)=g_1\big((s_1)_0,...,(s_{n-1})_0\big)\]
			Grad $\leq d$ in $X_1,...,X_{n-1}$ hat, da $f(X_1,...,X_{n-1},0)$ symmetrisch unter $S_{n-1}$ ist.\\
			Das Polynom $g_1(s_1,...,s_{n-1})\big)$ hat Grad $\leq d$ in $X_1,...,X_n$ weil die $s_j$ homogen sind.\\
			Das Polynom
			\[f_1(X_1,...,X_n)=\underbrace{f(X_1,...,X_n)}_{\substack{\text{Grad $\leq d$}\\\text{in $X_1,...,X_n$}}}
			-\underbrace{g(s_1,...,s_{n-1})}_{\substack{\text{Grad $\leq d$}\\\text{in $X_1,...,X_n$}}}\]
			hat Grad $\leq d$ in $X_1,...,X_n$ uns ist symmetrisch. \\
			Aus $f_1(X_1,...,X_{n-1},0)\geq 0$ folgt $X_n|f_1$. Damit auch $X_i|f_1$ und somit $s_n|f_1$.\\
			Dann gibt es $f_2$, sodass $f_1=s_nf_2$.\\
			Dabei ist $f_2$ symmetrisch unter $S_n$ und hat Grad $\leq d-n$.\\
			Nach Induktionshypothese gibt es ein Polynom $g_2\in A[X_1,...,X_n]$ mit Gewicht $\leq d-n$, sodass
			\[f_2=g_2(s_1,...,s_n)\]
			Es folgt $f=f_1+g_1=s_ng_2+g_1$.\\
			Dann ist
			\[f(X_1,...,X_n)=g_1(s_1,...,s_{n-1})+s_ng_2(s_1,...,s_n)=g(s_1,...,s_n)\]
			mit
			\[g(X_1,...,X_n)=\underbrace{g_1(X_1,...,X_n)}_{\text{Gewicht $\leq d$}}+\underbrace{\underbrace{X_n}_{\text{Gew $n$}}\underbrace{g_2(X_1,....,X_n)}_{\text{Gew $\leq d-n$}}}_{\text{Gew $\leq d$}}\]
			\item Siehe Lang
		\end{enumerate}
	\end{proof}


	\begin{theorem}
		Sie elementarsymmetrischen Polynome $s_1,...,s_n\in A[X-1,...,X_n]$ sind algebraisch unabhängig über $A$.
	\end{theorem}
	\begin{proof}
		Durch Induktion über $n$.\\
		Für $n=1$ ist die Behauptung klar.\\
		Sei $n>1$ und die $s_1,...,s_n$ seien nicht algebraisch unabhängig.\\
		Wähle $f\in A[X_1,...,X_n]$ mit kleinstem Grad und $f\neq 0$, sodass
		\[f(s_1,...,s_n)=0\]
		Schreibe $f$ als Polynom in $X_n$ mit Koeffizienten in $A[X_1,...,X_{n_1}]$.\\
		\[f(X_1,...,X_n)=f_0(X_1,...,X_{n-1})+f_1(X_1,...,X_{n-1})X_n+...+f_d(X_1,...,X_{n-1})X_n^d\]
		Angenommen $f=X_n\psi$ für ein $\psi\in A[X-1,...,X_n]$ und $x_n\psi(s_1,...,s_n)=0$, dann muss $\psi(s_1,...,s_n=0)$ sein.\\
		Dies ist ein Wiederspruch zu der Annahme, dass $f$ minimalen Grad hat.\\
		Also muss $f_0\neq 0$ sein.\\
		Wir setzen nun $x_i=s_i$ und erhalten
		\begin{align*}
		0&=f(s_1,...,s_n)\\
		&=f_0(s_1,...,s_n)+...+f_d(s_1,...,s_{n-1})s_n^d
		\end{align*}
		Nun setzen wir $X_n=0$. Dann ist 
		\[0=f\big((s_1)_0,...,(s_{n-1})_0\big)=f_0(\tilde s_1,...,\tilde s_{n-1})\]
		Nach Induktionshypothese sind die $\tilde s_1,...,\tilde s_n$ algebraisch unabhängig. Widerspruch!
	\end{proof}

	
	\begin{exm}
		Sei $n=3$ Dann ist $X_1^3+X-2^3+X_3^3$ ein symmetrisches Polynom. Es gilt \[X_1^3+X_2^3+X_3^3=s_1^3-3s_1s_2+3s_3\]
	\end{exm}

%VL 29.11.2017

	\begin{definition}
		Sei $f\in A[X]$ ein normiertes Polynom vom Grad $n$.\\
		Dann ist die \textbf{Diskriminante}  von $f$ definiert als
		\[D(f):=d_n(-c_1,c_2,-c_3,...,(-1)^nc_n)\in A\]
		Dabei ist $d_n\in\Z[X-1,...,X_n]$ mit
		\[d_n(s_1,...,s_n):=\prod_{i\leq j}(X_i-X_j)^2\]
		%TODO.. sauber Definieren
	\end{definition}

	\begin{satz}
		Sei $f\in A[X]$ ein normiertes Polynom. Ist
		\[f=\prod_{i=1}^n(X-\al i)\]
		eien Faktorisierung von $f$ in einem Oberring $B\supset A$, dann ist
		\[D(f)=\prod_{i\leq j}(\al_i-\al_j)^2\]
	\end{satz}
	\begin{proof}
		Es ist
		\[\prod_{i=1}^n=X^n+\sum_{i=1}^n(-1^i)s_iX^{n-i}\]
		so dass
		\[f=\prod_{i=1}^n(X-\al_i)=X^n+\sum_{i=1}^{n}(-1)^is_i(\al_1,...,\al_n)X^{n-1}=X^n+\sum_{i=1}^{n}x_iX^{n-i}\]
		d.h.
		\[c_i=(-1)^is_i(\al_1,..,\al_n)\]
		und
		\begin{align*}
		D(f)&=d_n(-c_1,c_2,...,(-1)^nc_n)\\
		&=d_n\big(s_1(\al_1,...,\al_n),...,s_n(\al_1,...,\al_n)\big)\\
		&=\prod_{i\leq j}(\al_1-\al_j)^2
		\end{align*}
	\end{proof}

	\begin{satz}
		Ist $B\supset A$ ein Integritätsbereich so gilt
		\[D(f)=0\Leftrightarrow\text{ $f$ hat Mehrfache Nullstellen in $B$}\]
	\end{satz}

	\begin{exmlist}
		\begin{exm}
			Für $f=X^2+aX+b$ ist $D(f)=a^2-4b$ (Wurzel der pq-Formel)
		\end{exm}
		\begin{exm}
			Für $f=X^3+aX+b$ ist $D(f)=-4a^3-27b^2$.
		\end{exm}
	\end{exmlist}


\section{Körpererweiterungen}
	\subsection{Grundbegriffe}
	
	\begin{definition}
		Sei $L$ ein Körper, $K\subset L$ heißt \textbf{Teilkörper} von $L$, wenn $K$ abgeschlossen beglich Addition und Multiplikation ist und unter diesen Operationen selbst wieder Körper ist.
	\end{definition}

	\begin{definition}
		Sei $K$ ein Körper. Sei $L\supset K$ selbst wieder Körper, dann bezeichnet man $L$ als \textbf{Erweiterungskörper} von $K$ und spricht von der \textbf{Körpererweiterung} $L/K$.
	\end{definition}

	\begin{definition}
		Sei $L/K$ eine Körpererweiterung. Dann heißt der Körper $M$ mit $K\subset M\subset L$ \textbf{Zwischenkörper} der Erweiterung $L/K$.
	\end{definition}

	\begin{definition}
		Sei $L/K$ eine Körpererweiterung und $M\subset L$. Dann bezeichnet man mit $K(M)$ den \textbf{kleinsten Teilkörper} von $L$, der $K\cup M$ enthält.\\
		Man sagt, dass $K(M)$ durch Adjunktion von $M$ zu $K$ entsteht.
	\end{definition}

	\begin{prop}
		Sei $L/K$ eine Körpererweiterung und $M\subset L$. Dann besteht $K(M)$ aus allen Elementen der Form
		\[\frac{f(a_1,...,a_n)}{g(a_1,...,a_n)}\]
		mit $f,g\in K[X_1,...,X_n]$, $g(a_1,...,a_n)\neq 0$ und $a_1,...,a_n\in M$.
	\end{prop}
	\begin{proof}[Beweisskizze]
		Die angegebenen Elemente bilden einen Teilkörper von $L$, der $K\cup M$ enthält und jeder Teilkörper von $L$ der $K\cup M$ enthält, enthält auch die angegebene Elemente.
	\end{proof}

	\begin{prop}
		Für jedes $a\in K(M)$ gibt es eine endliche Teilmenge $M'\subset M$, sodass $a\in K(M')$.
	\end{prop}

	\begin{definition}
		Sei $K$ ein Körper. Sei\begin{align*}
		\Z&\xrightarrow{\phi}K\\
		n&\mapsto n\cdot 1
		\end{align*}
		Dann ist $\Kern(\phi)=(n)$ für ein eindeutiges $n\in \N$. $n$ wird als \textbf{Charakteristik} von $K$ bezeichnet.
	\end{definition}

	\begin{kor}
		Sei $K$ ein Körper, dann ist $\cha(K)=0$ oder prim.
	\end{kor}
	\begin{proof}
		Da $\Z/(n)=\Z/\Kern(\phi)\isom\Img(f)\subset K$ keine Nullteiler hat.
	\end{proof}

	\begin{exm}
		\begin{enumerate}
			\item $\Q,\R,\C$ haben Charakteristik 0.
			\item Sei $p\in\Z$ eine positive Primzahl. Dann hat $\F_p=\Z/p\Z$ Charakteristik $p$.
		\end{enumerate}
	\end{exm}

	\begin{prop}
		Ist $K$ ein Teilkörper von $L$, so gilt
		\[\cha(K)=\cha(L)\]
	\end{prop}

	\begin{definition}
		Sei $K$ ein Körper. Dann heißt
		\[P:=\bigcap_{\text{$L$ Teilkörper von $K$}}L\]
		der \textbf{Primkörper} von $K$.
	\end{definition}

	\begin{satz}
		Sei $K$ ein Körper und $P$ der Primkörper von $K$. Dann gilt
		\begin{enumerate}
			\item $\cha(K)=p$ für $p>0$, $p$ prim $\Leftrightarrow$ $P\isom F_p$
			\item $\cha(K)=0$ $\Leftrightarrow$ $P\isom\Q$.
		\end{enumerate}
	\end{satz}

	\begin{definition}
		Ist $K$ ein Teilkörper von $L$, sokönnen wir $L$ als Vektorraum über $K$ auffassen.\\
		Die Dimension diese Vektorrausm heißt \textbf{Grad} von $L$ über $K$.
		\[[L:K]:=\dim_K(L)\]
	\end{definition}

	\begin{definition}
		Die Erweiterung $L/K$ heißt \textbf{endlich}, wenn $[L:K]<\infty$.\\
	\end{definition}

	\begin{prop}
		Ist $L$ endlich und $K$ kein Teilkörper von $L$, so gilt \[\abs{L}=\abs{K}^m\]
		mit $m=[L:K]$.
	\end{prop}

	\begin{theorem}[Gradsatz]
		Seien $K\subset L\subset M$ Körpererweitungen. Dann gilt
		\[[M:K]=[M:L][L:K]\]
		Ist $(x_i)_{i\in I}$ eine Basis von $L/K$ und $(y_j)_{j\in J}$ eine Basi von $M/L$, so ist $(x_iy_j)_{(i,j)\in I\times J}$ eine Basis von $M/K$.
	\end{theorem}
	\begin{proof}
		Es reicht die zweite Behauptung zu zeigen.\\
		Sei $z\in M$.Dann ist
		\[z=\sum_{j\in J}a_jy_j\]
		mit $a_j\in L$ und $a_j=0$ für fast alle $j\in J$.\\
		Wir können $a_j$ schreibe als
		\[a_j=\sum_{i\in I}b_{ji}x_i\]
		mit $b_{ij}\in K$ und $j_{ij}=0$ für fast alle $i\in I$.\\
		Also ist
		\[z=\sum_{i\in I,j\in J}b_{ij}x_iy_j\]
		d.h. $(x_i,y_j)_{(i,j)\in I\times J}$ ist ein Erzeugendensystem von $M/K$.\\
		Wir zeigen, dass die Vektoren $x_i,y_i$ linear unabhängig über $K$ sind.\\
		Sei
		\[\sum_{i,j}\underbrace{c_{ij}}_{\in K}\underbrace{x_i}_{\in K}\underbrace{y_i}_{\in M}=0\]
		Dann gilt für jedes $j$, dass
		\[\sum_{i\in I}c_{ij}x_i=0\]
		weil die $y_i$ linear unabhängig über $L$ sind.\\
		Aus der linearen Unabhängigkeit der $x_i$ über $K$ folgt $c_{ij}=0$.
	\end{proof}

	\subsection{Algebraische Körpererweiterungen}
	\begin{definition}
		Sei $L/K$ eine Körpererweiterung. $\al\in L$ heißt \textbf{algebraische} über $K$, wenn es ein Polynom $g\in K[X]\setminus 0$ gibt, mit $g(\al)=0$.\\
		Äquivalent:Der Homomorphismus $K[X]\to L$,$f\mapsto f\al$ hat nicht trivialen Kern.
	\end{definition}

	%VL 04.12.2017

	\begin{definition}
		Ist $\al\in L$ nicht algebraisch, so nennt man es \emph{transzendent}.
	\end{definition}

	\begin{definition}
		Der Körper $L$ heißt \emph{algebraisch} über $K$, wenn alle $\al\in A$ algebraisch sind.
	\end{definition}

	\begin{exm}
		$\pi$ und $e$ sind transzendent über $\Q$.
	\end{exm}

	\begin{definition}
		Sei $L/K$ eine Körpererweiterung und $\al\in L$ algebraisch über $K$. Sei $m_{\al,K}\in K[X]$ normiert und erzeuge den Kern von $\varphi:K[X]\to L,f\mapsto f(\al)$.\\
		Man nennt es das \emph{Minimalpolynom} in $\al$ über $K$.
	\end{definition}

	\begin{kor}
		$m_{\al,K}$ ist eindeutig und irreduzibel.
	\end{kor}

	\begin{satz}
		Sei $L/K$ eine Körpererweiterung und $a\in L$ algebraisch über $K$. Sei $\varphi:K[X]\to L$,$g\mapsto g(\al)$ und $\Img(\varphi)=K[\al]$.\\
		Dann induziert $\varphi$ eine Isomorphismus
		\[K[X]/(m_{\al,K})\isom K[\al]\]
		Insbesondere ist $K[\al]$ Körper und es gilt
		\[[K[\al]:K]=\deg(m_{\al,K})\]
	\end{satz}

	\begin{proof}
		Da $m_{\al,K}$ irreduzibel ist ist $(m_{\al,K})$ ein Maximales Ideal in $K[X]$ und damit $K[X]/(m_{\al,K})$ Körper.\\
		Aus der Isomorphie folgt dass $K[\al]$ Körper ist.\\ %TODO Woher Isomorphie?
		Sei nun $g\in K[X]$. Dann ist $g=qm_{\al,K}+r$ mit $q,r\in K[X]$ und $\deg(r)<\deg(m_{\al,K})$.\\
		Somit bilden die Restklassen $(m_{\al,K}),X+(m_{\al,K}),...,X^{n-1}(m_{\al,K})$ mit $n=\deg(m_{\al,K})$ eine Basis von $K[X]/(m_{\al,K})$ über $K$.\\
		Dann bildet auch $1,\al,...,\al^{n-1}$ eine Basis von $K[\al]$ über $K$.
	\end{proof}

	\begin{exm}
		Sei $n\in\Z$, $n>0$ und $p$ eine positive Primzahl. Dann ist $\sqrt[n]{p}$ eine Nullstelle von $f=X^n-p$.\\
		D.h. $\sqrt[n]{p}$ ist algebraisch über $\Q$.\\
		Das Eisenstein-Kriterium zeigt, dass $f$ ireduzibel über $\Q$ und $[\Q(\sqrt[n]{p}):\Q]=n$ ist.
	\end{exm}

	\begin{satz}
		Eine endliche Körpererweiterung $L/K$ ist algebraisch.
	\end{satz}

	\begin{proof}
		Sei $[L:K]=n$, $n<\infty$ und $\al\in L$.\\
		Dann sind ie Elemente $\al^0,\al^1,...a^n$ linear abhängig über $K$.\\
		(d.h. es existieren nicht trivial $c_0,...,c_n\in K$ sodass $c_0\al^0+...+c_n\al^n=0$.\\
		Also ist $\al$ Nullstelle des Polynoms un damit algebraisch.)
	\end{proof}

	\begin{definition}
		Eine Körpererweiterung heißt \emph{einfach}, wenn es $\al\in L$ gibt mit $L=K(\al)$.
	\end{definition}

	\begin{definition}
		Eine Körpererweiterung heißt \emph{endlich erzeugt}, wenn es endlich viele Element $\al_1,...,\al_n$ gibt sodass $L=K(\al_1,...,\al_n)$.
	\end{definition}

	\begin{satz}
		Sei $L=K(\al_1,...,\al_n)$ eine endlich erzeugte Körpererweiterung und $\al_1,...,\al_n$ algebraisch über $K$. Dann gilt
		\begin{enumerate}
			\item $L=K(\al_1,...,a_n))K[\al_1,...,\al_n]$
			\item $L$ ist endlich und somit insbesondere algebraische Erweiterung von $K$.
		\end{enumerate}
	\end{satz}

	\begin{proof}
		jeweils durch Induktion über $n$
		\begin{enumerate}
			\item Angenommen die Aussage gelte für $n$:
			\begin{align*}
				K[\al_1,...,\al_{n+1}]&=K[\al_1,...,\al_n][\al_{n+1}]\\
				&\overset{=}{IH}K(\al_1,...,\al_n)[\al_{n+1}]
				\intertext{Da $\al_{n+1}$ algebraisch über $K\subseteq K(\al_1,...,\al_n)$}
				&=K(\al_1,...,\al_n)(\al_{n+1})\\
				&=K(\al_1,...,\al_{n+1})
			\end{align*}
			\item Es gilt
			\[[K(\al_1,...,\al_{n+1}):K]=\underbrace{[K(\al_1,...,\al_n):K(\al_1,..,\al_n)]}_{<\infty}[K(\al_1,...,\al_n):K]<\infty\]
			da $\al_{n+1}$ algebraisch über $K$.
		\end{enumerate}
	\end{proof}
	
	\begin{kor}
		Sei $L/K$ eine Körpererweiterung. Dann sind äquivalent
		\begin{enumerate}
			\item $L/K$ ist endlich
			\item $L$ wird über $K$ von endlich vielen Elementen erzeugt.
			\item $L$ ist eine endlich erzeugte algebraische Erweiterung.
		\end{enumerate}
	\end{kor}

	\begin{kor}
		Sei $L/K$ eine Körpererweiterung. Dann sind äquivalent
		\begin{enumerate}
			\item $L/K$ ist algebraisch.
			\item $L$ wird über $K$ von algebraisch Elementen erzeugt.
		\end{enumerate}
	\end{kor}
	\begin{proof}
		\begin{description}
			\item[$1)\Rightarrow2)$] $K(L)=L$ ist algebraisch
			\item[$2)\Rightarrow1)$] Sei $K=L(m)$ für $ m\subset L$ algebraisch über $K$.\\
			Sei $\al\in L$. Dann gibt es $\al_1,...,\al_n\in M$ mit $\al\in K(\al_1,...,\al_n)$. Also ist $\al$ algebraisch über $K$.
		\end{description}
	\end{proof}

	\begin{satz}
		Seien $K\subset L\subset M$ Körpererweiterungen, sei $L/K$ algebraisch und $\al\in M$ algebraisch über $L$.\\
		Dann ist $\al$ algebraisch über $K$.\\
		Insbesondere:
		\[\text{$M/K$ ist algebraisch }\Leftrightarrow\text{$L/K$ ist algeraisch über $M/L$}\]
		%TODO?
	\end{satz}
	\begin{proof}
		Sei $\al$ algebraisch.  Dann existiert ein Polynom $G=\sum_{i=0}^{n}a_iX^i\in L[X]$ mit $g(\al)=0$.\\
		Also ist $a\in K(a_0,...,a_n,\al)=K(a_0,...,a_n)(\al)$.\\
		Durch $n$-faches anwenden des Gradsatzes folgt $[K(a_0,...,a_n):K]<\infty$.\\
		Da $\al$ algebraisch über $a_1,...,a_n$  gilt
		\[[K(a_0,...,a_n,\al):K(a_0,...,a_n)]<\infty\]
		Also ist auch $[K(a_0,...,a_n,\al):K]<\infty$ und daher folgt, dass $\al$ algebraisch über $K$ ist.
	\end{proof}
	
	\begin{definition}
		Sei $L/K$ eine Körpererweiterung. Dann ist
		\[L_{alg}:=\{a\in L\mid\text{$a$ algebraisch über $K$}\}\]
		der \emph{algebraische Abschluss} von $L$ in $K$.
	\end{definition}

	\begin{kor}
		Seien $a,b\in L$ algebraisch über $K$. Dann ist $K(a,b)$ algebraisch über $K$.\\
		Also ist auch $a-b\in K(a,b)$ algebraisch über $K$ und falls $b\neq0$ auch $ab^{-1}\in K(a,b)$.
	\end{kor}

	\begin{exm}
		Der algebraische Abschluss von $\Q$ in $\C$ ist algebraisch über $\Q$ aber nicht endlich über $\Q$da er u.a. die Teilkörper $\Q(\sqrt[n]{p})$ ($p$ prim) enthält.
	\end{exm}





	\subsection{Der algebraische Abschluss eines Körpers}
	
	\begin{satz}[von Kronecker]
		Sei $K$ ein Körper $f\in K[X]\setminus K$.\\
		Dann gibt es eine algebraische Erweiterung $L/K$ sodass $f$ eine Nullstelle in $L$ hat.\\
		Ist $f$ irreduzibel, so kann man $L=K[X]/(f)$ setzen.
	\end{satz}
	\begin{proof}
		Sei $f\in K[X]\setminus K$ und sei $f$ irreduzibel.\\
		Dann ist $(f)$ ein maximales Ideal in $K[X]$ und daher $K[X]/(f)$ Körper.\\
		Die Komposition $K\hookrightarrow K[X]\xrightarrow{\Pi} K[X]/(f)$ ist ein Homomorphismus von Körpern und daher injektiv.\\
		Fasse nun $K$ als Teilkörper von $L$ auf und definiere
		\[a:=\Pi(X)=X+(f)\]
		Dann gilt mit $f=\sum_{i=0}^{n}a_iX^i$, dass
		\begin{align*}
		f(a)=\sum_{i=0}^{n}a_ia^i&=\sum_{i=0}^{n}a_i\Pi(X^i)=\Pi\left(\sum_{i=0}^{n}a_iX^i\right)\\
		&=\Pi(f)\equiv 0
		\end{align*}
		D.h. $a$ ist Nullstelle von $f$.\\
		Sei nun $f$ nicht irreduzibel ist $a$ Nullstelle eines irreduziblen Faktors.\\
		Aus der Polynomdivision folgt, dass $[L:K]=\deg(f)<\infty$ und somit ist $L/K$ algebraisch.
	\end{proof}

	%VL 06.12.2017
	\begin{definition}
		Ein Körper $K$ heißt \textbf{algebraisch abgeschlossen} wenn jedes Polynom $f\in K[X]\setminus K$ eine Nullstelle in $K$ hat.\\
		(Äquivalent: $f$ zerfällt in Linearfaktoren)
	\end{definition}

	\begin{satz}
		Ein Körper $K$ ist genau dann algebraisch abgeschlossen wenn es keine echte algebraische Erweiterung $L/K$ zulässt.
	\end{satz}

	\begin{theorem}
		Sei $K$ ein Körper. Dann gibt es einen algebraische abgeschlossen Körper $L$ mit $K\subseteq L$.
	\end{theorem}
	\begin{proof}[Artin]
		Sei $K\hookrightarrow L_i$ eine Einbettung, sodass jedes nicht Konstante Polynome in $K[X]$ eine Nullstelle in $L_i$ hat. Sei$I=K[X]\setminus K$.\\
		Wir betrachten den Polynomring
		\[K[(X_i)_{i\in I}]\]
		Sei
		\[A=\left\{f(X_f)\mid f\in I\right\}\]\\
		
		Dann ist $A\neq K[(X_i)_{i\in I}]$, denn:\\
		Angenommen $A=K[(X_i)_{i\in I}]$, dann ist $1\in A$, d.h.
		\[1=\sum_{j=1}^{n}g_jf_j(X_{f_j})\]
		für geeignete $g_j\in K[(X_i)_{i\in I}]$ und $f_i\in I$.\\
		Es gibt aber einen Erweiterungskörper $K'$ von $K$, sodass jedes $f_j$ eine Nullstelle $a_j\in K'$ hat.\\
		Definiere 
		\[K([(X_i)_{i\in I}\to K[(X_i)_{i\in I}]\]
		mit $\varphi|_K=\id$.\\
		Dann ist $\varphi(X_i)=X_i$ für $i\in I\setminus \{f_1,...,f_n\}$ und $f(X_{f_i})=a_i$ für $i\in\{1,...,n\}$ und
		\[1=\varphi(1)=\sum_{j=1}^{m}f(g_j)\underbrace{f\big(f(x_{f_j})}_{=0}\big)=0\]
		Widerspruch, da in Körpern $1\neq 0$ sein muss.\\
		Also ist $A\subsetneq K[(X_i)_{i\in I}]$.\\
		\\
		Dann ist $A$ in einem maximalen Ideal $M$ enthalten und es gibt $\pi$
		\[K\hookrightarrow K[(X_i)_{i\in I}]\xrightarrow{\pi}\underbrace{K[(X_i)_{i\in I}]/M}_{=L_i}\]
		Setze $K=L_0$. Dann ist
		\[L_0\xhookrightarrow{\varphi_{01}}L_1\]
		Sei $f\in I$. Dann ist
		\[\underbrace{\varphi_{01}}_{\in L_1[X]}\big(\pi(X_f)\big)=\pi\big(f(X_f)\big)=0\]
		d.h. $\varphi_{01}(f)$ hat eine Nullstelle in $L_1$. Durch Fortführung dieser Konstruktion erhalten wir eine Sequenz
		\[L_0\xhookrightarrow[]{\varphi_{01}}L_1\xhookrightarrow[]{\varphi_{12}}...\]
		und die Abbildungen
		\[L_i\xhookrightarrow[]{\varphi_{ij}}\]
		Sei nun
		\[L=\projlim{ }L_i=\operatornamewithlimits{\dot{\bigcup}}_{i=1}^\infty L_i/\sim\]
		der Direkten $L_i$ und sein die Abbildungen
		\[\varphi_i:L_i\to L\]
		die entsprechenden Einbettungen.\\
		Dann ist $L$ ein Ring und die $\varphi_i$ Ringhomomorphismen.\\
		\\
		Seien $a,b\in L$. Dann existieren $a_i,b_i\in L_i$, mit $a\varphi_i(a_i)$, $b=\varphi_i(b_i)$ und
		\begin{align*}
		a+b&=\varphi_i(a_i+b_i)\\
		ab&=\varphi_i(a_ib_i)
		\end{align*}
		Somit ist $L$ Körper.\\
		\\
		Sei $g\in L[X]\setminus L$. Dann gibt es ein $i$ und ein $g_i\in L[X]\setminus L_i$, sodass
		\[g=\varphi_i(g_i)\]
		Dei Abbildung $\varphi_{ii+1}(g_i)$ zerfällt über $L_{i+1}$ in Linearfaktoren. Somit zerfällt auch
		\[g=\varphi_i(g_i)=\varphi_{j}\big(f_{ij}(g_i)\big)\]
 	\end{proof}
 
 	\begin{satz}
 		Sei $K$ ein Körper, dann gibt es einen algebraisch abgeschlossenen Körper $\overline{K}$, der $K$ enthält und algebraisch über $K$ ist. $\overline{K}$ wird als der algebraische Abschluss von $K$ bezeichnet.
 	\end{satz}
 	\begin{proof}
 		Es gibt einen algebraisch abgeschlossenen Körper $L$ der $K$ enthält. Setze\[\overline{K}=\{a\in L\mid \text{$a$ algebraisch über $K$}\}\]
 		Dann ist $\overline{K}$ ein Teilkörper von $L$ der $K$ enthält.\\
 		\\
 		Zz: $\overline{K}$ ist algebraisch abgeschlossen:\\
 		Sei $f\in\overline{K}[X]\setminus\overline{K}$. Dann hat $f$ eine Nullstelle $\al$ in $L$. $\al$ ist algebraisch über $\overline{K}$. Da $\overline{K}$ algebraisch über $K$ ist ist $\al$ auch algebraisch über $K$. Damit ist $\al\in\overline{K}$.\\
 	\end{proof}
 
 	\begin{kor}
 		Seien $L,L'$ algebraische Abschlüsse des Körpers $K$, dann ist $L\isom L'$.
 	\end{kor}
 	\begin{proof}
 		Ist $\sigma:K\to L$ ein Homomorphismus von Körpern, so induziert $\sigma$ einen Homomorphismus $K[X]\to L[X]$.\\
 		Ist $\al$ eine Nullstelle von $f\in K[X]$ in $K$, so ist $\sigma(\al)$ eine Nullstelle von $\sigma(f)$ in $L$.
 		%TODO Beweis neu machen
 	\end{proof}
 
 	\begin{satz}
 		Sei $K$ ein Körper und $K'=K(\al)$ eine einfache algebraische Körpererweiterung von $K$ und $\sigma:K\to L$ ein Homomorphismus. Dann gilt
 		\begin{enumerate}
 			\item Ist $\sigma':K'\to L$ ein Homomorphismus der $\sigma$ fortsetzt, so ist $\sigma'(\al)$ Nullstelle von
 			\[\sigma'(m_{\al,K})=\sigma(m_{\al,K})\]
 		\end{enumerate}
 	\end{satz}
 
 %VL 11.12.2017
 	\begin{satz}\label{satz:nsterhaltung}
 		Sei $K$ ei Körper $K'=K(\al)$ eine einfache algebraische Erweiterung von $K$ und $\sigma:K\to L$ ein Körperhomomorphismus.
 		\begin{enumerate}
 			\item Ist $\sigma':K'\to L'$ ein Homomorphismus, der $\sigma$ Fortsetzt, so ist $\sigma'(\al)$ Nullstelle von $\sigma(m_{\al,K})=\sigma'(m_{\al,K})$.
 			\item Es gibt zu jeder Nullstelle $\beta\in L$ von $\sigma(m_{\al,K})$ genau eine Fortsetzung $\sigma':K'\to L'$ von $\sigma$ mit $\sigma'(\al)=\beta$.
 		\end{enumerate}
 	\end{satz}
 	\begin{proof}
 		Wir zeige nur b)
 		\begin{enumerate}
 			\setcounter{enumi}{1}
 			\item Sei $\beta\in L$ Nullstelle von $\sigma(m_{\al,K})$ und sei
 			\begin{align*}
 			\phi:K[X]&\to K[\al]&\psi:K[X]\to L\\
 			g&\mapsto g(\al)&g&\mapsto \sigma(g)(\beta)
 			\end{align*}
 			Dann ist $(m_{\al,K})=\Kern(\varphi)$ und $(m_{\al,K})\subset\Kern(\psi)$.\\
 			Wir erhalten das kommutierende Diagramm \ref{fig:cdiag-alg-einf-KE}
 			\begin{figure}
 				\centering
 				\begin{tikzcd}
 					&K[X] \ar[ld,"\varphi"] \ar[d,"\pi"] \ar[rd,"\psi"]&\\
 					K[\al] & K[X]/(m_{\al,K})\ar[l,"\ol{\varphi}"]\ar[r,"\ol{\psi}"] & L
 				\end{tikzcd}
 			\caption{Kommutierendes Diagramm der Algebraischen Körpererweitungen}
 			\label{fig:cdiag-alg-einf-KE}
 			\end{figure}
 			invertierbar. Definiere $\sigma':=\ol{\psi}\circ\ol{\varphi}^{-1}$.\\
 			Dann ist $\sigma:K[\al]\to L$ und \[\sigma'(\al)=\ol{\psi}\big(X+(m_{\al,K})\big)=\psi(X)=\beta\]
 			Das beweist die Existenz von $\sigma'$. Die Eindeutigkeit folgt draus, dass jedes Fortsetzung $\sigma'$ durch ihren Wert auf $\al$ festgelegt ist.
 		\end{enumerate}
 	\end{proof}
 
	 \begin{theorem}[Fortsetzungssatz]\label{satz:Fortsetzung}
	 	Sei $K$ ein Körper, $L$ ein algebraisch abgeschlossener Körper und $\sigma:K\to L$ ein Körperhomomorphismus. Sei $K'/K$ eine algebraische Körpererweiterung.\\
	 	Dann lässt sich $\sigma$ fortsetzen zu einem Homomorphismus $\sigma':K'\to L$.\\
	 	Ist $K'$ zusätzlich abgeschlossen und $L$ algebraisch über $\sigma(K)$, so ist jedes Fortsetzung $\sigma'$ von $\sigma$ ein Isomorphismus.
	 \end{theorem}
 	\begin{proof}
 		Sei $M$ die Menge der Paare $(F,\tau)$, wobei $K\subset F\subset K'$ ein Zwischenkörper und $\tau:F\to L$ eine Fortsetzung von $\sigma$ ist. Dann ist $M$ partiell geordnet durch
 		\[(F_1,\tau_1)\leq (F_2\tau_2)\Leftrightarrow F_1\subset F_2\text{ und }\tau_2|F_1=\tau_1\]
 		Es gilt $M\neq\emptyset$, weil $(K,\sigma)\in M$.\\
 		Jede Kette in $M$ hat eine obere Schranke. somit hat $M$ ein maximales Element $(F,\tau)$. \\
 		Es gilt $F=K'$, denn:\\
 		Angenommen $F\neq K'$. Sei $\al\in K'\setminus F$. Dann lässt sich $\tau$ fortsetzen zu einem Homomorphismus $\tau:F(\al)\to L$. Widerspruch!\\
 		\\
 		Sei nun $K'$ algebraisch abgeschlossen, $L$ algebraisch über $\sigma(K)$ und $\sigma':K'\to L$ eine Fortsetzung von $\sigma$.\\
 		$L$ ist algebraisch über $\sigma(K)$ und damit über  $\sigma'(K')$.\\
 		$\sigma'(K')$ ist aber bereits algebraisch abgeschlossen.\\
 		Es folgt $\sigma'(K')=L$.
 	\end{proof}
 
 	\begin{kor}
 		Sei $K$ ein Körper und seien $\ol K_1$ und $\ol K_2$ algebraische Abschlüsse von $K$. Dann gibt es einen Isomorphismus $\sigma:\ol K_1\to \ol K_2$ der die Identität auf $K$ fortsetzt.
 	\end{kor}
 	\begin{proof}
 		Die Einbettung $\sigma:K\hookrightarrow \ol K_2$ lässt sich fortsetzen zu einem Homomorphismus $\sigma:\ol K_1\to\ol K_2$ . Diese ist ein Isomorphismus.\\
 		(Dieser ist i.a. nicht Kanonisch)
 	\end{proof}
 
 	\begin{exm}
 		Der algebraische Abschluss $\ol{\Q}=\{a\in\C\mid\text{$a$ ist algebraisch über $\Q$}\}$ von $\Q$ in $\C$ ist ein algebraischer Abschluss von $\Q$.
 	\end{exm}


 

\subsection{Zerfallskörper}
	
	\begin{definition}
		Seien $K/L$ und $L'/K$ Körpererweiterungenund sien $\sigma:L\to L'$ ein Homomorphismus.\\
		$\sigma$ wird als \textbf{$K$-Homomorphismus} ($\sigma|_K=\id|_K$) bezeichnet, wenn $\sigma$ eine Fortsetzung der Identität auf $K$ ist.
	\end{definition}

	\begin{definition}
		Sei $L/K$ eine Körpererweiterung und $F\subset K[X]\setminus K$eine Menge nicht-konstanter Polynome.\\
		Eine Erweiterung $L/K$ heißt \textbf{Zerfällungskörper} von $F$, über $K$, wenn
		\begin{enumerate}
			\item Jedes $f\in K$ zerfällt in Linearfaktoren über $L$
			\item Die Körpererweiterung $L/K$ wird con Nullstellen der $f\in F$ erzeugt.
		\end{enumerate}
	\end{definition}

	\begin{lem}
		Sei $\ol K$ ein algebraischer Abschluss von $K$ und $M$ die Menge der Nullstellen der Polynome von $F$ in $\ol K$. Dann ist $L=K(M)\subset \ol K$ ein Zerfällungskörper von $F$.
	\end{lem}

	\begin{satz}\label{satz:abbZFK}
		Sei $F\subset K[X]\setminus K$ und seine $L_1$ und $L_2$ zwei Zerfällungskörper von $F$ über $K$.
		Sei $\sigma:L_1\to\ol L_2$ ein $K$-Homomorphismus in einen algebraischen Abschluss von $L_2$.\\
		Dann gilt $\sigma(L_1)=L_2$.
	\end{satz}
	\begin{proof}
		Wir beweisen schrittweise:
		\begin{itemize}
			\item Wir nehmen zuerst an, dass $F$ nur eine Polynom $f$ enthält.\\
			Seien $a_1,...,a_n$ die Nulsstelle von $f$ in $L_1$ und $b_1,...,b_n$ die Nullstelle von $f$ in $L_2$. Dann ist
			\[f=\cal\prod_{i=1}^{n}(X-a_i)\]
			mit $c\in K$ und
			\[\sigma(f)=c\prod_{i=1}^{n}\big(X-\sigma(a_i)\big)=c\prod_{i=1}^n(X-b_i)\]
			d.h. nach Umnummerierung also $\sigma(a_i)=b_i$.\\
			Es folgt
			\[L_2=K(b_1,...,b_n)=K\big(\sigma(a_1),...,\sigma(a_n)\big)=\sigma\big(K(a_1,...,a_n)\big)=\sigma(L_1)\]
			\item Falls $f$ endlich viele Polynome enthält, so argumentiert man anlog mit dem Produkt der Polynome.
			\item Sei $F$ nun unendlich, $M_1$ die Menge der Nullstelle von $F$ in $L_1$, $M_2$ die Menge der Nullstellen von $F$ in $L_2$ und sei $a\in L_1$.\\
			Dann gibt es eine endliche Teilmenge $M_1'\subset M_1$, sodass $a\in K(M_1')$, d.h. es gibt eine endliche Teilmenge $F'\subset F$, sodass $a$ im Zerfällungskörper $L_1'$ von $F'$ über $K$ in $L_1$ liegt.\\
			Dann gilt $\sigma(L_1')=L_2'$, d.h. $\sigma(a)\in L_2$ und $\sigma(L_1)\subset L_2$.\\
			Analog gilt $L_2\subset\sigma(L_1)$.
		\end{itemize}
	\end{proof}

	\begin{kor}
		Sei $F\in K[X]\setminus K$ und seien $L_1$ und $L_2$ Zerfällungskörper von $F$ über $K$.\\
		Dann gibt es einen $K$-Isomorphismus $L_1\to L_2$
	\end{kor}
	\begin{proof}
		Die Inklusion $K\hookrightarrow \ol L_2$ lässt sich zu einer $K$-Homomorphismus $L_1\xrightarrow{\sigma} \ol{L_2}$ fortsetzen. Für diesen gilt $\sigma(L_1)=L_2$
	\end{proof}

	\begin{theorem}\label{thm:zfk}
		Sei $L/K$ eine algebraische Körpererweiterung. Dann sind äquivalent:\begin{enumerate}
			\item $L$ ist der Zerfällungskörper einer Menge nicht-konstanter Polynome in $K[X]$.
			\item Ist $\sigma:L\to\ol L$ ein $K$-Homomorphismus, so gilt $\sigma(L)=L$.
			\item Jedes irreduzible Polynom $f\in K[X]$, das mindestens eine Nullstelle hat zerfällt in $L$ vollständig in Linearfaktoren.
		\end{enumerate}
	\end{theorem}
	\begin{proof}
		\begin{description}
			\item[$1)\Rightarrow 2)$] Folgt aus \ref{satz:abbZFK} mit $L_1=L_2=L$
			\item[$2)\Rightarrow 3)$] Sei $f\in K[X]$ irreduzibel und $a\in L$ eine Nullstelle von $f$ in $L$. Dann ist $f$ bis auf eine Konstante das Minimalpolynom $m_{\al,K}$. Ist $b$ eine weitere Nullstelle von $f$ in $\ol L$, so hat die Einbettung $K\hookrightarrow\ol L$ eine Fortsetzung $\sigma:K(\al)\to\ol L$ mit $\sigma(a)=b$ (\ref{satz:nsterhaltung}). Diese lässt sich Fortsetzen (\ref{satz:Fortsetzung}) zu einm $K$-Homomorphismus $\sigma:L\to \ol L$.\\
			Aus $\sigma(L)=L$ folgt $b\in L$.
			\item[$3)\Rightarrow1)$] Es ist $L=K(M)$ für eine Teilmenge $M\subset L$ bestehend aus algebraischen Elementen (über $K$).\\
			Für $a\in M$ ist $m_{a,K}$ irreduzibel über $K$ und hat $a$ als Nullstelle in $L$. Somit zerfällt $m_{a,K}$ in Linearfaktoren über $L$.\\
			Also ist $L$ der Zerfällungskörper der $m_{a,K}$.
		\end{description}
	\end{proof}

	\begin{definition}
		Eine algebraische Körpererweiterung $L/K$ die eine der Bedingungen von \ref{thm:zfk} erfüllt heißt \textbf{normal}.
	\end{definition}

	\begin{satz}
		Sei $L/K$ eine normale Körpererweiterung und $K\subset M\subset L$ ein Zwischenkörper. Dann ist auch $L/M$ normal.
	\end{satz}
	\begin{proof}
		Sei $\sigma\in\Hom_M(L,\ol L)$, dann ist $\sigma\in \Hom_K(L,\ol L)$. Dann ist $\sigma(L)=L$.
	\end{proof}

	\begin{exm}
		\begin{enumerate}
			\item Sei $L/K$ eine Körpererweiterung von Grad 2, dann ist $L/K$ normal.
			\item Die Erweiterungen $\Q(\sqrt[4]{2})/(\Q(\sqrt{2}))$ und $\Q(\sqrt{2})/\Q$ sind normal.\\
			Die Erweiterung $\Q(\sqrt[4]{2})/\Q$ hingegen nicht.
		\end{enumerate}
	\end{exm}


\subsection{Separabel Körpererweiterungen}
In diesem Abschnitt bezeichne $K$ ein Körper.

	\begin{definition}
		Ein Polynom $f\in K[X]$ heißt \textbf{separabel}, wenn $f$ nur einfache Nullstellen in einem algebraischen Abschluss $\ol K$ von $K$ hat.\\
		(Dies ist unabhängig von der Wahl von $\ol K$)
	\end{definition}

	\begin{satz}
		Sei $f\in K[X]$ irreduzible, dann
		\[\text{$f$ separabel}\Leftrightarrow f'\neq 0\]
	\end{satz}
	\begin{proof}
		Sei $\al\in K$ eine Nullstelle von $f$. Dann ist $f=cm_{\al,K}$ für ein $c\in K^*$ und  es gilt
		\[\text{$\al$ ist mehrfache Nullstelle}\Leftrightarrow f(\al)=f'(\al)=\Leftrightarrow \text{$f'=0$ weil $\deg(f')<\deg(f)$}\]
	\end{proof}

	\begin{definition}
		Sei $L/K$ eine algebraische Körpererweiterung. $a\in L$ heißt \textbf{separabel} über $K$, wenn $m_{a,K}$ separabel ist.
	\end{definition}

	\begin{definition}
		Sei $L/K$ eine algebraische Körpererweiterung. $L$ heißt \textbf{separabel} über $K$, wenn jedes $a\in L$ separabel über $K$ ist
	\end{definition}

%VL 13.12.2017

	\begin{satz}
		Sei $\cha(K)=0$ und $L/K$ eine algebraische Körpererweiterung. Dann ist $L/K$ separabel.
	\end{satz}

	\begin{definition}
		Sei $L/K$ eine algebraische Körpererweiterung und $\ol K$ der algebraische Abschluss von $K$.\\
		Der \textbf{Separabilitätsgrad} $[L:K]_S$ von $L$ über $K$ ist definiert als
		\[[L:K]_S:=\abs{\Hom_K(L,\ol K)}\]
		Diese Definition ist unabhängig von $\ol K$.
	\end{definition}

	\begin{satz}
		Sei $K(a)/K$ eien einfach algebraische Körpererweiterung. Dann gilt
		\begin{enumerate}
			\item Der Separabilitätsgrad $[K(a):K]_S$ ist gleich der Anzahl der verschiedenen Nullstellen von $M_{a,K}$ in einem algebraischen Abschluss $\ol K$ von $K$.
			\item $a$ist genau dann separabel über $K$, wenn $[K(a):K]_S=[K(a),K]$.
		\end{enumerate}
	\end{satz}
	\begin{proof}
		\begin{enumerate}
			\item \ref{satz:Fortsetzung} gibt, dass die Anzahl der verschiedene $K$-Homomorphismen $\varphi:K(a)\to\ol K$ gelilch der Anzahl der verschiedenen Nullstellen von $m_{a,K}$ in $\ol K$ ist.
			\item Es gilt
			\begin{align*}
			&\text{$a$ ist separabel über $K$ }\\
			&\Leftrightarrow \text{ $m_{a,K}$ ist separabel }\\
			&\Leftrightarrow\text{ $m_{a,K}$ hat nur einfache Nullstellen in $\ol K$ }\\
			&\Leftrightarrow\text{ die Anzahl der Nullstellen von $m_{a,K}$ ist $\deg(m_{a,K})$}\\
			&\Leftrightarrow [K(a):K]_S=[K(a):K]
			\end{align*}
		\end{enumerate}
	\end{proof}

	\begin{theorem}[Gradsatz der Separabilität]
		Sei $K\subset L\subset M$ algebraische Körpererweiterungen. Dann gilt\[[M:K]_S=[M:L]_S[L:K]_S\]
	\end{theorem}
	\begin{proof}
		Sei $\ol K$ eine algebraischer Abschluss von $M$. Dann ist $\ol K$ auch ein algebraischer Abschluss von $K$ und $K\subset L\subset M\subset\ol K$. Sei
		\begin{align*}
		\Hom_K(L,\ol K)&=\{\sigma_i\mid i\in I\}\\
		\Hom_L(M,\ol K)&=\{\tau_i\mid i\in J\}\\
		\end{align*}
		mit paarweise verschiedenen $\sigma_i$ und $\tau_i$.\\
		Wir können $\sigma_i:L\to\ol K$ zu einerm $K$-Automorphismus $\ol{\sigma_i}:\ol K\to\ol K$ fortsetzen. Es gilt
		\begin{enumerate}
			\item Die Abbildung $\ol{\sigma_i}\circ\ol{\tau_j}$ sind paarweise verschiedene, denn:\\
			Sei $\ol{\sigma_i}\circ \tau_j=\ol{\sigma}_{i'}\circ\tau_{j'}$. Die Restriktionen beider Seiten auf $L$ liefert $\sigma_i=\sigma_{i'}$, d.h. $i=i'$. Es folgt $\tau_j=\tau_{j'}$ und $j=j'$.
			\item $\Hom_K(M,\ol K)=\{\ol{\sigma}\circ \tau_j\mid i\in I,j\in J\}$, denn:\\
			Die Abbildungen $\ol{\sigma}_i\circ \tau_i$ sind $K$-Homomorphismen. Es beleibt zu zeigen, dass jedes Element in $\Hom_K(M,\ol K)$ dieser Form ist.\\
			Sei $\tau\in\Hom_K(M,\ol K)$. Dann ist $\tau|_L=\sigma_i$ für ein $i$. Die Abbildung $\ol{\sigma}_i^{-1}\circ \tau$ ist in $\Hom_L(M,\ol K)$. d.h.$\ol{\sigma_i}^{-1}\circ\tau=\tau_j$ für ein $j\in J$. Also ist $\tau=\ol{\sigma_i}\circ\tau_j$.
		\end{enumerate}
	Es folgt die Behauptung.
	\end{proof}

	\begin{satz}\label{satz:abschSepGrad}
		Sei $L/K$ eine endliche Körpererweiterung. Dann gilt\[[L:K]_S\leq [L:K]\]
	\end{satz}
	\begin{proof}
		$L/K$ ist algebraisch, d.h. $L=K(a_1,...,a_n)$ für geeigente $a_1,...,a_n\in L$.\\
		Sei $L_0=K$, $L_1=K(a_1)$,..., $L_n=K(a_1,...,a_n)$.\\
		Äquivalent dazu ist $L_i=L_{i-1}(a_i)$ für $1\leq i\leq n$.\\
		Dann gilt 
		\begin{align*}
		[L_{i}:L_{i-1}]&=[L_{i-1}(a_i):L_{i-1}]=\deg(m_{a_i,L_{i-1}})\\
		&\geq\text{ Anzahl der verschidenen Nullstellen von $m_{a_iL_{i-1}}$ in $\ol K$}
		&=[L_i:L_{i-1}]_S
		\end{align*}
		
		Da aber zusätzlich
		\begin{align*}
		[L:K]&=\prod_{i=1}^{n}[L_i:L_{i-1}]\\
		[L:K]_S&=\prod_{i=1}^{n}[L_i:L_{i-1}]_S
		\end{align*}
		folgt $[L:K]_S\leq[L:K]$
	\end{proof}

	\begin{theorem}
		Sei $L/K$ eine endliche Körpererweiterung. Dann sin äquivalent
		\begin{enumerate}
			\item $L/K$ ist separabel.
			\item Es gibt über $K$ separabele Elemente $a_1,...,an\in L$, sodass $L=K(a_1,...a_n)$.
			\item $[L:K]_S=[L:K]$
		\end{enumerate}
	\end{theorem}
	\begin{proof}
		\begin{description}
			\item[$1)\Rightarrow 2)$] ist klar.
			\item[$2)\Rightarrow 3)$] Setze $L_0=K$, $L_I=L_{i-1}(a_i)$.\\
			Dann ist $a_i$ separable über $K$, d.h. $m_{a_i,K}$ hat nur einfache Nullstellen in $\ol K$. Es gilt $m_{a_i,L_{i-1}}|m_{a_i,K}$.\\
			Also hat auch $m_{a_i,L_{i-1}}$ nur einfache Nullstellen in $\ol K$.\\
			Somit ist $a_i$ separabel über $L_{i-1}$ und daher gilt $[L_i:L_{i-1}]_S=[L_1:L_{i-1}]$.\\
			Es folgt
			\[[L:K]_S=\prod_{i=1}^{n}[L_i:L_{i-1}]_S=\prod_{i=1}^{n}[L_i:L_{i-1}]=[L:K]\]
			\item[$3)\Rightarrow 1)$] Sei $a\in L$. Dann ist $a$ algebraisch über $K$ und $K\subset K(a)\subset L$.\\
			Dann gilt mit dem Gradsatz
			\begin{align*}
			[L:K]_S&=[L:K(a)]_S[K(a):K]_S\\
			[L:K]&=[L:K(a)][K(a):K]
			\end{align*}
			Mit der Annahme dass $[L:K]=[L:K]_S$ und \ref{satz:abschSepGrad}
			\begin{align*}
			[L:K(a)]_S&\leq [L:K(a)]\\
			[K(a):K]_S&\leq [K(a):K]
			\end{align*}
			folgt, dass
			\[[K(a):K]_S\leq [K(a):K]\]
			d.h. $a$ ist separabel über $K$.
		\end{description}
	\end{proof}

	\begin{satz}
		Sei $f\in K[X]\setminus K$ separabel. Dann ist auch der Zerfällungskörper von $f$ über $K$ separabel.
	\end{satz}
	\begin{proof}
		Seien $a_1,...,a_n$ die Nullstellen von $f$ in $\ol K$. Dann ist $K(a_1,...,a_n)$ ein Zerfällungskörper von $f$ über $K$. Aus $f(a_i)=0$ folgt, dass $m_{a_i,K}|f$. Somit hat $m_{a_i,K}$ nur einfache Nullstellen in $\ol K$ d.h. $a_i$ ist separabel über $K$.
	\end{proof}

	\begin{kor}
		Sei $L/K$ eine algebraische Körpererweiterung und $M\subset L$, sodass $L=K(M)$. Dann sind äquivalent
		\begin{enumerate}
			\item $L/K$ ist separabel
			\item Alle $a\in M$ sind separabel über $K$.
		\end{enumerate}
		Ist ein dieser Bedingungen erfüllt, so gilt
		\[[L:K]_S=[L:K]\]
	\end{kor}
	\begin{proof}
		\begin{description}
			\item[$1)\Rightarrow 2)$] klar.
			\item[$2)\Rightarrow 1)$] Sei $c\in L$. Dann gibt es immer endlich viele $a_1,...,a_n\in M$, sodass $c\in K(a_1,...,a_n)$. Nach \ref{satz:endlicheVersion} ist $K(a_1,...,a_n)$ separabel über $K$ und somit auch $c$.
		\end{description}
		Für $L/K$ endlich gilt \ref{satz:endlicheVersion}.\\
		Sei also $[L:K]=\infty$. Da $L/K$ separabel ist, gilt dies auch für jeden Zwischenkörper $K\subset E\subset L$. Falls $[E:K]<\infty$, so gilt
		\[[L:K]_S=[L:E]_S[E:K]_S\geq[E:K]_S=[E:K]\]
		Es folgt $[L:K]_S=\infty$, weil $L/K$ Zwischenkörper beliebig hohen aber endlichen Grad hat. 
	\end{proof}

	\begin{kor}
		Seien $K\subset L\subset M$ algebraische Körpererweiterungen. Dann gilt $M/K$ ist genau dann separabel, wenn $M/L$ und $L/K$ separabel sind.
	\end{kor}
	\begin{proof}
		\begin{description}
			\item[$\Rightarrow$] Sei $M/K$ separabel. Dann ist auch $L/K$ separabel. Sei $a\in M$. Dann gilt $m_{a,L}|m_{a,K}$, d.h. $a$ ist separabel über $L$.
			\item[$\Leftarrow$] Seien $M/L$ und $L/K$ separabel. Sei $a\in M$. Der Erweiterungskörper $L'$ von $K$ der von den Koeffizienten von $m_{a,L}$ erzeugt wird ist endlich über $K$.\\
			Aus $L'\subset L$ folgt $m_{a,L}|m_{a,L'}$. Da $m_{a,L}\in L'[X]$ gilt aber auch $m_{a,L'}|m_{a,L}$. Also ist $m_{a,L}=m_{a,L'}$ und $L'(a)/L'$ ist separabel.
		\end{description}
	\end{proof}

	\begin{theorem}[Satz vo primitiven Element]\label{satz:primElem}
		Sei $L/K$ einen endliche separable Körpererweiterung. Dann gibt es ein $a\in L$, sodass $L=K(a)$
	\end{theorem}
	\begin{proof}
		\begin{description}
			\item[$K$ endlich] Sei $K$ endlich, so auch $L$. Sei $a$ ein Erzeuger von $L^*$, Dann ist $L=K(a)$.
			\item[$K$ unendlich] Sei $K$ unendlich. Da $L/K$ einen endliche Erweiterung ist gibt es Elemente $a_1,...,a_n\in L$, sodass $L=K(a_1,...,a_n)$. Durch zusammenfassen $a_ia_j$ zu $c$ reicht es für $n=2$ zu zeigen:\\
			Sei $L=K(a,b)$ für geeignete $a,b\in L$ gegeben. Sei $m=[L:K]_S$ und seien $\sigma_1,...,\sigma_m$ die verschiedenen Elemente in $\Hom_K(L,\ol K)$. Definiere
			\[g=\prod_{i\neq j}\left(\big(\sigma_i(a)-\sigma_j(a)\big)+\big(\sigma_i(b)-\sigma_j(b)\big)\right)\in \ol K[X]\]
			Dann ist $g$ nicht das Nullpolynom, denn für $i\neq j$ ist $\sigma_i(a)\neq \sigma_j(a)$ oder $\sigma_i(b)\neq \sigma_j(b)$. Da $K$ unendlich gibt es ein $c\in K$, sodass mit $g(c)\neq 0$. Es folgt
			\[\left(\big(\sigma_i(a)-\sigma_j(a)\big)+\big(\sigma_i(b)-\sigma_j(b)\big)\right)c\neq 0\]
			bzw. $\sigma_{i}(a+bc)\neq \sigma_j(a+bc)$ für alle $i\neq j$.\\
			Die Elemente $\sigma(a+bc)$ si also paarweise verschieden. Sei $f$  das Minimalpolynom von $a+bc$ über $K$. Es folgt
			\[[L:K]_S=m\leq \deg(f)=[K(a+bc):K]\leq [L:K]\]
			Da $L/K$ separabel ist folgt Gleichheit.
		\end{description}
	
	\end{proof}



%VL 18.12.2017

	\subsection{Endliche Körper}
	\begin{definition}
		Sei $p$ eine positiv Primzahl. Dann ist $\F_p=\Z/p\Z$ ein Körper mit $p$ Elementen und $\cha(\F_p)=p$.
	\end{definition}

	\begin{satz}\label{satz:primErweitEigenschaften}
		%TODO unterscheidung \F_p und F_p
		Sei $F$ ein endlicher Körper, dann ist $\cha(F)=p>0$ und $F$ enthält $q=p^n$ Elemente, wobei $n=[F:\F_p]$.\\
		$F$ ist der Zerfällungskörper des Polynoms $X^q-X$ über $\F_p$. Die Erweiterung$F/\F_p$ ist normal.
	\end{satz}
	\begin{proof}
		Da $F$ endlich ist hat $F$ einen endlichen Primkörper $\F_p$ und $\cha(F)=p$.\\
		$F$ ist eine endlich-dimensionaler Vektorraum über $\F_p$, d.h. $F=\F_p^n$ mit $n=[F:\F_p]$ und $\abs{F}?p^n=q$.\\
		Die multiplikative Gruppe $F^*$ hat $q-1$ Elemente, d.g. $a^{q-1}=1$ für alle $a\in F^*$.\\
		Jedes $a\in F$ ist also Nullstelle von $f=X(X^{q-1}-1)=X^q-X$.\\
		$F$ ist also der Zerfällungskörper von $f=X^q-X$ über $\F_p$.
	\end{proof}

	\begin{theorem}
		Sei $p$ eine positive Primzahl. Dann gibt es zu jedem positiven $n\in \N$ einen Erweiterungskörper $\F_q/\F_p$ mit $q=p^n$ Elementen. $\F_q$ ist bis auf Isomorphie eindeutig charakterisiert, als der Zerfällungskörper von $X^q-X$ über $\F_p$ und besteht aus den $q$ Nullstellen. dieses Polynoms. Jeder endliche Körper ist isomorph zu genau einem Körper des Typs $\F_q$.
	\end{theorem}
	\begin{proof}
		Sei $f=X^q-X\in \F_p[X]$ und $L\subset\ol{\F_p}$ der Zerfällungskörper von $f$ über $\F_p$.\\
		Da $f'=-1$hat $f$ nur einfache Nullstellen in $\ol{\F}_p$.\\
		Seien $a,b\in\ol{\F_p}$ zwei Nullfolge von $f$. Dann gilt
		\begin{align*}
		(a+b)^q&=\sum_{j=0}^{q}\binom{q}{j}a^{q-j}b^j\\
		&=a^q+\underbrace{\binom{q}{1}}_{=0}a^{q-1}b+...+b^q\\
		=q^q+b^q\\
		&=a+b
		\end{align*}
		Da heißt $a-b$ ist Nullstelle von $f$ in $\ol{\F_p}$. Für $b\neq 0$ ist
		\begin{align*}
		(ab^{-1})^{q}&=a^q(b^{-1})^q\\
		&=a^{q}(b^q)^{-1}\\
		&=ab^{-1}
		\end{align*}
		D.h. $ab^{-1}$ ist Nullstelle von $f$.\\
		Die Nullstellen von $f$ in $\ol{\F_p}$ bilden als einen Teilkörper von $\ol{\F_p}$.\\
		Folglich besteht $L$ aus den $q$ Nullstellen von $f$ in $\ol{\F_p}$.\\
		Sei $F$ ein zweiter Körper mit $q$ Elementen, dann ist na\ref{satz:primErweitEigenschaften} $F$ ein Zerfällungskörper von $X^q-X$ über seinem Primkörper $\F_p$. $F$ ist somit isomorph.
	\end{proof}

	\begin{bem}
		Wir können di Körper $\F_q$ auch Konstruieren, indem wir die Nullstellen eines irreduziblen Polynoms zu $\F_p$ adjungiert.
	\end{bem}

	\begin{satz}
		Sei $n\in\N$. Dann gibt es ein irreduzibles Polynom $f$ mit $\deg_{\F_p}(f)=n$.
	\end{satz}
	\begin{proof}
		Sei $q=p^n$. Dann ist $\F_q/\F_p$ eine separable Erweiterung vom Grad $n$.\\
		Nach dem Satz vom Primitven Element \ref{satz:primElem} ist $\F_q=\F_p(a)$ für ein $a\in\F_q$.\\
		Dann ist $m_{a,\F_p}$ irreduzibel und vom Grad $n$.
	\end{proof}

	\begin{exm}
		Das Polynom $X^2+1$ ist irreduzible über $\F_3$.\\
		Also
		\[\F_9=\F_3(\theta)=\{a+\theta b\mid a,b\in\F_3\})\F_3[X]/(X^2+1)\]
		mit $\theta^2=-1$.
	\end{exm}

	\begin{satz}
		Sei $F$ eine endlicher Körper und $K/F$ eine algebraische Erweiterung. Dann ist $K/F$ normal und separabel.
	\end{satz}
	\begin{proof}
		Sei $\F_p$ der Primkörper von $F$ und $\ol K$ ein algebraischer Abschluss von $\F_p$.Dann ist $\ol K$ auch ein algebraischer Abschluss von $F_p$. Schreibe $\ol K=\ol{F_p}$ Dann
		\[F_p\subset F\subset K\subset \ol{F_p}\]
		Falls $\abs K\leq \infty$, so ist $K$ isomorph zu $\F_q$ mit $q=p^n$ und $K$ ist als Zerfällungskörper des separablen Polynom  $X^{q}-X$ normal und separabel über $F_p$ und somit über $F$.\\
		Sei $\abs K=\infty$. Wähle $M\subset K$ mit $K=F(M)$.\\
		Dann ist $K$ die Vereinigung von Körper $F(M')$ wobei $M'$ eine endliche Teilmenge von $M$ ist.\\
		$F(M')$ ist eine endliche Erweiterung von $F$ und somit von $F_p$, d.h. $F(M')$ ist isomorph zu $F_q$. Somit ist $K$ normal und separabel über $F$.
	\end{proof}

	\begin{definition}
		Sei $F_q$ mit $q=p^n$ ein endlicher Körper.\\
		Dann ist die Abbildung
		\begin{align*}
		\Fr:F_q&\to F_q\\
		x&\mapsto x^p
		\end{align*}
		ein $F_p$-Automorphismus von $F_q$. Diese wir als \textbf{Frobenius-Automorphismus} bezeichnet.
	\end{definition}

	\begin{theorem}
		Sei$q=p^n$, dann ist die Gruppe $\Aut_{F_p}(F_q)$ zyklisch mit Ordnung $n$. Und $\Aut_{F_p}(F_q)=\grp{\Fr}$ wird vom Frobenius-Automorphismus erzeugt.
	\end{theorem}
	\begin{proof}
		Sei $s$ die Ordnung von $\Fr$, d.h. $s=\abs{\grp{\Fr}}$.\\
		Für $a\in F_q$ gilt
		\[\Fr^n(a)=a^{p^n}=a^q=a\]
		s.h. $s|n$. Andererseits ist $\Fr^s(a)=a^{p^s}=a$ für alle $a\in F_q$.\\
		\\
		Das Polynom $X^{p^s}-X$ hat höchsten $p^s$ verschiedene Nullstellen, d.h. $p^s\geq q=p^n$. Also gilt $s=n$.\\
		Die Erweiterungen $F_q/F_p$ ist normal und separabel, sodass
		\begin{align*}
		\abs{\Aut_{F_p}(F_q)}&=\abs{\Hom_{F_p}(F_q,\ol F_p)}
		\intertext{da $F_q/F_p$ normal ist.}
		&=[F_q:F_p]_S\\
		&=[F_q:F_p]
		\intertext{Da $F_q/F_p$ separabel ist}
		&=n
		\end{align*}
		d.h. $\Fr$ erzeugt $\Aut_{F_p}(F_q)$.
	\end{proof}
	
	%Einschub 10.01.2018
	%TODO Einschub: Zeta-Funktion 
	%Ende Einschub
	
	\section{Galois-Erweiterungen}
	\begin{definition}
		Eine algebraische, normale, separabele Körpererweiterung $L/K$ heißt \textbf{Galoiserweiterung}.
	\end{definition}
	\begin{definition}
		Man bezeichnet $\Aut_K(L)$ als \textbf{Galoisgruppen} von $L/K$ und schreibt $G(L/K)$ für $\Aut_K(L)$.
	\end{definition}
	
	\begin{exm}
		\item Sei $F$ ein endlicher Körper und $K/F$ eine algebraische Körpererweiterung. Dann ist $K/F$ eine Galois-Erweiterung.
		\item Sei $p$ ein positive Primzahl und $q=p^n$. $\F_q/\F_p$ ist eine Galois-Erweiterung. Die Galoisgruppe ist zyklisch der Ordnung $n$ und wird vom Frobenius-Automorphismus erzeugt.
		\item $\C/\R$ ist eine Galois-Erweiterung. Die Galoisgruppe wird von der komplexen Konjugation erzeugt.
	\end{exm}

%VL 20.12.2017

	\begin{satz}
		Sei $L/K$ eine normale Körpererweiterung und $f\in K[X]$ irreduzible. Dann permutiert $\Aut_K(L)$ die Nullstellen von $f$ transitiv.
	\end{satz}
	\begin{proof}
		Falls $f$ keine Nullstellen in $L$ hat so ist nichts zu zeigen.\\
		Sei $a\in L$ eine Nullstelle von $f$ und $\varphi\in\Aut_K(L)$. Dann gilt
		\[f\big(\varphi(a)\big)=\varphi\big(\underbrace{f(a)}_{=0}\big)=0\]
		d.h. $\varphi(a)$ ist Nullstelle von $f$.\\
		Weiterhin ist $f=cm_{a,K}$ für ein $c\in K$.\\
		Sei nun $b$ eine weitere Nullstelle von $f$ in $L$. Dann ist $b$ auch Nullstelle von $m_{a,K}$ und die Einbettung
		\[K\hookrightarrow \ol L\]
		lässt sich fortsetzen als
		\[K(a)\xhookrightarrow{a} \ol L\]
		mit $\sigma(a)=b$zu einem $K$-Homomorphismus
		\[L\xrightarrow{\sigma}\ol L\]
		Da $L/K$ normal ist gilt $\sigma(L)=L$.\\
		Somit ist $\sigma\in\Aut_K(L)$ mit $\sigma(a)=b$.
	\end{proof}

	\begin{satz}\label{satz:SepGradAut}
		Sei $L/K$ eine normale Körpererweiterung dann gilt\[\abs{\Aut_K(L)}=[L:K]_S=\abs{\Hom_K(L,\ol K)}\]
	\end{satz}
	\begin{proof}
		Sei $\ol L$ ein algebraischer Abschluss von $L$. Dann ist $\ol L$ auch ein algebraischer Abschluss von $K$.\\
		Ist $\varphi:L\to\ol L$ ein $K$-Homomorphismus, so gilt $\varphi(L)=L$. Als ist die Abbildung
		\[\Hom_K(L,\ol K)\to\Aut_K(L)\]
		eine Bijektion.
	\end{proof}
	
	\begin{satz}
		Sei $L/K$ eine endliche Galois-Erweiterung. Dann ist
		\[[L:K]=\abs{G(L/K)}\]
	\end{satz}
	\begin{proof}
		Nach \ref{satz:SepGradAut} gilt mit Separabilität
		\[\abs{\Aut_K(L)}=[L:K]_S=[L:K]\]
	\end{proof}

	\begin{definition}
		Sei $L$ ein Körper und $G$ eine Untergruppe von $\Aut_K(L)$. Dann ist
		\[L^G:=\{x\in L\mid g(x)=x\forall g\in G\}\]
		ein Teilkörper von $L$. Dieser wird als \textbf{Fixkörper} von $G$ bezeichnet.
	\end{definition}

	\begin{satz}
		Sei $L/K$ eine Galois-Erweiterung. Dann sit der Fixkörper von $G(L/K)$ genau $K$.
	\end{satz}
	\begin{proof}
		Sei $G=G(L/K)$. Dann ist $\subset L^G$.\\
		Sei $a\in L/K$. Dann ist $\deg(m_{a,K})\geq 2$. Da $L/K$ normal ist, zerfällt $m_{a,K}$ über $L$ in Linearfaktoren. Weil $L/K$ separabel ist, ist $a$ eine einfache Nullstelle von $m_{a,K}$. Es gibt als ein $b\in L$ mit $b\neq a$ mit $m_{a,K}(b)=0$. Da $G(L/K)$ die Nullstellen von $m_{a,K}$ transitiv permutiert gibt es ein $\varphi\in G(L/K)$ mit $\varphi(a)=b$.
	\end{proof}

	\begin{satz}
		Sei $L$ ein Körper und $H$ eine endliche Untergruppe von $\Aut_K(L)$. Dann ist $L/L^H$ eine endliche Galois-Erweiterung mit Galoisgruppe $H$ und
		\[[L:L^H]=\abs{H}\]
	\end{satz}
	\begin{proof}
		Sei $a\in L$ un $Y_a=\{\varphi(a)\mid\varphi\in H\}\subset L$.\\
		seine $a_1,...,a_n$ die verschiedenen Elemente von $Y_a$. Sei
		\[f_a=\prod_{i=1}^{n}(X-a_i)\]
		Dann ist für $\varphi\in H$ 
		\[\varphi(f_a)=\prod_{i=1}^{n}\big(X-\varphi(a_i)\big)=f_a\]
		Also ist $f_a\in L^H[X]$. Da $a$ Nullstelle des Polynoms $f_a$ ist ist $a$ separabel.\\
		Die Erweiterung $L/L^H$ ist als separabel.\\
		Dann ist $L$ der Zerfällungskörper der Polynome $F=\{f_a\mid a\in L\}$. Somit ist $L/L^H$ eien Galoiserweiterung.\\
		Aus $m_{a,L^H}| f_a$ folgt 
		\[\deg(m_{a,L^H})\leq \deg(f)\leq \abs{H}\tag{$\star$}\]
		Ist $\abs{H}<[L:L^H]\leq\infty$, so gibt es eine endliche Teilmenge $S\subset L$, sodass für $M=L^H(S)$ gilt
		\[\infty>[M:L^H]>\abs H\]
		Zusätzlich ist $M/L^H$ separabel, da $L/L^H$ separabel ist.\\
		Nach Satz \ref{satz:primElem} gibt es ein $c\in L$, sodass $M=L^H(c)$ ist. Dann gilt
		\[\deg(m_{c,L^H})=[M:L^H]>\abs{H}\]
		Widerspruch zu $(\star)$.\\
		Also ist $[L:L^H]\leq\abs H$.\\
		D.h. $L/L^H$ ist einen endliche Galoiserweiterung.\\
		Aus $H\subset \Aut_{L^H}(L)$ folgt
		\[\abs H\leq\abs{\Aut_{L^H}(L)}=[L:L^H]\leq\abs{H}\]
		Somit gilt $H=\Aut_{L^H}(L)$
	\end{proof}

	\begin{bem}
		Für $a\in  L$ ist $m_{a,L^H}=f_a$ in der Notation des Beweises.
	\end{bem}
	
	\begin{theorem}[Hauptsatz der Galoistheorie]\label{satz:hsGal}
		Sei $L/K$ eine endliche Galois-Erweiterung. Sei $U$ die Menge der Untergruppen von $G(L/K)$ und $Z$ die Menge der Zwischenkörper von $L/K$. Dann sind die Abbildungen
		\begin{align*}
		\Phi:Z&\to U&\Psi:U&\to Z\\
		M&\mapsto G(L/M)&H&\mapsto L^H\\
		\end{align*}
		zueienander inverse Bijektionen. Für einen Zwischenkörper $M$ von $L/K$ ist die Erweiterung $M/K$ normal genau dann wenn $G(L/M)$ normal in $G(L/K)$ ist.\\
		In diesem Fall ist
		\begin{align*}
		G(L/K)&\to G(M/K)\\
		\sigma&\mapsto\sigma|_M
		\end{align*}
		eine surjektiver Gruppenhomomorphismus mit $\Kern()=G^0(L/M)$.\\
		Dieser induziert einen Isomorphismus
		\[G(M/K)\isom G(L/K)/G(L/M)\]
	\end{theorem}
	\begin{proof}
		Sei $M$ ein Zwischenkörper von $L/K$. Dann ist $L/M$ eine Galois-Erweiterung und $G(L/M)=\Aut_M(L)$, sowie $c\Aut_K(L)=G(L/K)$, weil $L\subset M$.\\
		Somit ist $\Phi$ wohldefiniert. Sei $M\in Z$, dann ist
		\begin{align*}
		M&=L^{G(L/M)}=L^{\Phi(M)}\\
		&=\Psi(\Phi(M))
		\end{align*}
		Somit ist $\Psi\circ\Phi=\isom_Z$.\\
		Sei $H\in U$. Dann ist $L/L^H$ eine Galois-Erweiterung mit Galoisgruppe $H$. Also ist
		\begin{align*}
		H&=G(L/L^H)=\Phi(L^H)\\
		&=\Phi(\Psi(H))
		\end{align*}
		d.h. $\Phi\circ\Psi=\id_U$.\\
		Somit sind $\Phi$ und $\Psi$ zueinander inverse Bijektionen.\\
		\\
		Sei $M$ ein Zwsichenkörper von $L/K$. Dann ist $M=L^H$ für ein $H\in U$.\\
		Ist die Erweiterung $M/K$ normal, so ist die Abbildung
		\begin{align*}
		\varphi:G(L/K)&\to G(M/K)\\
		\sigma&\mapsto \sigma_M
		\end{align*}
		ein surjektiver Gruppenhomomorphismus.\\
		Sei $\ol L$ ein algebraischer Abschluss von $L$. Dann ist $\ol L$ auch ein algebraischer Abschluss von $K$ und von $M$. Sei $\sigma\in G(L/K)$. Dann ist
		\[M\xrightarrow{\sigma}\ol L\]
		Da $M$ normal ist gilt $\sigma(M)=M$ d.h. $\sigma|_M\in G(M/K)$.\\
		Also ist $\varphi$ wohldefiniert. Weiterhin gilt
		\[(\sigma_1\sigma_2)|_M=\sigma_1|_M\sigma_2|_M\]
		Sei $\sigma\in G(M/K)$. Dann lässt sich die Abbildung
		\[M\xrightarrow{\sigma}\ol L\]
		fortsetzen zu einem $K$-Homomorphismus
		\[L\xrightarrow{\sigma}\ol L\]
		weil $L/M$ algebraisch ist. Da $L/K$ normal ist folgt $\sigma(L)=L$. $\varphi$ ist also surjektiv.\\
		Es gilt $\Kern(\varphi)=G(L/M)$, d.h. $G(L/M)$ ist eine normaler Untergruppe von $G(L/K)$.\\
		Sei nun $H$ eine normale Untergruppe von $G(L/K)$. Wir zeigen, dass die Erweiterung $L/L^H$ normal ist:\\
		Sei $\ol L$ ein algebraischer Abschluss von $L$ und $\sigma:L^H\to\ol L$ ein $K$-Homomorphismus. Dann gilt $\sigma(L^H)=L^H$. Da $K\subset L^H\subset L\subset\ol L$ können wir $\sigma$ zu einem $K$-Homomorphismus $\sigma:L\to\ol L$ fortsetzen weil $L/L^H$ algebraisch ist. Da $L/K$ normal ist gilt $\sigma(L)=L$. Wir können $\sigma$ also auffassen als $K$-Homomorphismus $\sigma:L^H\to L$.\\
		Sei $b\in\sigma(L^H)$. Dann ist $b=\sigma(a)$ für ein $a\in L^H$.\\
		Sei $\tau\in H$. Da $H\sigma=\sigma H$ ist gibt es $\tau'\in H$, sodass
		\[\tau(b)=\tau(\sigma(a))=\sigma(\underbrace{\tau'(a)}_{=a})=\sigma(a)=b\]
		d.h. $b\in L^H$ und $\sigma(L^H)\subset L^H$.\\
		\\
		Zum Beweis der Gleichheit setzen wir den $K$-Homomorphismus
		\[\underbrace{\sigma(L^H)}_{\subset L^H}\xrightarrow{\sigma^{-1}}L^H\to\ol L\]
		zu einem $K$-Homomorphismus $\rho:L^H\to\ol L$ fort.\\
		Diesen können wir wie oben als $K$-Homomorphismus $L^H\to L$ auffassen.\\
		Dann ist $\rho(L^H)\subset L^H$ und
		\[L^H\xrightarrow{\sigma}L^H\xrightarrow{\rho}L^H\]
		ist die Identität auf $L^H$, d.h. $\rho\sigma=\id_{L^H}$.\\
		\\
		Analog konstruieren wir einen $K$-Homomorphismus $\eta:L^H\to L$ mit $\eta(L^H)\subset L^H$ und $\eta\rho=\id_{L^H}$.\\
		Es folgt
		\[\sigma\rho=\id_{L^H}\sigma\rho=\eta\rho\sigma\rho=\eta\rho=\id_{L^H}\]
	\end{proof}

	\begin{satz}
		Sei $L/K$ eine endliche Galois-Erweiterung. Seien $L_1,L_2$ Zwischenkörper von $L/K$ die zu Untergruppen $H_1$ und $H_2$ von $G(L/K)$ korrespondieren.\\
		Dann gilt für $\sigma\in G(L/K)$
		\[\sigma(L_1)=L_2\Leftrightarrow\sigma H_1\sigma^{-1}=H_2\]
	\end{satz}

	\begin{satz}[Translationssatz]
		Seien $L/K$ und $M/K$ Körpererweiterungen, sodass $L$ und $M$ in einem Gemeinsamen Erweiterungskörper von $K$ enthalten sind.\\
		Ist $L/K$ eine endliche Galois-Erweiterung, so ist auch $L/K$ eine endliche Galois-Erweiterung und die Abbildung
		\begin{align*}
		G(L\cdot M/M)&\to G(L/K)\\
		\sigma&\mapsto\sigma|_L
		\end{align*}
		definiert einen Isomorphismus 
		\[G({L\cdot M}/M)\isom G(L/{L\cap M})\]
		(Dabei ist $L\cdot M$ das Kompositum $L\cdot M:=L(M)=M(L)$)
	\end{satz}
	\begin{proof}
		Sei $a$ ein Primelement der Erweiterung $L/K$ und seien $a_1,...,a_n$ die Nullstelle von $m_{a,K}$ in $L$. Dann ist
		\[L=K(a_1,...,a_n)\]
		und damit
		\[L\cdot M=M(L)=M(a_1,...,a_n)\]
		d.h. ${L\cdot M}/M$ ist eine endliche Galois-Erweiterung.\\
		\begin{description}
			\item[Wohldefiniertheit] Sei $\sigma\in G({L\cdot M}/M)$ und $b\in L$. Dann zerfällt $m_{b,K}$ in $L$, also
			\[m_{b,K}=\prod_{j=1}^n(X-\underbrace{b_j}_{\in L})\]
			und
			\[m_{b,K}=\sigma(m_{b,K})=\prod_{j=1}^{n}(X-\underbrace{\sigma(b_i)}_{\in L})\]
			Es folgt $\sigma(b)\in L$.\\
			\item[Injektivität] Sei $\sigma\in G({L\cdot M}/M)$ mit $\sigma|_L=\id_L$. Aus $L\cdot M=M(L)=M(a_1,...,a_n)$ und $\sigma(a_i)=a_i$ folgt $\sigma=\id$.
		\end{description}
		Sei $H$ das Bild der Abbildung. Dann ist
		\[L^H=L\cap (L\cdot M)^{G({L\cdot M}/M)}=L\cap M\]
		Die Erweiterung $L/L^H$ ist eine endliche Galois-Erweiterung mit Galoisgruppe $H$. Aus der Injektivität der Abbildung folgt
		\[G({L\cdot M}/M)\isom H=H(L/L^H)=G(L/{L\cap M})\]
	\end{proof}

	\begin{theorem}[Produktsatz]
		Seien $L_1/K$ und $L_2/K$ endliche Galois-Erweiterungen, sodass $L_1$ und $L_2$ in einem gemeinsamen Erweiterungskörper enthalten sind. Dann ist ${L_1\cdot L_2}/K$ eine endliche Galois-Erweiterung und die Abbildung
		\begin{align*}
		G({L_1\cdot L_2}/K)\to G(L_1/K)\times G(L_2/K)\\
		\sigma&\mapsto(\sigma|_{L_1},\sigma|_{L_2})
		\end{align*}
		definiert einen injektiven Gruppenhomomorphismus.\\
		Ist $L_1\cap L_2=K$, so ist die Abbildung ein Isomorphismus.
	\end{theorem}
	\begin{proof}
		Sei $L_1=K(a)$ und $L_2=K(b)$. Seien $a_1,...,a_n$ die Nullstellen von $m_{a,K}$ und $b_1,...,b_n$ die Nullstellen on $m_{b,K}$. Dann ist
		\begin{align*}
		L_1&=K(a_1,...,a_n)\\
		L_2=K(b_1,...,b_m)\\
		L_1\cdot L_2&=L_1(L_2)=L_2(L_1)\\
		&=K(a_1,...,a_n,b_1,...,n_m)
		\end{align*}
		${L_1\cdot L_2}/K$ ist als einen endliche Galois-Erweiterung.\\
		\begin{description}
			\item[Wohldefiniertheit] wie oben.
			\item[Injektivität] Sei $\sigma\in G({L_1\cdot L_2}/K)$ mit $\sigma|_{L_1}=\id_1$ und $\sigma|_{L_2}=\id_2$.\\
			Dann folgt, aus $L_1\cdot L_2=L_1(L_2)$, dass $\sigma=\id_{l_1\cdot L_2}$ ist.\\
			Die Gruppen $G({L_1\cdot L_2}/L_1)$ und $G({L_1\cdot L_2}/L_2)$ sind Untergruppen von $G({L_1\cdot L_2}/K)$.\\
			Sei nun $L_1\cap L_2=K$. Dann
			\[G({L_1\cdot L_2}/L_1)\cap G({L_1\cdot L_2}/L_2)=\{1\}\]
			Aus dem Translationssatz folgt dann
			\begin{align*}
			G({L_1\cdot L_2}/L_1)&\isom G(L_2/{L_1\cap L_2}) = G(L_2/K)\\
			G({L_1\cdot L_2}/L_2)&\isom G(L_1/{L_1\cap L_2}) = G(L_1/K)
			\end{align*}
			Die Abbildung ist in diesem Fall also ein Isomorphismus.
		\end{description}
	\end{proof}

	\begin{theorem}
		Sei $L/K$ eine endliche Galois-Erweiterung und sei $a$ ein primitives Element, d.h. $L=K(a)$. Sei außerdem $H\subset G(L/K)$. Dann ist\[L^H=K(a_0,...,q_1)\]
		wobei die $a_i$ die Koeffizienten von
		\[f=\prod_{\sigma\in H}(X-\sigma(a))=\sum_{i=0}^{n}a_iX^i\]
		sind.
	\end{theorem}

	%VL 10.01.2018
	
	
	\subsection{Die Galoisgruppe einer Gleichung}
	In diesem Abschnitt sei $K$ ein Körper
	\begin{definition}
		Sei $f$ ein separabeles Polynom und $L$ ein Zerfällungskörper von $f$ über $K$. Dann ist $L/K$ einen endliche Galois-Erweiterung und $G(L/K)$ eird in diesem Falls als \textbf{Galoisgruppe von $f$ über $K$} bezeichnet.
	\end{definition}

	\begin{satz}
		Sei $f\in K[X]\setminus K$ separabel und vom Grad $n$ mit Zerfällungskörper $L$ über $K$.\\
		Seien $a_1,...,a_n$ die Nullstellen von $f$ in $L$. Dann definiert die Abbildung
		\begin{align*}
		G(L/K)&\to S(\{a_1,...,a_n\})\\
		\sigma&\mapsto\sigma|_{\{a_1,...,a_n\}}
		\end{align*}
		einen injektiven Gruppenhomomorphismus. Insbesondere gilt $\abs{G(L/K)}|n!$.\\
		$f$ ist genau dann irreduzible über $K$ wenn $G(L/K)$ transitiv auf dem Nullstellen on $f$ operiert.
	\end{satz}
	\begin{proof}
		Sei $\sigma\in G(L/K)$. Da $\sigma(f)=f$ bildet $\sigma$ Nullstellen von $f$ in Nullstellen von $f$ ab.\\
		Da $\sigma$ injektiv ist, ist die Einschränkung auf $\{a_1,...,a_n\}$ eine Bijektion.\\
		Wegen $L=K(a_1,...,a_n)$ ist $\sigma\in G(L/K)$ eindeutig durch seine Operation auf $\{a_1,...,a_n\}$ festgelegt.\\
		Somit ist $f$ injektiv.\\
		\\
		Wir haben bereits gesehen, dass $G(L/K)$ trasnitiv auf den Nullstellen von $f$ operiert, wenn $f$ irreduzibel ist.\\
		Angenommen $G(L/K)$ permutiert die Nullstellen von $f$ transitiv.\\
		Sei $a$ eine Nullstellen von $f$. Dann sind die Nullstellen von $f$ gegeben durch $\sigma_1(a),...,\sigma_n(a)$ für geeignete $\sigma_i\in G(L/K)$ und
		\[f=c\prod_{i=1}^{n}\big(X-\sigma(a)\big)\]
		Es ist $f=cm_{a,K}$, denn $\sigma_1(a),...,\sigma_n(a)$ sind auch Nullstellen on $m_{a,K}$. Somit ist $f$ irreduzibel.
	\end{proof}

	\begin{kor}\label{04korTeilbarkeit}
		Sei $L/K$ eine endliche Galoiserweiterung vom Grad $n$. Dann ist $G(L/K)$ eine Untergruppe von $S_n$.
	\end{kor}

	\begin{exm}
		Sei $K$ ein Körper mit $\cha(K)\neq 2$, $f\in K[X]$ ein irreduzibles, separables, normiertes Polynom vom Grad $3$ und $L$ ein Zerfällungskörper von $f$. Dann gilt
		\[G(L/K)=\begin{cases}
		\Z/3\Z&\text{, falls $\Delta f$ ein Quadrat in $K$ ist}\\
		S_3&\text{, sonst}
		\end{cases}\]
	\end{exm}
	\begin{proof}
		Sei $a$ einen Nullstelle von $f$ in $L$. Dann ist
		\[[L:K]=[L:K(a)]\underbrace{[K(a):K]}_{=3}\]
		weil $f$ irreduzibel und nach \ref{04korTeilbarkeit} muss $[L:K]$ teilt $6$.\\
		D.h. $[L:K]=3$ oder $=6$. Im ersten Fall ist $G(L/K)$ eine Untergruppe von $S_3$ mit Index $2$. Also muss $G(L/K)\isom A_3\isom\Z/3\Z$.\\
		\\
		Seiene $a_1,a_2,a_3$ die Nullstellen von $f$ in $L$. Dann ist
		\[\delta:=(a_1-a_2)(a_1-a_3)(a_2-a_3)\neq 0\]
		Dann ist $\Delta(f)=\delta^2$.\\
		Falls $G(L/K)=S_3$ ist, so gilt
		\[\tau(\delta)=\operatorname{sgn}(\tau)\delta\]
		für alle $\tau\in G(L/K)$.\\
		Ist $G(L/K)=A_3$, so gilt $\tau(\delta)=\delta$ für alle $\tau\in G(L/K)$.\\
		Da $\cha(K)\neq 2$ folgt
		\[G(L/K)=A_3\Leftrightarrow\tau(\delta)=\delta\forall\tau\in G(L/K)\Leftrightarrow\delta\in K\]
	\end{proof}

	\begin{exm}
		Für $f=X^3+aX+b$ ist
		\[\Delta(f)=-4a^3-27b^2\]
		Das Polynom $f=X^3-x+1\in\Q[X]$ ist irreduzibel und hat Diskriminante 
		\[\Delta(f)=4-27=-23\]
		somit gilt für den Zerfällungskörper $L$ von $\Q$, dass $G(L/\Q)=S_3$.
	\end{exm}

	%VL 15.01.2018
	\begin{exm}
		Sei $f=X^4-2\in\Q[X]$. Dann gilt
		\[f=(X-a)(X+a)(X-ia)(X+ia)\]
		mit $a=\sqrt[4]{2}$. Der Zerfällungskörper von $f$ über $\Q$ ist $L=\Q(a,i)$.\\
		Das Eisenstein Kriterium zeigt, dass $f$ irreduzibel über $\Q$ ist. Somit ist $f=m_{a,\Q}$ und $[Q(a),\Q]=4$.\\
		Weiterhin ist $[L:\Q(a)]=2$, da $\Q(a)$ keine negativen Quadrate hat und damit nicht $i$ enthält. Es folgt
		\[[L:\Q]=8\]
		\\
		Wir bestimmen die Galoisgruppe von $f$. Da $f$ 4 Nullstellen hat und die Galoisgruppe die Nullstellen permutiert muss $G(L/\Q)\subset S_4$ sein.\\
		Jedoch muss zusätzlich für $\sigma\in G(L/K)$ gelte, dass
		\begin{align*}
		\sigma(-a)&=-\sigma(a)\\
		\sigma(-ia)&=-\sigma(ia)
		\end{align*}
		Es gibt $8$ Permutationen in $S(\{a,-a,ia,-ia\})$ die die Bedingungen erfüllen.\\
		Diese sind somit die Elemente in $G(L/\Q)$.\\
		Seien $\sigma,\tau\in G(L/\Q)$ durch
		\begin{align*}
		\sigma(a)&=ia\\
		\sigma(ia)&=-a
		\end{align*}
		(d.h. $\sigma(i)=i$)
		\begin{align*}
		\tau(a)&=-a\\
		\tau(ia)&=ia
		\end{align*}
		(d.h. $\tau(i)=-i$)\\
		\\
		Die von $\sigma$ erzeugt Untergruppe $\grp{\sigma}$ hat Ordnung $4$ und ist somit normal in $G(L/\Q)$.\\
		Weil $\tablename\notin\grp{\sigma}$ gilt
		\begin{align*}
		G(L/\Q)&=\grp{\sigma}\cup\grp{\sigma}\tau\\
		&=\grp{\sigma}\cup\tau\grp{\sigma}\\
		&=\{1,\sigma,\sigma^2\sigma^3,\tau,\tau\sigma,\tau\sigma^2,\tau\sigma^3\}
		\end{align*}
		$\tau$ und $\sigma$ genügen der Relation $\tablename\sigma=\sigma^3\tau$.\\
		Für Untergruppen von $G(L/\Q)$ erhält man folgendes Schema \ref{fig:GalUG}
		
		\begin{figure}[h]
			\centering
			\[\begin{tikzcd}
				&&G(L/\Q)\ar[lld]\ar[d]\ar[rrd]&&\\
				\{1,\sigma,\sigma^2,\tau,\sigma^2\tau\}\ar[d]\ar[rd]\ar[rrd]&&\{1,\sigma,\sigma^2\sigma^3\}\ar[d]&&\{1,\sigma^2,\sigma\tau,\sigma^3\tau\}\ar[lld]\ar[ld]\ar[d]\\
				\{1,\tau\}\ar[drr]&\{1,\sigma^2\tau\}\ar[rd]&\{1,\sigma^2\}\ar[d]&\{1,\sigma\tau\}\ar[dl]&\{1,\sigma^3\tau\}\ar[lld]\\
				&&\{1\}&&
			\end{tikzcd}\]
			\caption{Untergruppen}
			\label{fig:GalUG}
		\end{figure}
	\end{exm}
	%TODO satz und def Trennen
	\begin{definition}
		Sei $L=K(X_1,...,X_n)$ der Quotientenkörper von $K[X_1,...,X_n]$. Die Element von $L$ sin die rationalen Funktionen $f/g$ mit $f,g\in K[X_1,...,X_n]$ und $g\neq 0$.\\
		$S_n$ operiert durch Permutationen der $X_i$ auf $L$.\\
		$M=L^{S_n}$ wird als Körper der \textbf{symetrischen rationalen Funktionen} bezeichnet.
		Die Erweiterung $L/M$ ist eine endliche Galois-Erweiterung mit Galoisgruppe $S_n$.
	\end{definition}
	\begin{proof}
		Es gilt $M=K(s_1,...,s_n)$:\\
		Die Inklusionen $K(s_1,...,s_n)\subset M\subset L$ impliziert
		\[L:K(s_1,...,s_n)=\underbrace{[L:M]}_{=n!}[M:K(s_1,...,s_n)]\]
		Das Polynom
		\[f=\prod_{i=1}^{n}(X-X_i)\in K(s_1,...,s_n)[X]\subset L[X]\]
		ist separabel und hat $L$ als Zerfällungskörper. Also ist
		\[[L:K(s_1,...,s_n)]\leq n!\]
		Es folgt die Behauptung.
	\end{proof}
	
	\begin{satz}
		Sei $G$ eine endliche Gruppe, dann gibt es eine Galois-Erweiterung $L/K$ mit $G(L/K)\isom G$.
	\end{satz}
	\begin{proof}
		Sei $n\abs{G}$. Für $a\in G$ definiere
		\begin{align*}
		\tau_a:G&\to G\\
		g&\mapsto ag
		\end{align*}
		Dann ist $\tau_a$ eine Permutation von $G$. Weiterhin ist
		\[\tau_{a}\tau_b=\tau_{ab}\]\\
		Wir haben also eine Injektion
		\[G\to S_n\]
		Wir können $G$ also mit einer Untergruppe von $S_n$ identifizieren.\\
		Dann operiert $G$ auf $L=K(X_1,...,X_n)$ durch Permutation der $X_i$.\\
		Sei $M=L^G$ dann ist $L/M$ eine Galoiserweiterung mit Galoisgruppe $G$.
	\end{proof}
	
	
	
	
	\subsection{Kreisteilugspolynome}
	In diesem Abschnitt sei $K$ ein Körper und $\ol K$ ein algebraischer Abschluss von $K$.
	\begin{definition}
		Die Nullstellen des Polynom $X^n-1$ $n\geq 0$ werden als $n$-te \textbf{Einheitswurzeln} in $\ol K$ bezeichnet.\\
	\end{definition}
	\begin{prop}
		Die $n$-ten Einheitswurzeln bilden einer Untergruppe $U_n$ von $\ol K^*$.\\
		Ist $\cha(K)=0$ oder $\cha(K)\not|n$, so haben $X^n-1$ und seine Ableitung $nX^{n-1}$ keine gemeinsamen Nullstellen. Also ist $X^n-1$ separabel.\\
		In diesem Fall ist $\abs{U_n}=n$.\\
		Falls $\cha(K)=p>0$ und $p|n$, so schreibt man $n?mp^r$ mit $(m,p)=1$.\\
		Dann ist
		\[(X^m-1)^{p^r}=X^n-1\]
		Die Nullstellen von $X^m-1$ stimmen mit den Nullstellen von $X^n-1$ überein und $U_m=U_n$.\\
	\end{prop}

	\begin{satz}
		Sei $K$ ein Körper und $n\in\Z$, $n>0$ mit $\cha(K)\not|n$, dann ist $U_n$ eine zyklische Gruppe der Ordnung $n$.
	\end{satz}

	\begin{definition}
		$\xi\in U_n$ heißt \textbf{primitive} $n$-te Einheitswurzel, wenn $\xi$ die Gruppe $U_n$ erzeugt. 
	\end{definition}

	\begin{satz}
		Seien $m,n\in\Z$, $m,n>0$ mit $(m,n)=1$ und $K$ ein Körper mit $\cha(K)\neq| mn$.\\
		Dann ist die Abbildung
		\begin{align*}
		U_m\times U_n&\to U_{mn}\\
		(\xi,\eta)&\mapsto\xi\eta
		\end{align*}
		ein Isomorphismus  von Gruppen.
	\end{satz}

	\begin{definition}
		Für $n\in\Z$, $n>0$ definiert
		\[\varphi(n)=\abs{(Z/n\Z)^*}\]
		die \textbf{Eulersche $\varphi$-Funktion}.
	\end{definition}

	\begin{lem}
		Ist $p$ eine Primzahl, so gilt
		\[\varphi(p^k)=p^k-p^{k-1}=p^k(1-\frac{1}{p})\]
	\end{lem}

	\begin{satz}
		Seien $m,n\in\Z$ mit $m,n>0$ und $(m,n)=1$. Dann ist
		\[\varphi(mn)=\varphi(m)\varphi(n)\]
	\end{satz}
	\begin{proof}
		Die aussage folgt aus dem Chinesischen Restsatz:\\
		Der Ring-Isomorphismus
		\begin{align*}
		\Z/mn\Z&\to \Z/m\Z\times \Z/n\Z\\
		(x\mod mn)&\mapsto(x\mod m,x\mod n)
		\end{align*}
		liefert einen Isomorphismus
		\[(\Z/mn\Z)^*\to(\Z/m\Z\times \Z/n\Z)^*=(\Z/m\Z)^*\times (\Z/n\Z)^*\]
		Daraus folgt die Multiplikativ der $\varphi$ Funktion.
	\end{proof}

	\begin{satz}
		Sei $n\in\Z$, $n>0$. Ein Element $a$ erzeugt die additive zyklische Gruppe $\Z/n\Z$ genau dann wenn $a$ eine Einheit in $\Z/n\Z$ ist.
	\end{satz}

	\begin{satz}
		Sei $K$ ein Körper und $n\in\Z$, $n\geq0$ mit $\cha(K)\not|n$. Dann enthält $U_n$ genau $\varphi(n)$ primitive $n$-te Einheitswurzeln.\\
		Ist $\xi$ primitive $n$-te Einheitswurzel, so ist $\xi^r$ genau dann primitive $n$-te Einheitswurzel, wenn $(r,n)=1$ ist.
	\end{satz}
	
	%Vl 17.01.2018
	
	\begin{satz}
		Sei $\cha(K)\not|n$ und $\xi$ eine primitive Einheitswurzel.\\
		Dann ist $K(\xi)$ der Zerfällungskörper
		 von $X^n-1$.\\
		 Außerdem ist $K(\xi)/K$ eine endliche Galois-Erweiterung.
	\end{satz}
	\begin{definition}
		Falls $K=\Q$ ist so heißt $\Q(\xi)$ der \textbf{$n$-te Kreisteilungskörper}.
	\end{definition}

	\begin{theorem}
		Sei $\xi\in\ol{\Q}$ eine primitive $n$-te Einheitswurzel. Dann ist $\Q(\xi)/\Q$ eine endliche Galois-Erweiterung mit 
		\[[\Q(\xi):\Q]=\varphi(n)\]
	\end{theorem}
	\begin{proof}
		Jedes $\sigma\in G(\Q(\xi)(\Q)$ bildet $U_n$ nach $U_n$ (Menge der $n$-ten Einheitswurzeln).\\
		Insbesondere ist $\sigma(\xi)$ wieder eine primitive $n$-te Einheitswurzel. \\
		Sei $f=m_{\xi,\Q}$. Da $f$  irreduzibel über $\Q$ ist operiert $G(\Q(\xi)/\Q)$ transivit auf den Nullstellen von $f$, d.h. jede Nullstelle von $f$ ist eine primitive $n$-te Einheitswurzel.\\
		Also gilt
		\[[Q(\xi):\Q]\leq\varphi(n)\]
		Wir zeigen jetzt, dass jede primitive $n$-te Einheitswurzel Nullstelle von $f$ ist.\\
		Da $\xi$ Nullstelle von $X^n-1$ ist gilt
		\[X^n-1=fg\]
		für ein normiertes $g\in\Q[X]$.\\
		Sei $p$ Primzahl. Wir betrachten die $p$-adische Bewertung. Da $X^n-1$ nicht von $p$ geteilt wird ist  %TODO ref p-adische Bewertung
		\begin{align*}
		0&=\nu_p(fg)\\
		0&=\underbrace{\nu_p(f)}_{\geq 0}+\underbrace{\nu_p(g)}_{\geq 0}
		\end{align*}
		Dann muss aber $\nu_p(f)=\nu_p(g)=0$ für alle Primzahlen $p$ gelten.\\
		Somit ist $f,g\in\Z[X]$.\\
		\\
		Sei nun $p$ eine Primzahl mit $p\not|n$. Dann ist $\xi^p$ eine primitive $n$-te Einheitswurzel.\\
		\\
		Angenommen $f(\xi^p)\neq0$, dann muss $g(\xi^p)=0$ (da $\xi$ Nullstelle von $fg$).\\
		D.h. $\xi$ ist Nullstelle von $X^p$, dann $f|g(X^p)$. Sei $g(X^p)=fh$, dann ist $h$ ein normiertes Polynom in $\Z[X]$.\\
		Reduzieren der Koeffizienten $\mod p$ 
		\[\Z[X]\to(\Z/p\Z)[X]\]
		Dann geht
		\[g=\sum_{j=0}^{m}a_jX^j\]
		über in
		\[\ol g=\sum_{j=0}^{m}\ol a_jX^j\]
		In $\F_p$ gilt $a^p=a$, sodass
		\begin{align*}
		\ol g^p&=\left(\sum_{j=0}^{m}\ol a_jX^j\right)^p\\
		&=\sum_{j=0}^{m}\ol a_j^pX^{jp}\\
		&=\sum_{j=0}^{m}\ol a_jX^{jp}\\
		&=\ol g(X^p)\\
		&=\ol f\ol h
		\end{align*}
		Aus $\ol g^p=\ol f \ol h$ folgt, dass $\ol f$ und $\ol g$ nicht teilerfremd sind in $\F_p$. Somit hat $X^n-1=\ol f\ol g$ merhfache Nullstellen in $\F_p$. Dies widerspricht $p\not|n$!\\
		Also muss $\xi^p$ eine Nullstelle von $f$.\\
		\\
		Sei nun $\eta$ eine beliebige primitive $n$-te Einheitswurzel. Dann ist $\eta=\xi^m$ mit $(m,n)=1$. Sei $m=p_1\cdot...\cdot p_k$ die Zerlegung von $m$ im Primfaktoren, sodass
		\[\eta=\xi^m=(...(\xi^{p_1})^{p_2}...)^{p_k}\]
		Also ist $\xi^{p_1}$ eine Nullstelle in $f$. $f$ ist auch das Minimaplolynom von $\xi^{p_1}$. Analog zeigt man, dass $(\xi^{p_1})^{p_2}$ eine Nullstelle von $f$ ist.\\
		Es folgt schließlich, dass $\eta$ eine Nullstelle von $f$ ist.\\ 
		\\
		Dann folgt
		\[\Q(\xi):\Q]=\varphi(n)\]
	\end{proof}
	
	
	\begin{satz}
		Seien $\xi_m,\xi_n\in\ol{\Q}$ primitive $m$-te bzw $n$-te Einheitswurzeln mit $(m,n)=1$.\\
		Dann ist
		\[\Q(\xi_m)\cap \Q(\xi_n)=\Q\]
	\end{satz}
	\begin{proof}
		Es ist $\xi_{mn}=\xi_n\xi_m$ auch primitive Einheitswurzel. Es folgt
		\[\Q(\xi_{mn})=\Q(\xi_m,\xi_n)\]
		und
		\[\underbrace{[\Q(\xi_{mn}):\Q]}_{=\varphi(mn)}=[\Q(\xi_{mn}:\Q(\xi_m))]\underbrace{[\Q(\xi_m):\Q]}_{=m}\]
		sodass
		\[[\Q(\xi_{mn}):\Q(\xi_m)]=\varphi(m)\]
		\begin{figure}[h]
			\centering
			\[\begin{tikzcd}
				&\Q(\xi_{mn})\ar[dl,"\varphi(n)"]\ar[dr,"\varphi(m)"]&\\
				\Q(\xi_m)\ar[dr,"\varphi(m)"]&&\Q(\xi_n)\ar["\varphi(n)",dl]\\
				&\Q&
			\end{tikzcd}\]
			\caption{Körperdiagramm mit Erweiterungsgrad}
			\label{fig:cdUnitRoot}
		\end{figure}
	
		Sei $L=\Q(\xi_m)\cap\Q(\xi_n)$. Es ist
		\begin{align*}
		\deg(m_{\xi_m,\Q(\xi_n)}&)=\varphi(m)\\
		\deg(m_{\xi_m,L}&)\geq\varphi(m)
		\end{align*}
		und
		\[\Q\subset L\subset \Q(\xi_m)\]
		\[\Q(\xi_m)\subset L(\xi_m)\subset \Q(\xi_m)\]
		d.h.
		\[L(\xi_m)=\Q(\xi_m)\]
		Damit folgt
		\[\underbrace{[L(\xi_m):\Q]}_{=\varphi(n)}=\underbrace{[L(\xi_m):L]}_{\geq \varphi(m)}[L:Q]\]
		Also muss $[L:\Q]=1$, also $L=\Q$.
	\end{proof}


	\begin{satz}
		Sei $\xi\in\ol K$ eine primitive $n$-te Einheitswurzel mit $\cha(K)\not|n$.\\
		Dann gilt
		\begin{enumerate}
			\item $K(\xi)$ ist der Zerfällungskörper des separablen Polynom $X^n-1$ über $K$.\\
			Und die Erweiterung $K(\xi)/K$ ist eine endliche Galois-Erweiterung mit Grad $\leq\varphi(n)$ und abelscher Galoisgruppe.
			\item Zu jedem $\sigma\in G(K(\xi)/K)$ gibt es eine positive ganze Zahl, $r(\sigma)$ mit $\sigma(\xi)=\xi^{r(\sigma)}$, wobei die Restklasse $\ol{r(\sigma)}\in \Z/n\Z$ eine Einheit ist, die unabhängig von der Wahl von $\xi$ eindeutig durch $\sigma$ bestimmt ist.\\
			\\
			Und die Abbildung
			\begin{align*}
			G(K(\xi)/K)&\to(\Z/n\Z)^*\\
			\sigma&\mapsto\ol{r(\sigma)}
			\end{align*}
			ist ein injektiver Gruppenhomomorphismus.
		\end{enumerate}
	\end{satz}
	\begin{proof}
		\stepcounter{enumi}
		\item Sei $\sigma\in G(K(\xi)/K)$. Dann ist $\sigma(U_n)=U_n$. Also $\sigma(\xi)=\xi^{r(\sigma)}$ für ein positive ganze Zahl $r(\sigma)$.\\
		Da $\xi$ primtive $n$-te Einheitswurzel ist ist $r(\sigma)$ eundeutig modulo $n$ und $\big(n,r(\sigma)\big)=1$.\\
		Es gilt
		\begin{align*}
		\sigma(\xi^s)=\sigma(\xi)^s&=(\xi^{r(\sigma)})^{s}=(\xi^s)^{r(\sigma)}
		\end{align*}
		sodass $r(\sigma)$ nicht von der Wahl von $\xi$ abhängt. Die Abbildung
		\begin{align*}
		\Psi:G(K(\xi)/K)&\to (\Z/n\Z)^*\\
		\sigma&\mapsto\ol{r(\sigma)}
		\end{align*}
		ist ein Gruppenhomomorphismus, denn für $\sigma,\tau \in G(K(\xi)/K)$
		\begin{align*}
		(\sigma\tau)(\xi)&=\sigma(\tau(\xi))\\
		&=\sigma(\xi^{r(\tau)})\\
		&=(\xi^{r(\tau)})^{r(\sigma)}\\
		&=\xi^{r(\tau)r(\sigma)}
		\end{align*}
		sodass
		\begin{align*}
		\Psi(\sigma\tau)&=\ol{r(\sigma\tau)}\\
		&=\ol{r(\sigma)r(\tau)}\\
		&=\ol{r(\sigma)}\ol{r(\tau)}\\
		&=\Psi(\sigma)\Psi(\tau)
		\end{align*}
		Aus $\ol{r(\sigma)}=1$ folgt, dass $\sigma(\xi)=\xi$, also ist $\sigma$ die Identität auf $K(\xi)$.
	\end{proof}
	
	
	\begin{kor}
		Sei $\xi\in\ol \Q$ eine primitive $n$-te Einheitswurzel. Dann ist $\Q(\xi)/\Q$ eine endliche Galois-Erweiterung mit Galoisgruppe $(\Z/n\Z)^*$.
	\end{kor}
	
	Wir zeigen nun, dass sich jede endliche abelsche Gruppe als Galoisgruppe über $\Q$ realisieren lässt.
	
	\begin{theorem}[Dirichlet]\label{satz:Dirichlet}
		Sei $a,b\in\Z$ mit $a,b>0$ und $(a,b)=1$. Dann enthält $\{a+nb\mid n\in\Z\}$ unendlich viele Primzahlen.
	\end{theorem}

%VL 22.01.2018
	\begin{theorem}
		Sei $G$ eine endliche abelsche Gruppe.Dann gibt es eine endliche Galoiserweiterung $K/\Q$ mit $G(K/\Q)\isom G$.
	\end{theorem}
	\begin{proof}
		$G$ zerfällt in zyklische Gruppen, d.h.\[G=\bigoplus_{i=1}^n \Z/p_i^{l_i}\]
		mit $p_i$ prim.\\
		Nach \ref{satz:Dirichlet} gilt $\{1+m_ip_i^{l_i}\}$ enthält unendliche viele Primzahlen, d.h. wir können  teilerfremde Primzahlen $q_i$ wählen, mit
		\[q_i=1\mod p_i^{l_i}\]
		Schreibe $q_i=1-m_ip_i^{l_i}$. Sei $q=\prod_{i=1}^n q_i$, $\xi\in\ol{\Q}$ eine primitive $q$-te Einheitswurzel und wähle $K=\Q(\xi)$. Dann ist
		\begin{align*}
		G(K/\Q)\isom \left(\Z/q\Z \right)^*\\
		&=\bigoplus_{i=1}^n\left(\Z/q_i\Z\right)^*\\
		&=\bigoplus_{i=1}^n\Z/m_ip_i^{l_i}\Z\\
		\end{align*}
		Wähle nun
		\[H_i=p_i^{l_i}\Z/m_ip_i^{l_i}\Z\]
		dann ist $H_i$ eine Untergruppe von $\Z/m_ip_i1^{l_i}\Z$ mit
		\[\frac{\Z/m_ip_i^{l_i}\Z}{H_i}=\Z/p_i^{l_i}\Z\]
		\\
		Definiere nun $H=\bigoplus_{i=1}^n H_i$. Dann ist
		\[G(K/\Q)/H\isom G\]
		d.h. $K^H/\Q$ ist eine Galoiserweiterung mit Galoisgruppe
		\[G(K^H/\Q)=\frac{G(K/\Q)}{G(K/K^H)}=\frac{G(K/\Q)}{H}=G\] 
	\end{proof}


	\begin{theorem}[Kronecker-Weber]
		Sei $K/\Q$ eine endliche Galoiserweiterung mit abelscher Galoisgruppe. Dann ist $K$ in einem Kreisteilungskörper enthalten.
	\end{theorem}

	\begin{definition}
		Sei $n\in\Z$, $n>0$ und $\cha(K)\not|n$. Seien $\xi_1,...,\xi_m$ mit $m=\varphi(n)$ die primitiven $n$-ten Einheitswurzeln in $\ol K$.\\
		Dann heißt
		\[\Phi_{n,K}=\prod_{i=1}^{m}(X-\xi_i)\]
		das $n$-te \textbf{Kreisteilungspolynom} über $K$.\\
		Im Fall $K=\Q$schreiben wir $\Phi_n$ für $\Phi_{n,K}$.
	\end{definition}

	\begin{satz}
		\begin{enumerate}
			\item $\Phi_{n,K}$ ist ein normiertes separables POlynom über $K$ vom Grad $\phi(n)$
			\item Für $K=\Q$ gilt $\Phi_n\in\Z[X]$ und $\Phi_n$ ist irreduzibel in $\Z[X]$ und in $\Q[X]$.
			\item $X^n-1=\prod_{d|n}\Phi_{d,K}$
		\end{enumerate}
	\end{satz}
	\begin{proof}
		\begin{enumerate}
			\item Sei $L=K(\xi_i)$. Dann ist $L/K$ eine Galoiserweiterung un $L^{G(L/K)}=K$. Sei $\sigma\in G(L/K)$.\\
			Dann permutiert $\sigma$ die Primitiven Einheitswurzeln, d.h. $\Phi_{n,K}=\Phi_{n,K}$. Somit liegen die Koeffizienten von $\Phi_{n,K}$ in $K$.
			\item Sei $\xi\in\ol \Q$ primitive $n$-te Einheitswurzel. Dann hat $m_{\xi,\Q}$ Grad $\varphi(n)$. Da $\Phi_n(\xi)=0$ ist und $\Phi_n$ Grad $\varphi(n)$ hat ist $\Phi_n=m_{\xi,\Q}$.\\
			Somit ist $\Phi_n$ irreduzibel über $\Q$. Aus $\Phi_n|(X^n-1)$ und der Normiertheit von $\Phi_n$ folgt $\Phi_n\in\Z[X]$.
			\item Es ist
			\begin{align*}
			X^n-1=\prod_{\xi\in U_n}(X-\xi)=\prod_{d|n}\prod_{\xi\in P_d}(X-\xi)\\
			&=\prod_{d|n}\Phi_{d,K}
			\end{align*} 
			wobei $P_d$ die Menge der $d$-ten Einheitswurzeln in $U_n$ ist.
		\end{enumerate}
	\end{proof}

	\begin{satz}
		Sei $n\in\Z$, $n>0$ und $p$ prim mit $p\not| n$. Sei $e$ die Ordnung von $p$ in $\left(\Z/n\Z\right)^*$. Dann zerfällt $\Phi_{n,\F_p}$ in $\varphi(n)/e$ verschiedene Faktoren vom Grad $e$ über $\F_p$.
	\end{satz}
	\begin{proof}
		Sei $f$ ein irreduzibler normierter Faktor von $\Phi_{n,\F_p}$. Dann ist $f$ das Minimalpolynom einer primitiven $n$-te Einheitswurzel $\xi\in\ol{\F_p}$ über $\F_p$. Sei $K=\F_p(\xi)$ und $m=[K:\F_p]$. Dann ist $m=\deg(f)$. \\
		Wir zeigen $m=e$:\\
		$\xi$ hat Ordnunng $n$ in $U_n$, $K^*$ ist zyklisch der Ordnung $P^{m-1}$. Es folgt
		\begin{align*}
			n&|p^m-1\\
			p^{m}&=1\mod n\\
			e&|m\\
			e&\leq m
		\end{align*}
		Andererseits folgt aus $p^e=1\mod n$, dass
		\[\xi^{p^e}=\xi^1=\xi\]
		so dass di Abbildung
		\begin{align*}
		k&\to K\\
		x&\mapsto x^{p^e}
		\end{align*}
		trivial auf $K$ ist, da das Polynom $X^{p^e}-X$ höchsten $p^e$ Nullstellen hat, ist
		\begin{align*}
		\abs{K}&\leq p^e\\
		p^m&\leq p^e\\
		m&\leq e
		\end{align*}
		Es folgt $m=e$.
	\end{proof}

	\begin{exm}
		Sei $p$ eine ungerade Primzahl und $\xi\in\ol{F_p}$ eine primitive 8-te Einheitswurzel. Dann ist
		\[G\left(\F_p(\xi)/\F_p\right)\hookrightarrow \left(\Z/8\Z\right)^*\]
		$=\{1,3,5,7\}=\Z/2\Z\times\Z/2\Z$.\\
		Somit ist
		\[G(\F_p(\xi)/\F_p)=\begin{cases}
		1&\text{, falls }p=1\mod 8\\
		\Z/2\Z\text{, sonst}
		\end{cases}\]
	\end{exm}

	\begin{bem}
		Sei $p$ eine ungerade Primzahl. Dann ist $(\Z/p^n\Z)^*$ zyklisch der Ordnung $p^{n}-p^{n-1}$.\\
		Für $p=2$ ist
		\begin{align*}
		\left(\Z/2\Z\right)^*&=1\\
		\left(\Z/4\Z\right)^*&=\Z/2\Z\\
		\left(\Z/2^n\Z\right)^*&=\Z/2\Z\times \Z/2^{n-2}\Z\text{ für $n\geq 3$}
		\end{align*}
	\end{bem}





	\section{Moduln}
	\subsection{Definitionen}
	\begin{definition}
		Sei $R$ ein Ring. Ein \textbf{Linksmodul} über $R$ ist eine abelsche Gruppe $M$ mit einer Abbildung
		\[R\times M\to M\]
		sodass
		\begin{align*}
		a(x+y)&=ax+ay\\
		(a+b)x&=ax+bx\\
		a(bx)&=(ab)x\\
		1x&=x
		\end{align*}
		für alle $a,b\in R$ und $x,y\in M$.
	\end{definition}

	\begin{definition}
		Seien $M',M$ $R$-Moduln. Eine Abbildung
		\[f:M\to M'\]
		heißt \textbf{$R-$linear} oder \textbf{Modulhomomorphismus}, wenn
		\begin{align*}
		f(x+y)&=f(x)+f(y)\\
		f(ax)&=af(x)
		\end{align*}
		für alle $a\in R$ und $x,y\in M$.
	\end{definition}

	\begin{exm}
		\begin{enumerate}
			\item Sei $G$ eine abelsche Gruppe. Dann ist $G$ ein $\Z$-Modul unter
			\[ng=\begin{cases}
			\underbrace{g+...+g}_{\text{$n$ Summanden}}&n>0\\
			0&n=0\\
			\underbrace{(-g)+...+(-g)}_{\text{$n$ Summanden}}&n<0
			\end{cases}\]
			\item Jeder $\Z$-Modul ist eine abelsche Gruppe (indem man die Modul-Struktur vergisst)
			\item Zwei $\Z$-Moduln sind genau dann isomorph, wenn sie als abelsche Gruppen isomorph sind.
			%VL 24.10.218
			\item Sei $R$ ein Ring und $M$ ein $R$-Modul und $f:M\to M$ ein Modulhomomorphismus. Dann ist $M$ ein $R[X]$-Modul unter
			\begin{align*}
			R[X]\times M&\to M\\
			\left(a_iX^i,v\right)&\mapsto\sum a_if^i(v)
			\end{align*}
			\item Für zwei $R$-Moduln $M$ und $M'$ ist die Menge der $R$-linearen Abbildungen unter
			\[(a f)(v)=af(v)\]
			ein $R$-Modul
		\end{enumerate}
	\end{exm}

	\begin{definition}
		Sei $;$ ein $R$-Modul. Ein Untermodul von $M$ ist eine Untergruppe $N$ von $M$, die Invariant unter Operstionen von $R$ ist, d.h. $ax\in N$ für alle $a\in R$, $x\in N$.
	\end{definition}

	\begin{exm}
		Sei $M$ ein $R$-Modul und $(M_i)_{i\in I}$ eine Familie von Untermoduln. Dann sind
		\[\bigcap_{i\in I}M_i\quad\text{ und }\quad \sum_{i\in I}M_i=\{\sum_{i\in I}x_i\mid x_i\in M_i\text{, fast allle $x_i=0$}\}\]
		Untermoduln von $M$.
	\end{exm}
	
	\subsection{Faktormoduln}
	\begin{definition}
		Sei $M$ ein $R$-Modul und $N\subset M$ ein Untermodul, so erhält man auf der \textbf{Faktorgruppe} $M/N$ eine $R$-Modulstruktur. Mit $a(x+N)=ax+N$ für $x\in M$, $a\in R$ wird $M/N$ als \textbf{Faktormodul} bezeichnet.\\
		Die Abbildung $\pi:M\to M/N$, $x\mapsto x+N$ ist ein Modulhomomorphismus.
	\end{definition}

	\begin{theorem}
		Seien $M$, $M'$ $R$-Moduln, $f:M\to M'$ ein Modulhomomorphismus und $N\subset\Kern(f)$ ein Untermodul von $M$. Dann gibt es eine eindeutigen Homomorphismus $\ol f:M/N\to M'$, sodass
		\begin{figure}[h]
			\centering
			\[\begin{tikzcd}
				M\ar[r,"f"]\ar[d]& M'\\
				M/N\ar[ur,"\ol f"]&
			\end{tikzcd}\]
		\end{figure}
	\end{theorem}

	\begin{satz}
		Sei $M$ ein $R$-Modul und $N$ ein Untermodul. Dann insuziert die Projektion $\pi:M\to M/N$ eine Bijektion zwischen den Untermoduln von $M$ die $N$ enthalten und den Untermoduln von $M/N$.
	\end{satz}

	\subsection{Direkte Summen und Produkte}
	\begin{definition}
		Sei $(M_i)_{i\in I}$ eine Familie von $R$-Moduln. \\
		Dann ist das \textbf{Modul-Produkt}
		\[\prod_{i\in I}M_i=\{(x_i)_{i\in I}\mid x_i\in M_i\}\]
		ein $R$-Modul und
		\[\bigoplus_{i\in I}M_i=\{(x_i)_{i\in I}\mid x_i\in M_i\text{ und fast alle $x_i=0$}\}\]
		ein Untermodul. Dieser wird als direkte Summe bezeichnet.
	\end{definition}

	\subsection{Erzeugendensysteme und Basen}
	\begin{definition}
		Sei $M$ ein $R$-Modul. Eine Familie $(x_i)_{i\in I}$ von Element in $M$ heißt \textbf{Erzeugendensystem} von $M$ über $R$, wenn
		\[m=\sum_{i\in I}Rx_i\]
		ist.\\
		Besitzt $M$ ein endliches Erzeugendensystem, so heißt $M$ \textbf{endliche erzeugt} oder \textbf{endlich}er $R$-Modul.\\
		Ein Familie $(x_i)_{i\in I}$ heißt \textbf{linear unabhängig}, wenn aus
		\[\sum_{i\in I}a_ix_i=0\]
		(mit fast alle $a_i=0$) folgt, dass alle $a_i=0$ sind.
	\end{definition}
	\begin{definition}
		Ein linear unabhängiges Erzeugendensystem wird als \textbf{Basis} bezeichnet.\\
		In diesem Falls lässt sich jedes $x\in M$ schreiben als
		\[x=\sum_{i\in I}a_ix_i\]
		mit eindeutig bestimmtem $a_i\in R$. In diesem Fall heißt $M$ \textbf{frei}.
	\end{definition}

	\begin{satz}
		Sei $R$ ein Ring mit $1\neq 0$ und $M$ ein $R$-Modul.\\
		Sind $(v_,...,v_m)$ und $(w_1,...,w_n)$ zwei $R$-Basen von $M$, so ist $m=n$.
	\end{satz}

	\subsection{Exakte Sequenzen}
	\begin{definition}
		Eine Folge von $R$-Moduln und $R$-linearen Abbildungen
		\[...\to M_{i-1}\xrightarrow{f_i}M_i\xrightarrow{f_{i+1}}M_{i+1}\to...\]
		heißt \textbf{exakt bei $M_i$}, wenn  $\Img(f_i)=\Kern(f_{i+1})$.
	\end{definition}
	
	\begin{definition}
		Eine Sequenz heißt \textbf{exakte Sequenz}, wenn sie an jedem $M_i$ exakt ist.
	\end{definition}

	\begin{definition}
		Ein \textbf{kurze exakte Sequenz} ist eine Sequenz der Form
		\[0\to M'\xrightarrow{f}M\xrightarrow{g}M''\to 0\]
		Exakheit bedeuet hierbei, dass $f$ injektiv, $g$ surjektiv und $\Img(f)=\Kern(g)$.
	\end{definition}

	\begin{exm}
		Sei $M$ ein $R$-Modul und $N\subset M$ ein Untermodul. Dann ist
		\[0\to N\hookrightarrow M\to M/N\to 0\]
		eine kurze exakte Sequenz.
	\end{exm}
	
	\begin{definition}
			Sei
		\[0\to M'\xrightarrow{f}M\xrightarrow{g}M''\to 0\]
		eine kurze exakte Sequenz von $R$-Moduln.\\
		Die Sequenz \emph{spaltet}, wenn es einen Untermodul $N\subset M$ mit $M=N\oplus\Kern(g)$ gibt.
	\end{definition}
	
	\begin{satz}
		Sei
		\[0\to M'\xrightarrow{f}M\xrightarrow{g}M''\to 0\]
		eine kurze exakte Sequenz von $R$-Moduln.\\
		Dann sind äquivalent:
		\begin{enumerate}
			\item Die Sequenz spaltet (Es gibt einen Untermodul $N\subset M$ mit $M=N\oplus\Kern(g)$)
			\item Es gibt eine $R$-lineare Abbildung $s:M''\to M$ mit $g\circ s=\id_{M'}$
			\item Es gibt eine $R$-lineare Abbildung $t:M\to M'$ mit $t\circ f=\id_{M'}$
		\end{enumerate}
	\end{satz}
	\begin{proof}
		\begin{description}
			\item[$1)\Rightarrow2)$] $g|_N'$ ist injektiv und
			$g(N')=g(M)+g(\Kern g)=M''$ %TODO Was? Warum?
			sodass $g:N\to M''$ einen Isomorphismus liefert.\\
			Wir erhalte die Abbildung $M''\xrightarrow{g^{-1}}N\hookrightarrow M$.\\
			Sei $s$ die Komposition dieser Abbildungen, dann isz $g\circ s=\id_{M''}$.
			\item[$1)\Rightarrow3)$] Es ist $M=\Kern(g)\oplus N'=f(M')\oplus N'$ und $f:M'\to M$ ist injektiv, also ist $M'\isom M'$, sodass wir die Abbildung $M\to f(M')\xrightarrow{f^{-1}} M'$ erhalten.\\
			Die Komposition erfüllt $f\circ f=\id_{M'}$.
			\item[$2)\Rightarrow1)$] Setze $N'=s(M'')$ und sei $x\in M$. Dann ist
			\[x=\underbrace{x-s\big(g(x)\big)}_{\in\Kern(g)}+\underbrace{s\big(g(x)\big)}_{\in N'}\]
			denn
			\[g\big(x-s(g(x))\big)=g(x)-gs\big(g(x)\big)=g(x)-\id g(x)=0\]
			Sei nun $y\in M''$ und $s(y)\in \Kern(g)$, dann ist $y=g\big(s(y)\big)=0$.
			\item[$3)\Rightarrow1)$]
		\end{description}
	\end{proof}

	\begin{satz}
		Sei $0\to M'\xrightarrow{f}M\xrightarrow{g}M''\to 0$ eine kurze exakte Sequenz von $R$-Moduln.\\
		Ist $M''$ frei, so spaltet die Sequenz $M\isom M'\oplus M''$.
	\end{satz}
	\begin{proof}
		Sei $(v_i)$ eine $R$-Basis von $M''$, $x_i\in M$ mit $g(x_i)=v_i$ für alle $i\in I$ und $s:M''\to M$ definiert durch $s(v_i)=x_i$.\\
		Dann gilt $g\circ s=\id_M$, d.h. die Folge spaltet.\\
		Außerdem gilt
		\begin{align*}
		N&\isom\Kern(g)\oplus s(M'')\\
		&\isom\Img(f)\oplus M''\\
		&\isom M'\oplus M''
		\end{align*}
	\end{proof}
	
%VL 29.01.2018
	\begin{kor}
		Sei
		\[0\to M'\xrightarrow{f}M\xrightarrow{g}M''\to 0\]
		eine kurze exakte Sequenz von $R$-Moduln. Sind $M'$ und $M''$  frei, so ist $M$ frei.
	\end{kor}
	\begin{proof}
		Da $M''$ frei ist gilt $M\isom M'\oplus M''$.
	\end{proof}

	\subsection{Endlich erzeugbare Moduln}
	\begin{definition}
		Ein $R$-Modul $M$ heißt \textbf{endlich erzeugbar}, wenn $M$ ein endliches Erzeugendensystem hat.\\
		Äquivalent: Es gibt einen surjektiven Homomorphismus $R^n\to M$.
	\end{definition}

	\begin{exm}
		Sei $K$ ein Körper und $R=K[X_1,X_2,...]$ der Polynomring über $K$ in abzählbar vielen Variablen und sei
		\[I=\{f\in R\mid \text{Konstanter Term }a_0=0\}\]
		Dann ist $I$ ein Ideal in $R$, d.h. $I$ ist ein $R$-Untermodul von $R$.\\
		Dann ist zwar $R$ endlich erzeugbar ($R^1\to R$) aber $I$ ist nicht endlich erzeugt als $R$-Modul.
	\end{exm}

	\begin{satz}
		Sei
		\[0\to M''\xrightarrow{f}M\xrightarrow{g}\to0\]
		eine kurze Exakte Sequenz von $R$-Moduln. Dann gilt
		\begin{enumerate}
			\item Ist $M$ endlich erzeugt, so auch $M''$.
			\item Sind $M'$ und $M''$ endlich erzeugt, so auch $M$.
		\end{enumerate}
	\end{satz}
	\begin{proof}
		\begin{enumerate}
			\item Ist $(v_1,...,v_n)$ ein Erzeugendensystem von $M$, so ist $\big(g(v_1),...,v(v_n)\big)$ ein Erzeugendensystem von $M''$.
			\item Sei $(v_1,...,v_N)$ ein Erzeugendensystem von $M'$ und $(x_1,...,w_m)$ ein Erzeugendensystem von $M''$. Setze
			\[s_i=f(v_i)\quad w_i=g(t_i)\]
			Dann ist $(s_1,...,s_n,t_1,...,t_m)$ ein Erzeugendensystem von $M$, denn:\\
			Sei $x\in M$. Dann gilt
			\begin{align*}
			g(x)&=\sum_{i=1}^{n}a_iw_i=\sum_{i=1}^{n}a_ig(t_i)
			\end{align*}
			Dann folgt, dass insbesondere
			\[g\left(x-\sum_{i=1}^{n}a_it_i\right)=0\]
			also ist
			\[x-\sum_{i=1}^{n}a_it_i\in\Kern(g)=\Img(f)\]
			\[x-\sum_{i=1}^{n}a_it_i=\sum_{j=1}^{m}b_js_j\]
			Sodass abschließend gilt
			\[x=\sum_{j=1}^{m}b_js_j+\sum_{i=1}^{n}a_it_i\]
		\end{enumerate}
	\end{proof}

	\begin{satz}
		Seien $M_1,...,M_n$ $R$-Moduln und sei $M=\bigoplus_{i=1}^nM_i$. \\
		Dann ist $M$ genau dann endlich erzeugt, wenn alle $M_i$ endlich erzeugt sind.
	\end{satz}
	\begin{proof}
		\begin{description}
			\item[\enquote{$\Leftarrow$}] Klar: Endliche Menge von endlichen Erzeugendensystemen.
			\item[\enquote{$\Rightarrow$}] Setze $M'=\bigoplus_{i\neq j}M_i$. Dann ist für jedes $j$
			\[0\to M'\xrightarrow{ }M\xrightarrow{ }M_j\to0 \]
			eine exakte Sequenz.\\
			Dann ist $M_j$ endlich nach \ref{satzdavor1)}.
		\end{description}
	\end{proof}

	\begin{definition}
		Ein $R$-Modul heißt \textbf{noethersch}, wenn jeder Untermodul von $M$ endlich erzeugbar ist.
	\end{definition}

	\begin{satz}
		Sei $M$ ein $R$-Modul. Dann sind äquivalent:
		\begin{enumerate}
			\item $M$ ist noethersch.
			\item Jede aufsteigende Kette von Untermoduln wird stationär.
			\item Jede nichtleere Teilmenge von Untermoduln von $M$ hat ein maximales Element
		\end{enumerate}
	\end{satz}

	\begin{satz}
		Sei \[0\to M'\xrightarrow{f}M\xrightarrow{g}M''\to0\]
		eine kurze exakte Sequenz von $R$-Moduln. Dann ist $M$ genau dann noethersch, wenn $M'$ und $M''$ noethersch sind.
	\end{satz}
	\begin{proof}
		\begin{description}
			\item[\enquote{$\Rightarrow$}] Sei $M$ noethersch. Dann ist $M'$ noethersch weil Untermoduln von $M'$ isomorph unter $f$ zu einem Untermodul von $M$ ist.\\
			Jeder Untermodul von $M''$ ist das homomorphe Bild eines Untermoduls von $M$ unter $g$ und somit endlich erzeugbar.
			\item[\enquote{$\Leftarrow$}] Seiene nun $M'$ und $M''$ noethersch. Sei $N$ ein Untermodul von $M$. Dann ist
			\[0\to f^{-1}\xrightarrow{f}N\xrightarrow{g}g(N)\to0\]
			exakte Sequenz. Da$f^{-1}(N)$ und $g(N)$ endlich erzeugt sind ist auch $N$ endlich erzeugt.
		\end{description}
	\end{proof}

	\begin{satz}
		Seien $M_1,...,M_n$ $R$-Moduln und sei $M=\bigoplus_{i=1}^n M_i$.\\
		Dann ist $M$ genau dann noethersch, wenn jedes $M_i$ noethersch ist.
	\end{satz}
	\begin{proof}
		\begin{description}
			\item[\enquote{$\Leftrightarrow$}] Durch Induktion über $n$. \\
			Für $n=1$ ist $M=M_1$.\\
			Sei $n>1$. Definiere $M'=\bigoplus_{i=1}^{n-1}M_i$. Dann definiert
			\[0\to M'\xrightarrow{ }M\xrightarrow{ }M_n\to0\]
			eine kurze exakte Sequenz bei der $M'$ und $M_n$ noethersch sind.\\
			Somit ist $M$ noethersch.
			\item[\enquote{$\Rightarrow$}] Sei $M'=\bigoplus_{i\neq j}$. Dann ist für jedes $j$
			\[0\to M'\xrightarrow{ }M\xrightarrow{ }M_j\to0\]
			eine kurze exakte Sequenz. Da $M$ noethersch ist auch $M_j$ noethersch.
		\end{description}
	\end{proof}

	\begin{satz}
		Sei $R$ ein noetherscher Ring und $M$ ein endlich erzeugbarer $R$-Modul. Dann ist $M$ noethersch.
	\end{satz}
	\begin{proof}
		Es gibt einen subjektiven Homomorphismus $g:R^n\to M$ und eine exakte Sequenz
		\[0\to \Kern(g)\xrightarrow{ }R^n\xrightarrow{g}M\to0\]
		Somit ist $M$ noethersch.
	\end{proof}




	\section{Ganze Ringerweiterungen}
	\subsection{Definitionen und Eigenschaften}
	\begin{definition}
		Sei $B$ ein Ring und $A\subset B$ ein Unterring.\\
		$x\in B$ heißt \textbf{ganz} über $A$, wenn es ein normiertes $f\in A[X]$ mit $f(x)=0$ gibt.
	\end{definition}

	\begin{satz}
		Sei $B$ ein Ring, $A\subset B$ ein Unterring und $x\in B$.\\
		Dann sind äquivalent:
		\begin{enumerate}
			\item $x$ ist ganz über $A$.
			\item Der Ring $A[x]$ ist ein endlich erzeugter $A$-Modul.
			\item Der Ring $A[x]$ ist ein einem Unterring $C\subset B$ enthalten, sodass $C$ ein endlich erzeugter $A$-Modul ist.
		\end{enumerate}
	\end{satz}
	\begin{proof}
		\begin{description}
			\item[\enquote{$1)\Rightarrow2)$}] Ist $x\in B$ ganz, so gibt es ein normiertes $f\in A[X]$ mit $f(x)=0$, d.h.
			\[x^n+a_{n-1}x^{n-1}+...+a_0=0\]
			für geeignete $a_i\in A$.\\
			Es folgt, dass
			\[x^n=-a_{n-1}x^{n-1}-...-a_0\]
			D.h. $A[x]$ wird von $1,x,x^2,...,x^{n-1}$ als $A$-Modul erzeugt.
			\item[\enquote{$2)\Rightarrow3)$}]Wähle $C=A[x]$.
			\item[\enquote{$3)\Rightarrow1)$}] Sei $C=\sum_{i=1}^{n}Ac_i$.\\
			Weil $A[x]\subset C$ gilt $xc_i\in C$.\\
			Es gibt also $\gamma_{ij}\in A$ mit
			\[xc_{i}=\sum_{j=1}^{n}\gamma_{ij}c_i\]
			Wir können diese Gleichung schreiben als
			\begin{align*}
			\sum_{j=1}^{n}(xc_j\delta_{ij}-\gamma_{ij}c_j)&=0\\
			\sum_{j=1}^{n}\underbrace{(x\delta_{ij}-\gamma_{ij})}_{:=m_{ij}}c_j&=0\\
			Mu&=0
			\end{align*}
			Definiere nun $M=(m_{ij})$ und $u=(c_1,...,c_n)^T$.  Sei $M^{ad}$ die zu $M$ adjungierte Matrix. Dann ist
			\[M^{ad}Mu=\det(M)u\]
			Es folgt
			\[\det(M)c_i=0\]
			und damit
			\[\det(M)c=0\]
			für alle $c\in C$. Da $1\in C$ ist $\det(M)=0$.\\
			Da $\det(M)$ ein normiertes Polynom in $X$ mit Koeffizienten in $A$ ist, ist $X$ ganz.
		\end{description}
	\end{proof}

%VL 30.01.2018
	\begin{kor}
		Sei $B$ ein Ring und $A$ ein Unterring.
		\begin{enumerate}
			\item Sind $x_1,...,x_n\in B$ ganz über $A$, so ist $A[x_1,...,x_n]$ ein endlich erzeugter $A$-Modul.
			\item Sei $B$ ein Unterring eines Rings $C$. Ist $B$ ein endlich erzeugter $A$-Modul und $y\in C$ ganz über $B$, so ist $y$ ganz über $A$.
		\end{enumerate}
	\end{kor}
	\begin{proof}
		\begin{enumerate}
			\item Durch Induktion über $n$. Im Fall $n=1$ gilt \ref{letzter Satz}.\\
			Sei $n>1$. Nach Induktionsvoraussetzung ist $A[x_1,...,a_{n-1}]$ ein endlich erzeugter $A$-Modul.\\
			$x_n$ ist ganz über $A$, somit ist $x_n$ auch ganz über $A[x_1,...,x_{n-1}]$. Somit ist $A[x_1,...,x_{n-1}][x_n]$ ein endlich erzeugter $A[x_1,...,x_{n-1}]$-Modul.
			\begin{align*}
			A[x_1,...,x_{n-1}]&=\sum_{i=1}^{k}Af_i
			\intertext{mit $f_i\in A[x_1,...,x_{n-1}]$}
			A[x_1,...,x_n]&=\sum_{j=1}^{l}A[x_1,...,x_{n-1}]g_j
			\intertext{mit $g_j\in A[x_1,...,x_{n}]$}
			=\sum_{j=1}^{l}\sum_{i=1}^{k}Af_ig_j
			\end{align*}
			Dann ist auch $A[x_1,...,x_n]$ ein endlich erzeugter $A$-Modul.
			
			\item $B[y]$ ist ein endlich erzeugter $B$-Modul. Da $B$ ein endlich erzeugter $A$-Modul ist gilt $A[y]\subset B[y]$ un dann mit \ref{satz davor}, dass $y$ ganz über $A$ ist.
		\end{enumerate}
	\end{proof}

	\begin{definition}
		Sei $B$ ein Ring und $A\subset B$ ein Unterring. Dann nennt man
		\[\ol A:=\{x\in B\mid \text{$x$ ist ganz über $A$}\}\]
		die \textbf{ganze Hülle} von $A$ in $B$.
	\end{definition}
	
	\begin{satz}
		Sei $B$ ein Ring und $A\subset B$ ein Unterring. Dann ist die ganze Hülle $\ol A$ von $A$ über $B$ ein Unterring von $B$.
	\end{satz}
	\begin{proof}
		Sind $x,y\in B$ ganz über $A$, so ist $A[x,y]$ ein endlich erzeugter $A$-Modul. Dieser enthält $x-y$, $x+y$ und $xy$. Somit sind diese Elemente ganz über $A[x,y]$ und somit auch über $A$.
	\end{proof}

	\begin{definition}
		Ist $\ol A=B$, so heißt $B$ \textbf{ganz} über $A$.
	\end{definition}

	\begin{satz}
		Seien $A\subset B\subset C$ Ringerweiterungen.\\
		Ist $C$ ganz über $B$ und $B$ ganz über $A$, so ist auch $B$ ganz über $A$.
	\end{satz}

	\begin{proof}
		Sei $c\in C$. Dann ist
		\[x^n+b_{n-1}c^{n-1}+...+b_0=0\]
		für geeigente $b_i\in B$.\\
		Sei $R=A[b_0,...,b_{n-1}]$. Dann ist $R[c]$ ein endlcih erzeugter $R$-Modul\\
		Da die $b_i$ ganz sind ist $R$ ein endlich erzeugter $A$-Modul. Somit ist $R$ ein endlich erzeugter $A$-Modul.\\
		Es folgt, dass $\ol A$  ganz abgeschlossen ist (also $x\in B$ ist ganz über $\ol A$ und $\ol A$ ist ganz über $A$. Also ist $x$ ganz über $A$).
	\end{proof}

	\begin{definition}
		Ein Integritätsbereich heißt \textbf{ganz abgeschlossen}, wenn er ganz abgeschlossen in seinem Quotientenkörper ist.
	\end{definition}

	\begin{satz}
		Sei $A$ ein faktorieller Integritätsbereich.\\
		Dann ist $A$ ganz abgeschlossen.
	\end{satz}
	\begin{proof}
		Sei $K$ der Quotientenkörper von $A$.\\
		Sei $\frac{a}{b}\in K$ mit $a,b\in A$,$(a,b)=1$ und ganz über $A$. Dann ist
		\[(\frac{a}{b})^n+c_{n-1}(\frac{a}{b})^{n-1}+...+c_0=0\]
		für geeignete $c_i\in A$.
		Multiplikation mit $b^n$ liefert
		\[a^n+c_{n-1}a^{n-1}b+...+c_0b^n=0\]
		d.h. $b|a^n$.\\
		Somit muss $b$ eine Einheit sein, also $\frac{a}{b}\in A$.
	\end{proof}


	\begin{satz}
		Sei $A$ ein Integritätsbereich mit Quotientenkörper $K$ und sei $A$ ganz abgeschlossen in $K$. Sei $L/K$ ein algebraisch Körpererweiterung.\\
		Dann ist $\al\in L$ genau dann ganz über $A$, wenn $m_{\al,K}\in A[X]$ liegt.
	\end{satz}
	\begin{proof}
		\begin{description}
			\item[\enquote{$\Leftarrow$}] Klar weil $m_{\al,K}$ normiert ist.
			\item[\enquote{$\Rightarrow$}] Sei $\al\in L$ ganz über $A$. Es gibt als eine normiertes Polynom $f\in A[X]$ mit $f(\al)=0$.\\
			In $K[X]$ gilt $m_{\al,K}|f$ \\
			Über einem geeigneten algebraisch Abschluss $\ol L$ von $L$ zerfällt $m_{\al,K}$ d.h.
			\[m_{\al,K}=\prod_{i=1}^n(X-\al_i)\]
			Aus $m_{\al,K}|f$ folgt, dass $f(\al_i)=0$ für alle $\al_i$.\\
			Somit ist jedes $\al_i$ ganz über $A$.\\
			Dann sind auch die Koeffizienten von $m_{\al,K}$ ganz über $A$.\\
			Da $A$ ganz abgeschlossen in $K$ ist gilt $m_{\al,K}\in A[X]$. 
		\end{description}
	\end{proof}



\subsection{Dedekindringe}
	\begin{definition}
		Ein Integritätsbereich $A$ heißt \textbf{Dedekindring}, wenn
		\begin{enumerate}
			\item $A$ noethersch
			\item $A$ ist ganz abgeschlossen
			\item Jedes Primideal $\neq 0$ ist maximal.
		\end{enumerate}
	\end{definition}

	\begin{definition}
		Ein \textbf{algebraischer Zahlkörper} $K$ ist eine endliche Erweiterung von $\Q$.
	\end{definition}

	\begin{definition}
		Die ganze Hülle von $\Z$ in $K$ wird als \textbf{Ring der ganzen Zahlen} in $K$ bezeichnet. Man schreibt diesen als
		\[O_K:=\{a\in K|\exists f\in \Z[X]\text{ normiert mit }f(a)=0 \}\]
	\end{definition}

	\begin{theorem}
		Sei $K$ ein algebraischer Zahlkörper. Dann ist $O_K$ ein Dedekindring.
	\end{theorem}

	\begin{exm}
		Sei $d\in\Z$, $n\neq 1$ und quadratfrei.\\
		Wähle $K=\Q(\sqrt{d})$ und
		\[\omega_d=\begin{cases}
		\sqrt d&\text{, falls $d=2,3\mod 4$}\\
		\frac{1}{2}(1+\sqrt{d})&\text{, falls $d=1\mod 4$}
		\end{cases}\]
		Dann ist
		\[O_K=\Z+\Z\omega_d\]
		\\
		Betrachte nun $\Q(\sqrt{-5})$. Dann ist in
		$O_K$
		\[21=3\cdot 7=(1-2\sqrt{-5})(1+2\sqrt{-5})\]
		D.h. man erhält Faktorisierungen in Primafaktoren von 21, die nicht zueinander assoziiert sind.
	\end{exm}

	\begin{theorem}
		Sei $A$ ein Dedekindring, $I\neq 0$ und $I\neq A$ ein Ideal in $A$.\\
		Dann gilt
		\[I=P_1...P_n\]
		mit eindeutigen Primidealen $P_i$.
	\end{theorem}



	\subsection{Der Noethersche Normalisierungssatz}
	Der Noethersche Normalisierungssatz impliziert den Hilbertschen Nullstellensatz und ist daher für die algebraische Geometrie von großer Bedeutung.
	
	\begin{theorem}\label{satz:501NeoNorm}
		Sei $K$ ein Körper und $B=[b_1,...,b_n]$ endlich erzeugter Ring. Dann existieren Elemente $x_1,...,x_r\in B$, die algebraisch unabhängig über $K$ sind, sodass $B$ als Modul endlich erzeugt über $K[x_1,...,x_r]$ ist.
	\end{theorem}
%VL 05.02.2018
	\begin{proof}
		Sind $b_1,...,b_n$ algebraisch unabhängig über $K$ so kann man $x_1,...,x_r\in B$ finden, die algebraisch unabhängig sind.\\
		Angenommen $b_1,...,b_n$ sind algebraisch abhängig über $K$. Dann existiert eine Relation
		\[\sum_{(\nu_1,...,\nu_n)\in I}a_{\nu_1,...,\nu_n}b_1^{\nu_i}...b_n^{\nu_n}=0\autotag\label{eq:NoeNormP1}\]
		mit $a_{\nu_1,...,\nu_n}\in K\setminus\{0\}$ und endlichem $I$.\\
		Sei
		\begin{align*}
		x_1&=b_1-b_n^{s}\\
		&\vdots\\
		x_{n-1}=b_{n-1}-b_n^{s_n-1}
		\end{align*}
		mit $s_1,...,s_{n-1}\in\N\setminus\{0\}$. Dann ist
		\begin{align*}
		B=K[b_1,...,b_n]\\
		&=K[x_1,...,x_{n-1},b_n]\\
		&=K[x_1,...,x_{n-1}][b_n]\\
		\end{align*}
		Setzt man $b_i=x_i+b_n^{s_i}$ in \ref{eq:NoeNormP1} und spaltet
		\[b_i^{\nu_i}=(x_i+b_n^{s_i})^{\nu_i}=b_n^{s_i\nu_i}+...\]
		so erhält man
		\[\sum_{(\nu_1,...,\nu_n)\in I} a_{\nu_1,...,\nu_n}b_n^{s_1\nu_1+s_2\nu_2+...+s_{n-1}\nu_{n-1}+\nu_n}+\underbrace{f(x_1,...,c_{n-1},b_n)}_{\in K[x_1,...,x_{n-1},b_n]}=0\autotag\label{eq:NoeNormP2}\]
		Dabei ist $f(x_1,...,x_{n-1},b_n)$ ein Polynom in $b_n$ mit Koeffizient in $K[x_1,...,x_{n-1}]$ wobei der Grad in $b_n$ echt kleiner ist als das Maximum der Summe $s_1\nu_1+...+s_{n-1}\nu_{n-1}+\nu_n$ mit $(\nu_1,...,\nu_n)\in I$.\\
		Wir können nun die Exponenten $s_1,...,s_{n-1}$ so wählen, dass die Summen $x_1\nu_1+...+s_{n-1}\nu_{n_1}+\nu_n$ für alle $(\nu_1,...,\nu_n)\in I$ paarweise verschieden sind.\\
		($\Q^n$ wird nicht durch endlich viele Hyperebenen ausgeschöpft)\\
		Dann ist \ref{eq:NoeNormP2} eine Gleichung der Form
		\[\underbrace{ab_n^{N}}_{\in K\setminus\{0\}}+\underbrace{g(x_1,...,x_{n-1,b_n})}_{\in K[x_1,...,x_{n-1}][b_n]}=0\]
		wobei $b_n^N$ die höchste Auftretenden Potenz von $b_n$ ist.
		Multiplikation mit $a^{-1}\in K$ zeigt, dass $b_n$ ganz über $K[x_1,...,x_{n-1}]$ ist.\\
		Somit ist
		\[B=K[x_1,...,x_{n-1}][b_n]\]
		ein endlich erzeugter $K[x_1,...,x_{n-1}]$-Modul.\\
		Sind $x_1,...,x_{n-1}$ algebraisch unabhängig über $K$ gilt die Behauptung. Ansonsten wenden wir das Verfahren auf den RIng $K[x_1,...,x_{n-1}]$ an und finden $y_1,...,y_{n-1}$, sodass $K[x_1,...,x_{n-1}]$ ein endlich erzeugter $K[y_1,...,y_{n-2}]$-Modul ist.\\
		Auf diese weise fährt man fort, bis man ein eine über $K$ algebraisch unabhängigen System gelangt ist.
	\end{proof}

	\begin{satz}
		Sei $A\subset B$ eine Ringerweiterungen, $B$ ganz über $A$ und seien $A$ und $B$ Integritätsbereiche.\\
		Dann ist $A$ genau dann Körper, wenn $B$ Körper ist.
	\end{satz}
	\begin{proof}
		Sei $A$ ein Körper und $b\in B\setminus\{0\}$. Wähle $f\in A[X]$ normiert udn minimalen Grades, sodass $f(b)=0$. Dann ist
		\[f=X^n+a_1X^{n-1}+...+a_n\]
		mit $a_n\neq 0$ und
		\begin{align*}
		b(b^{n-1}+a_{n-1}b^{n-1}+...+a_{n-1})&=0\\
		b\underbrace{\left(-\frac{1}{a_n}\right)}_{\in A}\underbrace{\left(b^{n-1}+a_{n-1}b^{n-2}+...+a_{n-1}\right)}_{\in B}=1
		\end{align*}
		Also ist $b\in B^*$.\\
		Sei $B$ ein Körper und $a\in A\setminus\{0\}$. Dann ist $a^{-1}\in B$ und $a^{-1}$ ist ganz über $A$, d.h.
		\[(a^{-1})^n+a_i(a^{-1})^{n-1}+...+a_n=0\]
		für geeignete $a_i\in A$.\\
		Es folgt 
		\begin{align*}
		a^{-n}&=-a_1a^{-n+1}-...-a_n\\
		a^{-1}&=\underbrace{-a_1-...-a_na^{n-1}}_{\in A}
		\end{align*}
		Also ist $A$ ein Körper.
	\end{proof}


	\begin{theorem}
		Sei $L/K$ eine Körpererweiterung und $L=K[x_1,...,x_n]$ für geeignete $x_1,...,x_n\in L$. Dann ist $L/K$ endlich.
	\end{theorem}
	\begin{proof}
		Nach dem Noetherschen Normalisierungssatz \ref{satz:501NeoNorm} gibt es über $K$ algebraisch unabhängige Elemente $y_1,...,y_r\in L$, sodass $L$ ein endlich erzeugter $K[y_1,...,y_r]$-Modul ist. Aus
		\[K[y_1,...,y_r]\subset L\] 
		folgt, dass $K[y_1,...,y_r]$ ein Körper ist. Also ist $r=0$.
	\end{proof}

	\begin{satz}
		Sei $K$ ein Körper und $\mathfrak m\subset K[X_1,...,X_n]$ ein maximales Ideal.\\
		Dann ist $L/K$ mit $L=K[X_1,...,X_n]/\mathfrak m$ eine endliche Körpererweiterung.
	\end{satz}
	\begin{proof}
		Es gilt $L=K[x_1,...,x_n]$ mit $x_i=X_i+m$.
	\end{proof}




	\subsection{Anfänge der algebraischen Geometrie}
	
	\begin{definition}
		Sei $K$ ein beliebiger Körper.
		\[A^n=A_K^n:=\{(a_1,...,a_n)\mid a_i\in K\}\]
		$A^n$ wird als \textbf{$n$-dimensionaler affiner Raum} bezeichnet.\\
	\end{definition}
	\begin{definition}
		Für $F\in K[x_1,...,x_n]$ definiert man
		\[V(F):=\{p\in A^n\mid F(p)=0\}\]
		die \textbf{V???-Menge}.\\ %TODO Varietäten?
		Für $S\subset K[X_1,...,X_n]$ sei
		\[V(S):=\{p\in A^n\mid F(p)=0\forall_{F\in S}\}=\bigcap_{F\in S}V(F)\]
	\end{definition}

	\begin{exm}
		Sei $n=2$, $K=\R$, $F=X^2_1-X_2$.
	\end{exm}

	\begin{definition}
		Eine Teilmenge $Y\subset A_n$ heißt algebraisch, wenn $y=V(S)$ für ein $S\subset K[X_1,...,X_n]$ ist.
	\end{definition}

	\begin{satz}
		Sei $S\subset K[X_1,...,X_n]$ und $I=(S)$ das erzeugte Ideal. Dann gilt
		\[V(S)=V(I)\]
	\end{satz}
	\begin{proof}
		\begin{description}
			\item[$\supset$] Ist klar.
			\item[$\subset$] Sei $p\in V(S)$ und $F\in I$. Dann ist $F=\sum c_iF_i$ mit $c_i\in K[X_1,...,X_n]$, $F_i\in S$ und
			\[F(p)=\sum c_i(p)\underbrace{F_i(p)}_{=0}=0\]
			Da $K[X_1,...,X_n]$ noethersch ist (Hilbertscher Basissatz \ref{HlbBasis}) ist
			\[I=(F_1,...,F_m)=\sum_i K[X_1,...,X_n]F_i\]
			für geeignete $F_i\in I$.\\
			Wie eben sieht man 
			\[V(I)=V\big((F_1,...,F_m)\big)=V(F_1,...,F_m)\]
		\end{description}
	\end{proof}
	
	%VL 07.02.2018
	
	\begin{definition}
		Sei $K$ ein Körper und $n\in\N$, dann ist $\A_K^n$ die Menge der Algebraischen Mengen in $K^n$.
		%TODO stimmt das so?
	\end{definition}
	
	\begin{exm}
		Betrachte $V\big(Y^2-X(X^2-1)\big)\subset \A^2_\R$.
		%TODO Img: X(X^2-1)
	\end{exm}

	\begin{satz}
		Die Abbildung
		\[V:\left\{\text{Ideale in}\atop{K[X_1,...,X_n]}\right\}\to\left\{\text{Algebraische Teilmengen von $\A^n_K$}\right\}\]
		hat folgende Eigenschaften
		\begin{enumerate}
			\item $V(0)=\A_K^n$, $V\big(K[X_1,...,X_n]\big)=\emptyset$
			\item Wenn $I\subset J$, dann gilt $V(J)\subset V(I)$.
			\item Für das Produkt gilt: $V(IJ)=V(I\cap J)=V(I)\cup V(J)$
			\item Für die Summe gilt: $V(\sum_i J_i)=\bigcap_i V(J_i)$
		\end{enumerate}
	\end{satz}
	\begin{proof}
		\begin{enumerate}
			Wir zeigen nur
			\addtocounter{enumi}{2}
			\item Es gilt
			\begin{align*}
			IJ\subset I\cap J&\subset I\\
			IJ\subset I\cap J&\subset J
			\end{align*}
			dann mit 2):
			\begin{align*}
			V(IJ)\supset V(I\cap J)&\supset V(I)\\
			V(IJ)\supset V(I\cap J)&\supset V(J)
			\end{align*}
			es folgt, dass
			\[V(IJ)\supset V(I\cap J)\supset V(I)\cup V(J)\]
			Sei nun $p\in \A_K^n\setminus(V(I)\cup V(J))$.\\
			Dann gibt es ein $f\in I$ mit $f(p)\neq 0$ und ein $g\in J$ mit $g(p)\neq 0$. Also ist 
			\[0\neq\underbrace{(fg)}_{\in IJ}(p)\]
			d.h. $p\notin V(IJ)$. Also ist $V(IJ)\subset V(I)\cup V(J)$ und es gilt Gleichheit.
		\end{enumerate}
	\end{proof}

	
	\begin{satz}
		Die Abbildung $I$ 
		\begin{align*}
		I:\left\{\text{Algebraische Teilmengen von $\A^n_K$}\right\}&\to\left\{\text{Ideale in}\atop{K[X_1,...,X_n]}\right\}\\
		M&\mapsto \{f\in K[x_1,...,x_n]\mid f(p)=0\forall p\in M\}
		\end{align*}
		hat folgende Eigenschaften:
		\begin{enumerate}
			\item Sei $M\subset N$, dann gilt $I(M)\supset I(N)$
			\item Für eine beliebige Teilmenge $M\subset \A_K^n$ gilt
			\[M\subset V\big(I(M)\big)\]
			Gleichheit gilt genau dann wenn $M$ algebraisch ist.
			\item Für ein Ideal $J\subset K[X_1,....,X_n]$ gilt
			\[J\subset I\big(V(J)\big)\]
		\end{enumerate}
	\end{satz}
	\begin{proof}
		Wir zeigen:
		\begin{enumerate}
			\stepcounter{enumi}
			\item \begin{description}
				\item[\enquote{$\Rightarrow$}] Sei $M=V\big(I(M)\big)$, so ist $M$ algebraisch.
				\item[\enquote{$\Leftarrow$}] Sei $M$ algebraisch, dann ist $M=V(J)$ für ein Ideal $J$. Dann ist
				\begin{align}
				J&\subset I(M)\\
				V(J)&\supset V\big(I(M)\big)
				\end{align}
				Es folgt Gleichheit.
			\end{description}
		\end{enumerate}
	\end{proof}

	\begin{definition}
		Sei eine Menge $\A_K^n$ abgeschlossen wenn sie algebraisch ist und deren Komplemente offen.\\
		Die erzeugte Topologie wird als \textbf{Zariski-Topologie} bezeichnet.
	\end{definition}

	\begin{exm}
		Sei $K$ algebraisch abgeschlossen. Dann sind in $\A_K^1$ die Menge $\A_K^1$ und $\{\}$ offen und abgeschlossen.\\
		Sei $M\subsetneqq\A^1_K$ abgeschlossen. Dann ist
		\begin{align*}
		M&\overset{\text{$M$ alg.}}{=}V(I)\overset{\text{$K[X]$ HIR}}{=}V(f)\\
		&=V\big((X-a_1)...(X-a_n)\big)\\
		\overset{\text{$K$ alg. abg.}}{=}\{a_1,...,a_n\}
		\end{align*}
		d.h. $M$ ist endlich. Die nicht-leeren offenen Teilmengen von $\A_K^1$ sind dicht in $A_K^1$.\\
		Die nicht-leere offenen Teilmenge von $A_K^1$ sind dicht in $A_K^1$.
	\end{exm}
	
	\begin{satz}
		Seien $a_1,...,a_n\in K$. Dann ist
		\[J=(X_1,a_1,x_2-a_2,...,X_n-a_n)\]
		maximal in $K[X_1,...,X_n]$ und $K$ ist isomorph zu $K[X_1,...,X_N]/J$
	\end{satz}
	\begin{proof}
		Sei $f\in  K[X_1,...,X_n]=K[X_1,...,X_{n-1}][X_n]$. Dann ist
		\[f=(X_n-a_n)g_n+c_n\]
		wobei $(X_n-a_n)$ Grad $1$ in $X_n$ hat, $g_n\in K[X_1,...,X_{n-1}][X_n]$ ist und $c_n$ Grad $0$ in $X_n$ hat, d.h. $c_n\in K[X_1,....,X_{n-1}]$. Dann folgt
		\begin{align*}
		f&=(X_n-a_n)g_n+(X_{n-1}a_{n-1})g_{n-1}+c_{n-1}\\
		&\vdots\\
		&=(X_n-a_n)g_n+...+(X_1-a_1)g_1+\underbrace{c_1}_{\in K}
		\end{align*}
		Also ist $K[X_1,...,X_n]/J=K$ und $J$ ist maximal.
	\end{proof}

	\begin{theorem}[Schwacher Nullteilersatz]
		Sei $K$ algebraisch abgeschlossen und
		\[J\subsetneqq K[X_1,...,X_n]\]
		Dann ist $V(J)\neq\emptyset$.
	\end{theorem}
	\begin{proof}
		$J$ ist in einem maximalen Ideal $\mathfrak m$ enthalten und $V(\mathfrak m)\subset V(J)$.\\
		Die Abbildung
		\[K\hookrightarrow K[X_1,...,X_n]\xrightarrow{\pi}L:=K[X_1,...,X_n]/\mathfrak m\]
		Liefert eine Einbettung $K\hookrightarrow L$.\\
		Sei $a_i=\pi(X_i)\in L$. Dann folgt aus dem Noethersche Normalisierungssatz \ref{satz:501NeoNorm}, dass $L=K[a_1,...,a_n]$ und dass $L/K$ algebraisch ist.\\
		Da $K$ algebraisch abgeschlossen ist folgt $K=L$.\\
		Weiterhin ist $X_i-a_i\in\mathfrak m$ und $(X_1-a_1,...,X_n-a_n)\subset \mathfrak m$. Da $(X_1-a_1,...,X_n-a_n)$ maximal ist folgt
		\[(X_1-a_1,...,X_n-a_n)=\mathfrak m\]
		Es folgt, dass
		\[V(\mathfrak m)=\{(a_1,...,a_n)\}\]
	\end{proof}

	\begin{theorem}[Hilbertscher Nullstellensatz]
		Sei $K$ algebraisch abgeschlossen und $J$ ein Ideal in $K[X_1,...,X_n]$. Dann gilt
		\[I\big(V(J)\big)=\rad(J)=\{f\in K[X_1,...,X_n]\exists n>0:\id f^n\in J\}\]
	\end{theorem}
	\begin{proof}[Rabinowitsch]
		\begin{description}
			\item[$\rad(J)\subset I\big(V(J)\big)$]: Sei $f\in \rad(J)$, dann ist $f^n$ in $J$. Dann $f^n\in I\big(V(J)\big)$, dann $f^n(p)=0\forall p\in V(J)$, dann $f(p)=0\forall p\in V(J)$ und damit $f\in I\big(V(J)\big)$.
			\item[$\rad(J)\supset I\big(V(J)\big)$] Sei $g\in I\big(V(J)\big)$. Schreibe $J=(f_1,...,f_t)$ und definiere
			\[I=(f_1,...,f_tX_{n+1}g-1)\subset K[X_1,...,X_n,X_{n+1}]\]
			Dann ist
			\begin{align*}
			f_1(x_1,...,x_n)&=0\\
			&\vdots\\
			f_t(x_1,...,x_n)&=0\\
			x_{n+1}f(x_1,...,x_n)-1&=0
			\end{align*}
			Dann ist $V(I)\subset A_K^{n+1}$ leer.\\
			(Denn: Sei $f_1(x_1,...,x_n)=0$,...,$f_t(x_1,...,x_n)=0$, dann ist $(x_1,...,x_n)\in V(J)$.\\
			Außerdem ist $x_{n+1}g(x_1,...,x_n)-1=0$ mit $g(x_1,...,x_n)=0$ auf $V(J)$. Widerspruch!)\\
			Aus dem Schwartzschen Nullstellensatz folgt $I=K[X_1,...,X_n]$.\\
			Schreibe nun 
			\[\sum_{i=1}^{t}A_i(x_1,...,x_{n+1})f_i+B(x_1,...,x_{n+1})(x_{n+1}g-1)=1\]\\
			Sei $x_{n+1}=\frac{1}{y}$.\\
			Multipliziert man mit einer genügend hohen Potenz von $Y$, so erhält man
			\[\sum_{i=1}^tC_i(X_1,...,X_n,Y)f_i+D(X_1,...X_n,Y)(g-Y)=Y\]
			dabei sind $C_i(X_1,...,X_n,Y),D(X_1,...,X_n,Y)\in K[X_1,...,X_n,Y]$.\\
			Setze $Y=g$, dann ist
			\[g^n=\sum_{i=1}^{t}\underbrace{C_i(X_1,...,X_n,g)}_{\in K[X_1,...,X_n]}f_i\in J\]
			Also ist $g\in\rad(J)$.
		\end{description}
	\end{proof}
\end{document}